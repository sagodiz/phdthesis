\documentclass[12pt,a4paper,twoside]{book}
\usepackage[utf8]{inputenc}
%\usepackage[T1]{fontenc}
\usepackage{setspace}
\usepackage{amsmath,amsfonts,amssymb}
\usepackage{cite}
\usepackage{algorithm,algpseudocode}
\usepackage{setspace}
\usepackage{bibentry}
\usepackage[bottom]{footmisc}
\usepackage[
a4paper,
twoside,
bindingoffset=1cm,
inner=2.5cm,
%outer=2.5cm,
outer=2cm,
top=3.5cm,
%top=3.5cm
bottom=3.5cm,
%headsep=1.2cm
headsep=1cm
]{geometry}
%\usepackage{geometry}
%\geometry{a4paper,twoside,textwidth=6.0in,textheight=8.5in}
\usepackage{charter}
\usepackage[sort,numbers]{natbib}
%\usepackage{nyul_thesis}
\usepackage{booktabs,caption,multicol,hhline,tikz,multirow,array}
\usetikzlibrary{arrows.meta}
\usepackage{colortbl}
\usepackage{arydshln}
\usepackage{url}
\usepackage{subcaption}
\usepackage{caption}
\usepackage{graphicx, multirow, array,amsmath}
\usepackage{tabularx}
\usepackage{hyperref}
\usepackage{enumitem}
\usepackage[acronym, nopostdot]{glossaries}

\usepackage{todonotes}

%%%%%%%%%%%%%%%%%%%%%%%%%%%%%%%%%%%%%%%%%%%%%%%%%%%%%%%%%%%%%%%%%%%
%%%%%%%%%%%%%%%%%%%%%%%% THESIS1 PACKAGES %%%%%%%%%%%%%%%%%%%%%%%%%
\usepackage{listings} %Thesis1
\usepackage{adjustbox} % Thesis1

%%%%%%%%%%%%%%%%%%%%%%%%%%%%%%%%%%%%%%%%%%%%%%%%%%%%%%%%%%%%%%%%%%%
%%%%%%%%%%%%%%%%%%%%%% END OF THESIS1 PACKAGES %%%%%%%%%%%%%%%%%%%%

%%%%%%%%%%%%%%%%%%%%%%%%%%%%%%%%%%%%%%%%%%%%%%%%%%%%%%%%%%%%%%%%%%%
%%%%%%%%%%%%%%%%%%%%%%%% THESIS3 PACKAGES %%%%%%%%%%%%%%%%%%%%%%%%%
\usepackage{xspace}
\usepackage{fancybox,framed}
%%%%%%%%%%%%%%%%%%%%%%%%%%%%%%%%%%%%%%%%%%%%%%%%%%%%%%%%%%%%%%%%%%%
%%%%%%%%%%%%%%%%%%%%%% END OF THESIS3 PACKAGES %%%%%%%%%%%%%%%%%%%%



\newcolumntype{C}[1]{>{\centering\arraybackslash}p{#1}}

% \newcolumntype{C}[1]{>{\centering\let\newline\\\arraybackslash\hspace{0pt}}m{#1}}
\newcolumntype{M}[1]{>{\centering\arraybackslash}m{#1}}
\newcolumntype{N}{@{}m{0pt}@{}}
\setlength{\belowcaptionskip}{10pt plus 3pt minus 2pt}
\DeclareMathOperator*{\argmin}{arg\,min}
\DeclareMathOperator*{\argmax}{arg\,max}

\usepackage{fancyhdr}
\fancypagestyle{plain}{%
	\fancyhead{}%
	\renewcommand{\headrulewidth}{0pt}%
	\renewcommand{\footrulewidth}{0pt}%
	\fancyfoot[C]{\bfseries\thepage}}
\fancypagestyle{mainmatter}{%
	\fancyhf{}
	\renewcommand{\headrulewidth}{0.4pt}%
	\renewcommand{\footrulewidth}{0pt}%
	\fancyhead[LE,RO]{\bfseries\thepage}
	\fancyhead[LO]{\bfseries\nouppercase\rightmark}
	\fancyhead[RE]{\bfseries\nouppercase\leftmark}}
\fancypagestyle{frontmatter}{%
	\fancyhead{}%
	\renewcommand{\headrulewidth}{0pt}%
	\renewcommand{\footrulewidth}{0pt}%
	\fancyfoot[C]{\bfseries\thepage}}

\captionsetup[figure]{labelfont={bf},textfont={it}}
\captionsetup[table]{labelfont={bf},textfont={it}}


%----------------------------------------------------------------------------------------
%	THESIS INFORMATION
%----------------------------------------------------------------------------------------


\newcommand{\thesispointone}{LLM comparison requires a detailed methodology}
\newcommand{\thesispointtwo}{Lesser known datasets provide perfect material for LLM hyper-parameter investigation}
\newcommand{\thesispointthree}{LLMs cannot properly replace human developers in vulnerability fixing on real-world projects}

\newcommand{\thesispointonehu}{Nyelvi modellek összehasonlítása részletes módszertant igényel}
\newcommand{\thesispointtwohu}{A kevésbé ismert adathalmazok tökéletesek a nyelvi modellek hiperparamétereinek vizsgálatára}
\newcommand{\thesispointthreehu}{A nyelvi modellek nem tudják teljesen helyettesíteni az embert az automatikus sérülékenység javításban.}

\renewcommand{\chapterautorefname}{Chapter}

\makeglossaries

\begin{document}
%\UseRawInputEncoding

\setstretch{1.1}

\nobibliography*
\newcounter{cont}

% Title

\thispagestyle{empty}

\newcommand{\dolgozatcim}{Exploring Large Language Models' Capabilities in Software Engineering}

\begin{center}
%	\vfill
	\vspace*{0.25cm}
	
	\begin{spacing}{2}
		{\Huge \textbf{\dolgozatcim}}
	\end{spacing}
	
	\vspace*{2cm}
	
	{\Large PhD Thesis}
	
%	\vspace{2.5cm}
	\vfill
	
	{\Large Zoltán Ságodi}
	
	\vspace{0.25cm}
	
	{\Large Supervisors: \\ \vspace{0.5em} István Siket, PhD\\ \vspace*{1em}Ferenc Rudolf, PhD}
	
%	\vspace{2cm}
	\vfill
	
	{\large Doctoral School of Informatics}
	
	\vspace{0.25cm}
	
	{\large Department of Software Engineering}
	
	\vspace{0.25cm}
	
	{\large Faculty of Science and Informatics}
	
	\vspace{0.25cm}
	
	{\large University of Szeged}
	
%	\vspace{4cm}
	\vfill
	
	\includegraphics[width=0.3\linewidth]{Figures/szte_logo}
	
	\vfill
	
	{\large Szeged
		
		 \today}
	
\end{center}
 


\newpage
\thispagestyle{empty}
\mbox{}

% \newpage
% \thispagestyle{empty}
% \begin{flushright}
% \parbox[t]{10cm}{\textit{"Scientists study the world as it is,}\\ \textit{\hphantom{"}engineers  create the world that never has been."}} \\
% \vspace{1em}
% \parbox[t]{5cm}{\small(Kármán Tódor)}
% \end{flushright}

% \newpage
% \thispagestyle{empty}
% \mbox{}

\frontmatter
\renewcommand{\chaptermark}[1]{\markboth{#1}{}}
\renewcommand{\sectionmark}[1]{\markright{\thesection\ #1}}
\pagestyle{frontmatter}

\tableofcontents

% Empty pages

\newpage
\thispagestyle{empty}
\mbox{}

% \newpage
% \thispagestyle{empty}
% \mbox{}

% Main

\mainmatter
\renewcommand{\chaptermark}[1]{\markboth{#1}{}}
\renewcommand{\sectionmark}[1]{\markright{\thesection\ #1}}
\pagestyle{mainmatter}

% List of Figures
\listoffigures
% List of Tables
\listoftables

% List of Acronyms
\newacronym{gcd}{GCD}{Greatest Common Divisor}
\newacronym{lcm}{LCM}{Least Common Multiple}

% Acronym List
\printglossary[type=\acronymtype, title=Abbreviations]

% Intro:
\chapter{Introduction}

%Software Engineering is an ever growing and evolving field.
%Not only are there many and many new fields, but the technology behind it is a constantly and rapidly changing topic.
%Since the manually created software, We got to a point where the Integrated Development Environments (IDEs),- which are a huge help for developers by the syntax highlight, navigation, formatting and so on features,- are included in every developers arsenal.
%These tools were created for the convinience of the  developers, although many has failed and they are constantly changing by the requirements of the developer community.

%Such tools are the recently introduced Artificial Intelligence (AI) based Large Language Models (LLMs) which are currently transforming our jobs as software engineers.
%Similarly, these tools and models require adaptation to developer requirements and also, the developers must discover what these tools are capable of, what are the main tasks which can be performed on a higher quality or same quality but faster by using them.

%%%%%%%%%%%%%%%%%%%%%%%%%%%%%%%%%%%%%%
Software engineering is a continuously growing and evolving field.
Not only do new subfields emerge constantly, but the underlying technologies are also changing at a rapid pace.
From the early days of manually written software, we have progressed to a point where Integrated Development Environments (IDEs)—which provide substantial support through features such as syntax highlighting, code navigation, and automated formatting—have become standard tools in every developer’s arsenal.
Although these tools were created for developer convenience, many have failed over time, and those that remain continue to evolve in response to the changing requirements of the developer community.

IDEs rely on a variety of analysis techniques, with source code analysis being a central component.
While some features may appear straightforward, such as navigating from a caller to a callee, they often depend on extensive research, such as the construction of call graphs.
Building call graphs is a complex task, as demonstrated in several of our studies~\cite{cg1, cg2, cg3}.

Just as a small subset of IDE features is underpinned by a substantial research foundation, newly developed tools also require a solid research base to ensure their effectiveness and reliability.
Such tools are the recently introduced Large Language Models (LLMs), which are currently reshaping the daily work of software engineers.
As with earlier tools, these models must adapt to developer needs, and developers, in turn, must explore their capabilities to determine which tasks can be performed at higher quality—or at the same quality more efficiently—through their use.
%%%%%%%%%%%%%%%%%%%%%%%%%%%%%%%%%%%%%

%Developers are facing many difficulties such as the need of change in coding habits and the way how creative work becomes mostly a review work by the appliance of AI models.
%In this work another difficulties are highlighted which are mostly considered from a technical point.
%With LLMs popping out daily developers cannot choose the right models for their work.
%Experiencing with the most hyped models takes time and could miss potential optimal models.
%It is a problem, that the newly presented tools or models do not have a common base of quality measurement, a way of comparing them, therefore, one model cannot be confidently preferred over another.

%Another technical problem is that although a few benchmarks are available, developers cannot completely trust in them, as the benchmarks either consist of smaller non-real-life scenarios or they are prone to overfitting on the benchmarks for better evaluation.
%Developers must know, what is the real current level of LLMs in a real-life scenario.

%Yet another problem is that these models are configured by many hyper-parameters, which provide a great customization to developers once they select the well-performing model for their usage, however, they cannot know how to configure for their task the model even if there is any pattern in configuring various models.
%Although generic information is available on many of such hyper-parameters in a larger scale, but as a daily user developers might not benefit or even notice these differences.

Developers today face numerous challenges, including the need to change established coding habits and adapt to a workflow in which creative work is increasingly transformed into review-oriented work through the use of AI models.
This work also highlights additional challenges that are primarily technical in nature.
With new LLMs appearing almost daily, developers struggle to select the most suitable model for their tasks.
Experimenting with the most popular models is time-consuming and may lead them to overlook potentially optimal alternatives.
A further issue is that newly introduced tools and models lack a common, standardized framework for quality measurement and comparison; as a result, developers cannot confidently prefer one model over another.
Although various benchmarks are available, those mostly measure the functionality of the generated code.
Even with versatile benchmarks, developers have to find those benchmarks and often evaluate the models on them as most of the models are not evaluated from the perspectives of source code quality, security, or other non-functional criteria.

As benchmarks are mentioned, another technical challenge is that, although some benchmarks are available, developers cannot fully trust them.
These benchmarks often rely on small, non–real-world scenarios.
Such benchmarks do not provide actual information about how models would perform when applied in actual development.
It is also a problem that these benchmarks are prone to overfitting, as models may be optimized specifically for better benchmark performance.
As model trainers know, which benchmarks are checked for quality measures, those benchmarks could be placed multiple times in the training set in order to seemingly increase the performance.
Consequently, developers need reliable insight into the true real-world capabilities of LLMs.

An additional problem is that these models are controlled by numerous hyper-parameters.
While these offer extensive customization once a well-performing model has been selected, developers often lack guidance on how to configure the model effectively for their specific tasks or whether consistent configuration patterns exist across different models.
Although general information about many hyper-parameters is available at a broader level, everyday users frequently cannot benefit from, or may not even notice, their impact in practice.

%This dissertation discusses the above mentioned three challenges in order.
%Although these challenges can be addressed through multiple fields of software engineering, source code generation, source code quality, and source code security are the main topics.
%Generation and quality is the base of all workflows as without code, there is no software and quality control is a commonly performed task in larger projects.
%Security on the other hand is not a new topic, however, definitely getting more and more relevant as the security incidents are growing day-by-day.

While these challenges may be approached from multiple subfields of software engineering, the primary focus of this work is on source code generation, source code quality, and source code security.
Source code generation and quality form the foundation of all software development workflows, as software cannot exist without code, and quality assurance is a routinely performed and essential activity in large-scale projects.
Source code security is a well-established research area; however, its importance continues to increase as the frequency and impact of security incidents grow steadily over time.

This dissertation addresses the three aforementioned challenges through three thesis points, see \autoref{tbl:thesispoints}. The \textit{first thesis point}, presented in \autoref{chapter_2} focuses on the problem of LLM comparison, outlining the essential evaluation steps and the main pitfalls. The \textit{second thesis point}, described in \autoref{chapter_4} investigates hyper-parameter usage and overfitted benchmarks, introducing a dataset to show that temperature remains an underexplored factor in mastering LLM-based code generation. Finally, the \textit{third thesis point}, detailed in \autoref{chapter_3} evaluates GPT-4 on vulnerability fixing in a real-world context, providing a clearer view of current model capabilities.

\begin{table}[!h]
	\begin{tabular}{|l|p{0.8\linewidth}|c|}
		\hline
		\textbf{\#} & \textbf{Thesis}       &  \textbf{Chapter} \\ \hline
		1           &  \thesispointone.      &  3                \\ \hline
		2           &  \thesispointtwo.      &  4                \\ \hline
		3           &  \thesispointthree.    &  5                \\
		\hline
	\end{tabular}
	\caption{List of thesis points}
	\label{tbl:thesispoints}
\end{table}

% Common background / Common intro
\chapter{Background}
\label{chap:background}

%Source code generation is the process of automatically producing code based on high-level abstractions defined in domain-specific languages (DSLs)~\cite{codegen_dsl} or other declarative languages.
%The objective of source code generation is to reduce the amount of low-level, repetitive coding work that developers need to do, enabling them to focus on higher-level tasks and reducing the likelihood of errors and bugs.
%While source code generation has a long history, from the 70's~\cite{old_code_gen} even to the 2010s~\cite{dsl_recent}, recent advances in LLMs~\cite{codex} allow developers to express required program code using natural language definitions.

In recent years, Large Language Models (LLMs) have gained significant attention across various domains, from psychology~\cite{rathje2023gpt} to medicine~\cite{cheng2023artificial}, but LLMs look highly promising within software engineering, too.
LLMs primarily deal with textual and linguistic elements, but since source code can also be considered a form of language, it is unsurprising that these models can be employed for tasks involving source code. 
Most of the LLMs are based on the transformer architecture~\cite{attention} that includes an encoder and a decoder component.

%%% possible CUT
An encoder-based model is BERT~\cite{bert} or CodeBERT~\cite{codebert}, which is designed specifically to work on code.
BERT models work with masked tokens, and predict them based on the previous and following text, therefore, the generation task is not natural for such models.
The other part of the transformer architecture is decoders, which are used in the Generative Pre-trained Transformer (GPT) architecture.
One such model, specifically trained on source code, is Codex~\cite{codex}, which is based on the GPT-3 architecture.
The latest model publicly available in this lineage is GPT-4.

Some of the most promising areas within software engineering that could leverage these models is program synthesis also known as program generation, and automatic program repair, as it involves generating new ``text'', the source code.
Source code generation is the process of automatically producing code based on high-level abstractions defined in domain-specific languages (DSLs)~\cite{codegen_dsl} or other declarative languages.
The objective of source code generation is to reduce the amount of low-level, repetitive coding work that developers need to do, enabling them to focus on higher-level tasks and reducing the likelihood of errors and bugs.
While source code generation has a long history, from the 70's~\cite{old_code_gen} even to the 2010s~\cite{dsl_recent}, recent advances in LLMs~\cite{codex} allow developers to express required program code using natural language definitions.

Automatic program repair aims to ease the developer tasks by assisting in the correction of faulty code.
Fixing a program is more challenging than creating new ones, as the developer has to understand the mindset of the original developer and devise an alternative approach as the original one was faulty.
Leveraging automated systems, e.g. Language Models, to fix these problems, developers get more time to do in the creative field.
This automatic fixing is especially critical in the field of vulnerabilities, where a fix might cannot wait for a developer to understand the error and fix it, it needs immediate fixing.

The term software vulnerability is defined as a security flaw or weakness hidden in the software that could be exploited by a malicious user, therefore, risking the safety of sensitive data, the integrity of the software or even financial loss.
\todo{Vulnerability connect to Quality}
\todo{Functional: describe}
\todo{Non-functional: describe}
% Thesis:
\chapter{Methodology for Code Synthesis Evaluation of LLMs Presented by a Case Study of ChatGPT and Copilot}
\label{chapter_2}

In this chapter, we present a methodology designed to support developers who intend to compare Large Language Models (LLMs).
By following the outlined steps, developers can ensure that all key aspects of LLM comparison, often dispersed across various studies, are systematically addressed.
The methodology synthesizes elements from multiple research approaches to provide a comprehensive framework.
For clarity and practical relevance, the chapter also includes a case study demonstrating how the methodology can be applied in practice, giving readers a concrete example of how to conduct an LLM comparison.

We consider this task a significant step as developers are already using multiple LLM-based tools and they will continue integrating them into their daily workflows due to the substantial advantages they provide in software development.

\vspace*{1cm}
The author’s contributions in the first thesis point consist of \textbf{(i)} collecting methodologies commonly used in the literature, \textbf{(ii)} integrating them into a single, comprehensive methodology, \textbf{(iii)} identifying an evaluation benchmark that satisfies multiple conditions for valid assessment, \textbf{(iv)} evaluating the language models, \textbf{(v)} analyzing the generated source code, \textbf{(vi)} designing, and \textbf{(vii)} organizing the human evaluation.

\section{Introduction}
\label{l:introduction}

%In everyday life, Large Language Models (LLMs) are becoming increasingly popular and are used for various purposes \cite{llm_usages}.
%It is the same in software engineering as developers leverage LLMs' Natural Language Processing (NLP) capabilities.
%LLMs are used for a variety of tasks such as source code comprehension, documentation generation, or various testing tasks, e.g. test case generation.
%In recent years, source code generation has emerged as a key component of LLM utilization.

%Source code generation is the process of automatically producing code based on high-level abstractions defined in domain-specific languages (DSLs)~\cite{codegen_dsl} or other declarative languages.
%The objective of source code generation is to reduce the amount of low-level, repetitive coding work that developers need to do, enabling them to focus on higher-level tasks and reducing the likelihood of errors and bugs.
%While source code generation has a long history, from the 70's~\cite{old_code_gen} even to the 2010s~\cite{dsl_recent}, recent advances in LLMs~\cite{codex} allow developers to express required program code using natural language definitions.

The use of LLMs can significantly reduce the time spent on coding, although it also comes with certain limitations.
While having more time for abstract tasks and generating code with low cost is advantageous, in software development, the quality of the source code is at least as important as the time spent on development.
There are multiple factors affecting source code quality, and there are several ways to measure it.
One approach is using static code analysis~\cite{static_analysis_poc}, which allows for the code to be analyzed without executing it, without any tests, or even without a compiled binary file.
The results of the analysis can be coding rule violations, detected code duplication, or code metrics.
Although static analysis is a powerful and widely used technique, it is not the only way to measure source code quality.
Code quality is often defined from a human perspective, and metrics may not adequately reflect this~\cite{why_metrics_not_enough}.

The previously mentioned aspects are the keys to successfully using LLMs in software engineering, although it is not shown how to choose the best LLM for a given task.
There are already studies that evaluated a chosen LLM~\cite{codex, copilot_code_evaluation} from various perspectives, but evaluating only one LLM might mean missing out on a better model.
Evaluating all the possible models by the previously described methods takes too much effort as they mostly require a lot of human actors and effort during the evaluation.

In this paper, we investigate different aspects of the quality of LLM generated source code.
It is important to know how to compare the generated source code; what are the main factors in which they differ most; and what aspects a developer should consider when choosing a development assist tool.

We do this by giving a simple methodology and showing its usefulness through an actual evaluation of two LLMs capable of source code generation: ChatGPT and Copilot.
We examine how these models work with natural language task definitions, how the generated code (for C++ and Java) behaves, and what the quality of the generated code is like from multiple perspectives.
We decide which model is better in every phase of the evaluation and finally we support our decisions by human reviewing, similar to other papers~\cite{copilot_code_evaluation, copilot_code_evaluation_3}.
The research questions addressed in this paper are:
\begin{itemize}
	\item \textbf{RQ 1: How does LLM-generated source code score in terms of source code quality?}
	\item \textbf{RQ 2: Is the generated source code accepted by developers?}
	\item \textbf{RQ 3: What aspects should be considered when choosing LLM-based generative tools?}
\end{itemize}

The main goal of this paper is to highlight the importance of proper comparisons, not to introduce optimized prompting or selecting up-to-date models as those are changing dynamically.
The latter might be decided using the provided methodology with actualized data and models.

This topic relates to RQ3, in which we discuss the main components of a decent comparison, but it cannot be discussed without the previous two RQs.
Our first RQ is in the first place because humans will eventually operate with source code, where quality still matters as people prefer to work with good quality code.
In our use case, we found that there are models that are capable of generating code with good quality.
Our second RQ is still required in our methodology, as various problems can be solved in various ways and programming manners.
Developers usually have their favorable style and way of thinking.
The use case showed that developers tend to accept the generated code conditionally, meaning that the generated code is not perfect but can be used by developers and it helps their work.


The paper's subsequent sections describe related works in Section \ref{l:related_work}.
We present our methodology in Section \ref{l:methodology}, and in Section \ref{l:results}, we compare two models in an actual case study and discuss the results.
Section \ref{l:threats_to_validity} addresses threats to validity.
Finally, we summarize the paper in Section \ref{l:summary}.
\section{Related work}
\label{l:related_work}

In this work, we compare LLMs based on the quality of their generated source code from different aspects.
Both LLMs and quality assurance are already integral parts of software engineering both in academic~\cite{academic_sonar_usage, academic_sonar_usage2} and industrial fields~\cite{sonarqube_good_active}.% to some extent.


There are various papers describing static analysis techniques and tools~\cite{static_analyzer1, static_analyzer2}, articles that compare these tools~\cite{static_analyzer_comparison}, and even papers that use machine learning for static analysis tasks\cite{rex}.
Static analysis is performed on the source code (or the generated byte code in the case of Java-based languages) thus it does not require running the code.
The code is mostly written by developers, but nowadays source code generation and synthesis allow developers to create code without any or with only a minimal amount of human interaction.
As source code synthesis has a long history~\cite{early_program_synthesis}, those methods used to rely on mathematical derivations and various rules.
Nowadays, neural networks are taking over these methods since LLMs are available such as Codex~\cite{codex}, which supports GitHub's Copilot, and the recently introduced ChatGPT.
Both models are based on GPT-3~\cite{gpt3}.
There are also models based on Google's LaMDA~\cite{lamda}, such as Bard.
These models are generative models~\cite{generative_models}, meaning they are capable of producing text from an input, usually natural language text, and the output can be source code.

As quality assurance and static analysis can be performed on the source code it is obvious that we must examine how the generated code performs.
There are works that evaluate LLMs.
Vaithilingam et al.~\cite{copilot_code_evaluation} evaluate Codex via the Copilot plugin and show the effects of such a tool on the development itself.
They included 24 developers and asked them to create a program using Intellisense and Copilot.
Dominik et al.~\cite{copilot_code_evaluation2} did a comparative study on Copilot and Genetic program synthesis.
They used the same benchmark~\cite{codegen_benchmark} we did; however, they added extra prompting by defining a function’s signature, thereby optimizing the prompt for Copilot.
Their main goal was to compare two methodologies for program synthesis and did not measure the quality of the synthetized code.
In the work of Madi~\cite{copilot_code_evaluation_3}, the readability is measured by static code analysis but they do not check further qualities and they rely heavily on human annotators; 21 people to be specific.
These works show that LLMs are capable of generating source code that is similar to human-written code in terms of readability and maintainability.
There are even works~\cite{copilot_code_similar_performance} showing the performance of the generated code matches human-written code.
Although there are works that evaluate not only one LLM~\cite{fix_program_synthesis_models}, that highlight the drawbacks of LLMs, they do not show a proper way to compare them, therefore, they give no suggestion for how the better model could be chosen.
These works mostly rely on a larger number of annotators or human reviewers, which is an expensive resource both in the academic and especially in industrial field.

LLMs are also used to fix vulnerabilities in source code~\cite{codex_as_fix}, which might suggest that these models are free from vulnerabilities, however, there are multiple works that show Copilot being prone to generate vulnerable code~\cite{copilot_code_vulnerable, copilot_code_vulnerable2, copilot_code_vulnerable3}.
ChatGPT is not free from vulnerabilities either as it is shown in the work of Khoury et al.~\cite{chatgpt_vulnerable}.
In their work, it is shown that ChatGPT is capable of fixing the vulnerabilities by further prompting, which leads us to the problem of optimal prompting.
\vspace{-5px}
\section{Methodology}
\label{l:methodology}

In this section, we describe a methodology of how LLMs can be compared and which are the main factors to consider.
We describe four phases: selecting the right prompt, checking functional validity, checking technical quality, and human evaluation.
The main goal of this paper is to emphasize the importance of each step.
The main steps and key elements are shown in Figure~\ref{fig:overview}.

\begin{figure}[htbp]
	\centering
  \includegraphics[width=0.99\linewidth]{Chapters/Thesis1/fig/overview.png}
	\caption{ Overview of the methodology with key elements.
	\label{fig:overview}}
\end{figure}
\vspace{-5 px}
\subsection{Selecting the right prompt}

First of all, the prompt provided for the models must be carefully selected, which presents a challenge as none of the compared models should be favored by optimized prompting.
The difficulty and level of detail of the problems' descriptions are the main factors to consider when working with NLP-based systems.

The difficulty of the problem greatly affects the quality and usability of the generated source code.
A simple ``hello world'' program provides no meaningful information about the models' capabilities.
When attempting to generate a complete project with multiple files and perform complex algorithms, the model may encounter difficulties that prevent it from generating proper source code.
This way the model's capabilities stay hidden.
The input task must have the appropriate level of difficulty to measure the difference between models.

The level of detail in the problem description is also a crucial factor.
Providing too much information gives no indication of whether the model can identify complex connections between elements within the problem.
Providing too little information may result in the models not generating a solution, or the solutions may be too different, which does not provide any additional information for our comparison.
The detail of the description must be at a level that challenges the models, but it should be clear from a description what the solution should be.

Another crucial factor in selecting the right prompt is prompt engineering.
Although LLMs are designed to interact with natural language, therefore, simple commands, such as ''Generate'', or ''Write code'' can lead to the desired results, various prompting techniques could optimize these results.
Such techniques must be taken into consideration, paying attention to the latest prompt engineering results relating to the models; e.g., Github's advice ~\cite{github_prompt_engineering}.
Denny et al.~\cite{codex_prompting} show that NL prompt changes affect the resulting source code.
It can be performed with prompt patterns such as the work of White et al.~\cite{chatgpt_prompt_design}.

\subsection{Checking functional validity}

Secondly, after using the right prompts, the LLM-generated source code has to be syntactically correct and fully functional.
To ensure this, functional testing has to be performed which must include various test cases both for selected edge cases and non-edge cases as this shows how robust the solution is.
To perform this testing, proper test cases need to be available with input and required output pairs.
Handling the input and output may be different in the code generated by different LLMs, so one might have to modify the generated code for automated tests, and the modifications must not have any effect on the functional code parts.

\subsection{Checking technical quality}

Thirdly, the generated code has to pass some quality requirements, which can be measured by static analysis.
The static analysis could include various techniques, rules, and metrics.
All of these elements depend on what context the generated code is used in.
An optimal solution would use the same pipeline for static analysis as the actual development.
If no such configuration is available, static analyzers should be configured according to best practices, e.g. by considering the work of Kaner and Bond~\cite{why_metrics_not_enough}.

\subsection{Human evaluation}

Finally, the most expensive part is human evaluation.
This phase is optional as human resources are expensive, however, one can be fully confident in the results after human reviews.
In contrast to other works, we do not suggest that a large set of developers should evaluate the generated source code as this phase is just additional to the previous phases.
During human evaluation, reviewers must not know about the result of the functional test phase and technical test phase.
Reviewers should be asked to evaluate the generated source code on an even-numbered scale, forcing them into decision making.

Although the methodology steps seem trivial, many of the related works lack this evaluation methodology.
Vaithilingam et al.~\cite{copilot_code_evaluation} and Dominik et al.~\cite{copilot_code_evaluation2} did evaluate program synthesis, but the results were conducted on functional results and the experience of the users, while there were no source code quality, security, or performance comparisons.
Although, in the work of Madi~\cite{copilot_code_evaluation_3}, they evaluate the static properties of the source code, they lack the proper description of functional properties and the way of prompting during the iterative development.
Related works lack the proper evaluation steps on the very basic steps, therefore, an abstract description must be given in order to properly compare source code synthetization methods.
\section{Case study}
\label{l:results}

In this section, we compare two LLMs, namely Copilot and ChatGPT using the methodology presented in Section \ref{l:methodology}.
This case study is a simplified version of an actual comparison in order to show how to evaluate actual models.
An actual evaluation must be done on a benchmark that fits the required programming qualities the best, such as graphical pipeline programming, high-end servers, low-latency requirements, etc.
During an actual evaluation, it is not the default SonarQube settings that should be used, but rather the available unique rule sets, configurations or even other, already-in-use static analyzers apart from SonarQube.

Selecting Copilot and ChatGPT was not based solely on their popularity in software engineering tasks but on their applicability.
They have great availability through APIs and UIs and, therefore, are likely to be used in actual projects.
Although there are other source code specific LLMs, such as CodeT5- or BERT-based models, which can be evaluated with our methodology, due to their limited availability through APIs they are less likely to be integrated.
We provide in the supplementary material the generated source code and the applied diffs to them.
We provide the scripts and random numbers we used during our research to run the benchmark evaluations.
We also include the validated results for static analysis and the raw numbers from the human evaluation phase.
To this case study we provide an online appendix available on Zenodo: \href{https://zenodo.org/record/8123647}{https://zenodo.org/record/8123647}.


\subsection{Setup of the case study}

To present the usage of our methodology we had to consider various factors such as which models to compare and what programming language to evaluate these models on.
In the following sections, we describe which models were selected and why.
We also describe our programming language selection.

\subsubsection{Utilized LLMs}

For our comparison, we utilized Copilot, a Visual Studio Code extension\footnote{https://marketplace.visualstudio.com/items?itemName=GitHub.copilot} and ChatGPT\footnote{Version Dec 15 - https://chat.openai.com/}, which we accessed through its online web interface.
Both models were employed to generate input data on the first week of January 2023.
As these models can interact with developers, there is no definitive answer to any given problem.
Copilot continuously generates code snippets as the developer writes code, while ChatGPT can modify the code if the developer provides additional conditions or a more precise description of the task.
To ensure a fair comparison, we used the same problem description for both models and did not interact further with them.
This means that we accepted the first generated code for a given input.
Using Copilot we ensured that the workspace is clean so the results were not altered.
Using ChatGPT we opened a new chat windows for every task in order to have a clean context.

\subsubsection{Programming language selection}

In terms of programming languages, we had to choose from multiple languages due to limited manpower during the comparison evaluations.
We selected languages that are widely used in academic and industrial fields.
Our first language of choice was C++, as it is easy to program using classes and an object-oriented programming (OOP) style or using only functions in a non-OOP style.
Additionally, C++ provides a wide range of tools that are easy to use but can also be used in a complex manner.
Furthermore, C++ is notorious for its memory management issues, making it a good choice for our code quality tests.

To mitigate the potential effects of language selection on our results, we decided to use another language that is a managed language where memory management issues do not arise and the language primarily supports OOP.
Our second language of choice was Java, which is a widely used programming language, and it is well-known for having a lot of boilerplate code.
Asking for Java code requires the model to generate every small detail to have a working solution, making Java a good choice for our second language.

To note, C++ and Java are both compiled languages, which is an additional condition against the generated sources as the code must compile.
\subsection{Selecting the right prompt}

The first step in our methodology is selecting the right prompts with right difficulty and level of detail.
As this task is very complicated we decided to use a program synthesis benchmark\cite{codegen_benchmark}, which includes 25 programming tasks from various sources.
We provided the task descriptions from this benchmark as the prompts with a fixed prefix e.g. ''Generate a <LANGUAGE> code to solve the problem!'', where LANGUAGE is C++ or Java.
A C++ prompt example is shown in Listing~\ref{lst:prompt}, where the first sentence is the prefix with the selected language and the remaining part is the text extracted from the benchmark.
\begin{lstlisting}[basicstyle=\small, label=lst:prompt, caption=Example prompt for a C++ solution for the basement problem]
	Generate a C++ code to solve the problem!
	Given a vector of integers, return the
	first index such that the sum of all
	integers from the start of the vector to
	that index (inclusive) is negative
\end{lstlisting}
We decided to use simple prompts as the case study is just a way of showing our methodology in action.
The selected prompts are likely to be used by a software developer who does not have prompt engineering experience.
The person evaluating will not necessarily use these exact prompts but instead use specific techniques that may not even exist yet. They might be using a model that needs a special kind of prompting (e.g. they have to provide elements in JSON because that is how the model was trained).
Therefore, since we cannot provide for all current and future use cases, optimizing the prompts was not something we were aiming for.

\subsection{Checking functional validity}
\label{l:benchmark_evaluation}


To evaluate the program's functional validity, we used the benchmark's input and output values for each task.
These values were divided into two disjunct parts:
edge cases, which included various tests for each task that were considered edge cases for the problem, and random cases.
The number of edge cases varied from 5 to 40 depending on the task.
We executed all available edge case tests.
The random test cases included one million input-output pairs per task.
We randomly\footnote{See supplementary material referenced in Section \ref{l:results} - \url{https://zenodo.org/record/8123647}} selected 10,000 input-output pairs from the pool of one million random cases for every task.
The selected input-output pairs were used as random cases for later evaluations.

However, the generated source code was not directly usable for executing the tests because it might or might not read inputs from the console or write results to the console.
Therefore, we had to further modify the code to unify the inputs and outputs by using the same way of reading input from the console and writing output to the console.
It also included that additional text answers were removed and only the required parts were kept.
Only the input and output lines were modified, not the functional code.
All source code differences are available\footnote{See supplementary material referenced in Section \ref{l:results}}.

Java source code did not require additional modifications.
In the case of C++, ChatGPT generated solutions mostly compiled with no error, but we had to subsequently correct the generated programs for 3 benchmark tasks manually:
\begin{itemize}
	\item \textit{leaders}: include \texttt{climits}, \texttt{algorithm} were missing.
	\item \textit{cut\_vector}: \texttt{climits} include was missing.
	\item \textit{vector\_distance}: \texttt{limits} include was missing.
\end{itemize}
All these errors are related to missing includes, which could be solved by a simple added line.
There were no problems during the linking phase after the missing includes were added.

Copilot also generated mostly compilable C++ code, but there were also some problems that we had to handle:
\begin{itemize}
	\item \textit{leaders}: It generated a vector variable with the same name as a previously declared function, thus resulting in a compile error as the function call ended up being a call-operator of the declared variable. Renaming the variable solved the error.
	\item \textit{indices\_of\_substring}: The generated source did not contain a \texttt{main} method, but only a function containing the solution, thus resulting in a linker fault. This problem can be solved by using a dummy \texttt{main} method.
	\item \textit{twitter}: The generated source did not contain a \texttt{main} method. We used the same solution as for \textit{indices\_of\_substring}.
\end{itemize}

Once the programs were ready to run and the input and output values were unified, we were able to compare the desired and actual output values.
We performed a strict comparison, meaning we compared each character in the output values.
The benchmark also included floating-point values with large precision as outputs, so numeric errors could lead to different values even with correct functional code.
To address this issue we compared floating-point values with an epsilon value, and differences smaller than the given epsilon did not count as differing values.
The epsilon value for comparison was 0.00001, which was determined by investigating the differences between programmed computations and closed formula computations for the tasks when it was possible.
We use these results as epsilon evaluation results.

Tables~\ref{t:tests-cpp}  and \ref{t:tests-java} present the performance results for C++ and Java, displaying the pass ratio for each model on every task.
The tables include edge case, random case tests and provide results for exact matching, as well as epsilon matching, which was included to accommodate comparisons involving floating point numbers.
Furthermore, the ``Diff'' columns in the tables provide the difference in pass ratio between the two models on the exact and epsilon matching.
This difference value indicates whether Copilot or ChatGPT performed better.
Specifically, a negative difference value indicates better performance from Copilot, while a positive value suggests better performance from ChatGPT.
The cells are colored, and the bluer a cell is, the better Copilot performed;
the more orange a cell is, the better ChatGPT performed.

Later on, we use the epsilon results as they are not altered by the floating point comparison error.
Although, the results are very similar in exact and random comparisons.
There are a few cases where epsilon comparison resulted in a larger difference:

\begin{itemize}
	\item C++ - bouncing\_balls
	\item C++ - snow\_day
	\item Java - bouncing\_balls
	\item Java - vector\_distance
	\item Java - snow\_day
\end{itemize}

The tasks bouncing\_balls and snow\_day are tasks where the generated source code contains iterative computation with floating-point variables, therefore, the floating-point error is not a surprise.
The task vector\_distance includes large dimensional vectors with floating point values, therefore, the floating-point error could cause differences in the outputs.

\subsubsection{C++}

During the edge case testing phase (see Table~\ref{t:tests-cpp}), a number of solutions encountered runtime errors.
Specifically, the ChatGPT model faced runtime errors in 80\% of the test cases for the solve\_boolean task.
After conducting a thorough investigation, we determined that an incorrect operator ordering was the root cause of the issue, which subsequently impacted the entire testing process whenever an operator was encountered.
On the other hand, the Copilot model only experienced a single instance of runtime error during edge case testing, which occurred with no input.

Despite these challenges, ChatGPT achieved pass rates of 100\% with epsilon evaluation on 18 out of 25 tasks, while Copilot achieved pass rates of 100\% with epsilon evaluation on 13 out of 25 tasks.
Notably, the models obtained identical pass-rates on 14 out of 25 tasks.
However, on average, Copilot outperformed ChatGPT with a pass-rate of 54\% on three tasks, with a standard deviation of 36.14, while ChatGPT performed better with an average pass-rate of 58\% on eight tasks, with a standard deviation of 35.98.

Although edge cases are an integral part of software testing, it is important to investigate how the generated sources perform with a larger input size.
On the random test set (see Table~\ref{t:tests-cpp}), ChatGPT managed to achieve a 100\% pass ratio on 19 tasks, whereas Copilot achieved a 100\% pass ratio on 13 tasks.

Examining the epsilon differences, ChatGPT exhibited superior performance with an average pass-rate of 72.88\% on eight tasks, with a standard deviation of 35.92.
Likewise, Copilot performed better on two tasks, with an average pass-rate of 51\% and a standard deviation of 69.3.
Based on the deviation, it is evident that ChatGPT outperforms Copilot in a more consistent manner.

Upon closer examination of the data, it appears that performing random tests instead of edge cases did not unconditionally increase the pass ratio.
In fact, it resulted in a decrease in a few tasks.
Solutions generated by ChatGPT with lower pass rate on random test cases were:
\textit{bouncing\_balls,
	bowling,
	dice\_game,
	solve\_boolean,
	vector\_distance}.

Only on one of these tasks did the ChatGPT-generated source code reach 100\% during edge case testing.
Since tasks with a pass-rate of less than 100\% could potentially worsen by increasing the number of erroneous cases, the task that originally reached the pass rate of 100\% (\textit{vector\_distance}) was of particular interest.
Upon further investigation, we discovered that the discrepancies in this solution were caused by floating-point errors exceeding our epsilon value.

Solutions generated by Copilot with lower pass rate on random test cases were:
\textit{bouncing\_balls,
	bowling,
	dice\_game,
	find\_pair,
	leaders,
	snow\_day,
	vector\_distance}.

In the case of Copilot, \textit{vector\_distance} was the only task that reached 100\% pass rate previously and lowered in the random case.
This disparity can be attributed to discrepancies in floating-point values that exceeded our epsilon value.
Notably, the values changed in unison for ChatGPT and Copilot relating to this task, and the generated solutions were nearly identical, save for the fact that Copilot lacked vector dimension validation, which was not an explicit requirement of the task.

We looked into how the models performed in both edge and random case testing, and also how the pass-rates decreased.
However, another important factor to consider in functional testing is whether the results remain similar or get better with random testing.
If the results are similar during random testing, it suggests that the edge cases do not negatively impact the pass-rate, and thus the solution is generally more robust.

We found that ChatGPT performed better on five tasks on the random case than on edge case.
The average difference was 21.4\% with a standard deviation of 8.44.
Copilot performed better on seven tasks, with an average of 39\% on the random case with a standard deviation of 21.56.
Since using random test cases mainly affected code that did not achieve a 100\% pass rate in the edge case scenario, and using random tests has increased the pass rate especially for Copilot, we conclude that ChatGPT is more robust.

Regarding C++, ChatGPT performs slightly better on both edge case and random selected testing using epsilon-matching.


\begin{table*}[!h]
	\resizebox{0.99\textwidth}{!}{
	\begin{tabular}{|p{2.9cm}|rr|r|rr|r||rr|r|rr|r|}
		\hline
		& \multicolumn{3}{c|}{Edge - Exact}                                                                    & \multicolumn{3}{c||}{Edge - Epsilon}                                                                                    & \multicolumn{3}{c|}{Random - Exact}                                                                       & \multicolumn{3}{c|}{Random - Epsilon}                                                                                   \\ \hline
		Task               & ChatGPT                          & \multicolumn{1}{l|}{Copilot}                          & \multicolumn{1}{l|}{Diff}   & \multicolumn{1}{l}{ChatGPT}      & \multicolumn{1}{l|}{Copilot}                          & \multicolumn{1}{l||}{Diff}   & \multicolumn{1}{l}{ChatGPT} & \multicolumn{1}{l|}{Copilot}                 & \multicolumn{1}{l|}{Diff}    & \multicolumn{1}{l}{ChatGPT}      & \multicolumn{1}{l|}{Copilot}                          & \multicolumn{1}{l|}{Diff}    \\ \hline
		basement           & \textcolor{teal}{100\%}          & \multicolumn{1}{r|}{\textcolor{teal}{100\%}}          & \cellcolor[HTML]{FFFFFF}0   & \textcolor{teal}{100\%}          & \multicolumn{1}{r|}{\textcolor{teal}{100\%}}          & \cellcolor[HTML]{FFFFFF}0   & \textcolor{teal}{100\%}     & \multicolumn{1}{r|}{\textcolor{teal}{100\%}} & \cellcolor[HTML]{FFFFFF}0    & \textcolor{teal}{100\%}          & \multicolumn{1}{r|}{\textcolor{teal}{100\%}}          & \cellcolor[HTML]{FFFFFF}0    \\ \hline
		bouncing\_balls    & \textcolor{red}{0\%}             & \multicolumn{1}{r|}{\textbf{80\%}}                    & \cellcolor[HTML]{6E9EEC}-80 & 10\%                             & \multicolumn{1}{r|}{\textbf{80\%}}                    & \cellcolor[HTML]{80AAEE}-70 & \textcolor{red}{0\%}        & \multicolumn{1}{r|}{\textbf{5\%}}                     & \cellcolor[HTML]{F5F8FD}-5   & \textbf{20\%}                    & \multicolumn{1}{r|}{5\%}                              & \cellcolor[HTML]{FFF0D9}15   \\ \hline
		bowling            & \textbf{33\%}                    & \multicolumn{1}{r|}{10\%}                             & \cellcolor[HTML]{FFE8C5}23  & \textbf{33\%}                    & \multicolumn{1}{r|}{10\%}                             & \cellcolor[HTML]{FFE8C5}23  & \textcolor{red}{0\%}        & \multicolumn{1}{r|}{\textcolor{red}{0\%}}    & \cellcolor[HTML]{FFFFFF}0    & \textcolor{red}{0\%}             & \multicolumn{1}{r|}{\textcolor{red}{0\%}}             & \cellcolor[HTML]{FFFFFF}0    \\ \hline
		camel\_case        & \textcolor{teal}{100\%}          & \multicolumn{1}{r|}{\textcolor{teal}{100\%}}          & \cellcolor[HTML]{FFFFFF}0   & \textcolor{teal}{100\%}          & \multicolumn{1}{r|}{\textcolor{teal}{100\%}}          & \cellcolor[HTML]{FFFFFF}0   & \textcolor{teal}{100\%}     & \multicolumn{1}{r|}{\textcolor{teal}{100\%}} & \cellcolor[HTML]{FFFFFF}0    & \textcolor{teal}{100\%}          & \multicolumn{1}{r|}{\textcolor{teal}{100\%}}          & \cellcolor[HTML]{FFFFFF}0    \\ \hline
		coin\_sums         & \textcolor{teal}{100\%}          & \multicolumn{1}{r|}{\textcolor{teal}{100\%}}          & \cellcolor[HTML]{FFFFFF}0   & \textcolor{teal}{100\%}          & \multicolumn{1}{r|}{\textcolor{teal}{100\%}}          & \cellcolor[HTML]{FFFFFF}0   & \textcolor{teal}{100\%}     & \multicolumn{1}{r|}{\textcolor{teal}{100\%}} & \cellcolor[HTML]{FFFFFF}0    & \textcolor{teal}{100\%}          & \multicolumn{1}{r|}{\textcolor{teal}{100\%}}          & \cellcolor[HTML]{FFFFFF}0    \\ \hline
		cut\_vector        & \textcolor{teal}{\textbf{100\%}} & \multicolumn{1}{r|}{\textcolor{red}{0\%}}             & \cellcolor[HTML]{FF9900}100 & \textcolor{teal}{\textbf{100\%}} & \multicolumn{1}{r|}{\textcolor{red}{0\%}}             & \cellcolor[HTML]{FF9900}100 & \textcolor{teal}{\textbf{100\%}}     & \multicolumn{1}{r|}{\textcolor{red}{0\%}}    & \cellcolor[HTML]{FF9900}100  & \textcolor{teal}{\textbf{100\%}} & \multicolumn{1}{r|}{\textcolor{red}{0\%}}             & \cellcolor[HTML]{FF9900}100  \\ \hline
		dice\_game         & 20\%                             & \multicolumn{1}{r|}{\textbf{33\%}}                    & \cellcolor[HTML]{E7EFFC}-13 & 20\%                             & \multicolumn{1}{r|}{\textbf{33\%}}                    & \cellcolor[HTML]{E7EFFC}-13 & 1\%                         & \multicolumn{1}{r|}{\textbf{3\%}}                     & \cellcolor[HTML]{FBFCFE}-2   & 1\%                              & \multicolumn{1}{r|}{\textbf{3\%}}                     & \cellcolor[HTML]{FBFCFE}-2   \\ \hline
		find\_pair         & \textcolor{teal}{\textbf{100\%}} & \multicolumn{1}{r|}{84\%}                             & \cellcolor[HTML]{FFEFD7}16  & \textcolor{teal}{\textbf{100\%}} & \multicolumn{1}{r|}{84\%}                             & \cellcolor[HTML]{FFEFD7}16  & \textcolor{teal}{\textbf{100\%}}     & \multicolumn{1}{r|}{28\%}                    & \cellcolor[HTML]{FFB648}72   & \textcolor{teal}{\textbf{100\%}} & \multicolumn{1}{r|}{28\%}                             & \cellcolor[HTML]{FFB648}72   \\ \hline
		fizz\_buzz         & \textcolor{teal}{100\%}          & \multicolumn{1}{r|}{\textcolor{teal}{100\%}}          & \cellcolor[HTML]{FFFFFF}0   & \textcolor{teal}{100\%}          & \multicolumn{1}{r|}{\textcolor{teal}{100\%}}          & \cellcolor[HTML]{FFFFFF}0   & \textcolor{teal}{100\%}     & \multicolumn{1}{r|}{\textcolor{teal}{100\%}} & \cellcolor[HTML]{FFFFFF}0    & \textcolor{teal}{100\%}          & \multicolumn{1}{r|}{\textcolor{teal}{100\%}}          & \cellcolor[HTML]{FFFFFF}0    \\ \hline
		fuel\_cost         & \textcolor{teal}{100\%}          & \multicolumn{1}{r|}{\textcolor{teal}{100\%}}          & \cellcolor[HTML]{FFFFFF}0   & \textcolor{teal}{100\%}          & \multicolumn{1}{r|}{\textcolor{teal}{100\%}}          & \cellcolor[HTML]{FFFFFF}0   & \textcolor{teal}{100\%}     & \multicolumn{1}{r|}{\textcolor{teal}{100\%}} & \cellcolor[HTML]{FFFFFF}0    & \textcolor{teal}{100\%}          & \multicolumn{1}{r|}{\textcolor{teal}{100\%}}          & \cellcolor[HTML]{FFFFFF}0    \\ \hline
		gcd                & \textcolor{teal}{100\%}          & \multicolumn{1}{r|}{\textcolor{teal}{100\%}}          & \cellcolor[HTML]{FFFFFF}0   & \textcolor{teal}{100\%}          & \multicolumn{1}{r|}{\textcolor{teal}{100\%}}          & \cellcolor[HTML]{FFFFFF}0   & \textcolor{teal}{100\%}     & \multicolumn{1}{r|}{\textcolor{teal}{100\%}} & \cellcolor[HTML]{FFFFFF}0    & \textcolor{teal}{100\%}          & \multicolumn{1}{r|}{\textcolor{teal}{100\%}}          & \cellcolor[HTML]{FFFFFF}0    \\ \hline
		indices\_of\_subs. & 15\%                             & \multicolumn{1}{r|}{15\%}                             & \cellcolor[HTML]{FFFFFF}0   & 15\%                             & \multicolumn{1}{r|}{15\%}                             & \cellcolor[HTML]{FFFFFF}0   & 23\%                        & \multicolumn{1}{r|}{23\%}                    & \cellcolor[HTML]{FFFFFF}0    & 23\%                             & \multicolumn{1}{r|}{23\%}                             & \cellcolor[HTML]{FFFFFF}0    \\ \hline
		leaders            & \textcolor{teal}{\textbf{100\%}} & \multicolumn{1}{r|}{50\%}                             & \cellcolor[HTML]{FFCC80}50  & \textcolor{teal}{\textbf{100\%}} & \multicolumn{1}{r|}{50\%}                             & \cellcolor[HTML]{FFCC80}50  & \textcolor{teal}{\textbf{100\%}}     & \multicolumn{1}{r|}{18\%}                    & \cellcolor[HTML]{FFAC2E}82   & \textcolor{teal}{\textbf{100\%}} & \multicolumn{1}{r|}{18\%}                             & \cellcolor[HTML]{FFAC2E}82   \\ \hline
		luhn               & \textcolor{teal}{\textbf{100\%}} & \multicolumn{1}{r|}{\textcolor{red}{0\%}}             & \cellcolor[HTML]{FF9900}100 & \textcolor{teal}{\textbf{100\%}} & \multicolumn{1}{r|}{\textcolor{red}{0\%}}             & \cellcolor[HTML]{FF9900}100 & \textcolor{teal}{\textbf{100\%}}     & \multicolumn{1}{r|}{\textcolor{red}{0\%}}    & \cellcolor[HTML]{FF9900}100  & \textcolor{teal}{\textbf{100\%}} & \multicolumn{1}{r|}{\textcolor{red}{0\%}}             & \cellcolor[HTML]{FF9900}100  \\ \hline
		mastermind         & \textcolor{teal}{\textbf{100\%}} & \multicolumn{1}{r|}{\textcolor{red}{0\%}}             & \cellcolor[HTML]{FF9900}100 & \textcolor{teal}{\textbf{100\%}} & \multicolumn{1}{r|}{\textcolor{red}{0\%}}             & \cellcolor[HTML]{FF9900}100 & \textcolor{teal}{\textbf{100\%}}     & \multicolumn{1}{r|}{\textcolor{red}{0\%}}    & \cellcolor[HTML]{FF9900}100  & \textcolor{teal}{\textbf{100\%}} & \multicolumn{1}{r|}{\textcolor{red}{0\%}}             & \cellcolor[HTML]{FF9900}100  \\ \hline
		middle\_character  & \textcolor{teal}{100\%}          & \multicolumn{1}{r|}{\textcolor{teal}{100\%}}          & \cellcolor[HTML]{FFFFFF}0   & \textcolor{teal}{100\%}          & \multicolumn{1}{r|}{\textcolor{teal}{100\%}}          & \cellcolor[HTML]{FFFFFF}0   & \textcolor{teal}{100\%}     & \multicolumn{1}{r|}{\textcolor{teal}{100\%}} & \cellcolor[HTML]{FFFFFF}0    & \textcolor{teal}{100\%}          & \multicolumn{1}{r|}{\textcolor{teal}{100\%}}          & \cellcolor[HTML]{FFFFFF}0    \\ \hline
		paired\_digits     & \textcolor{teal}{\textbf{100\%}} & \multicolumn{1}{r|}{65\%}                             & \cellcolor[HTML]{FFDCA6}35  & \textcolor{teal}{\textbf{100\%}} & \multicolumn{1}{r|}{65\%}                             & \cellcolor[HTML]{FFDCA6}35  & \textcolor{teal}{\textbf{100\%}}     & \multicolumn{1}{r|}{81\%}                    & \cellcolor[HTML]{FFECCF}19   & \textcolor{teal}{\textbf{100\%}} & \multicolumn{1}{r|}{81\%}                             & \cellcolor[HTML]{FFECCF}19   \\ \hline
		shopping\_list     & \textcolor{teal}{100\%}          & \multicolumn{1}{r|}{\textcolor{teal}{100\%}}          & \cellcolor[HTML]{FFFFFF}0   & \textcolor{teal}{100\%}          & \multicolumn{1}{r|}{\textcolor{teal}{100\%}}          & \cellcolor[HTML]{FFFFFF}0   & \textcolor{teal}{100\%}     & \multicolumn{1}{r|}{\textcolor{teal}{100\%}} & \cellcolor[HTML]{FFFFFF}0    & \textcolor{teal}{100\%}          & \multicolumn{1}{r|}{\textcolor{teal}{100\%}}          & \cellcolor[HTML]{FFFFFF}0    \\ \hline
		snow\_day          & 50\%                             & \multicolumn{1}{r|}{50\%}                             & \cellcolor[HTML]{FFFFFF}0   & \textbf{92\%}                    & \multicolumn{1}{r|}{50\%}                             & \cellcolor[HTML]{FFD594}42  & \textcolor{red}{0\%}        & \multicolumn{1}{r|}{\textcolor{red}{0\%}}    & \cellcolor[HTML]{FFFFFF}0    & \textcolor{teal}{\textbf{100\%}} & \multicolumn{1}{r|}{5\%}                              & \cellcolor[HTML]{FF9F0D}95   \\ \hline
		solve\_boolean     & 20\%                             & \multicolumn{1}{r|}{\textcolor{teal}{\textbf{100\%}}} & \cellcolor[HTML]{6E9EEC}-80 & 20\%                             & \multicolumn{1}{r|}{\textcolor{teal}{\textbf{100\%}}} & \cellcolor[HTML]{6E9EEC}-80 & \textcolor{red}{0\%}        & \multicolumn{1}{r|}{\textcolor{teal}{\textbf{100\%}}} & \cellcolor[HTML]{4A86E8}-100 & \textcolor{red}{0\%}             & \multicolumn{1}{r|}{\textcolor{teal}{\textbf{100\%}}} & \cellcolor[HTML]{4A86E8}-100 \\ \hline
		spin\_words        & \textcolor{teal}{100\%}          & \multicolumn{1}{r|}{\textcolor{teal}{100\%}}          & \cellcolor[HTML]{FFFFFF}0   & \textcolor{teal}{100\%}          & \multicolumn{1}{r|}{\textcolor{teal}{100\%}}          & \cellcolor[HTML]{FFFFFF}0   & \textcolor{teal}{100\%}     & \multicolumn{1}{r|}{\textcolor{teal}{100\%}} & \cellcolor[HTML]{FFFFFF}0    & \textcolor{teal}{100\%}          & \multicolumn{1}{r|}{\textcolor{teal}{100\%}}          & \cellcolor[HTML]{FFFFFF}0    \\ \hline
		square\_digits     & 97\%                             & \multicolumn{1}{r|}{97\%}                             & \cellcolor[HTML]{FFFFFF}0   & 97\%                             & \multicolumn{1}{r|}{97\%}                             & \cellcolor[HTML]{FFFFFF}0   & \textcolor{teal}{100\%}     & \multicolumn{1}{r|}{\textcolor{teal}{100\%}} & \cellcolor[HTML]{FFFFFF}0    & \textcolor{teal}{100\%}          & \multicolumn{1}{r|}{\textcolor{teal}{100\%}}          & \cellcolor[HTML]{FFFFFF}0    \\ \hline
		substitution\_cip. & \textcolor{teal}{100\%}          & \multicolumn{1}{r|}{\textcolor{teal}{100\%}}          & \cellcolor[HTML]{FFFFFF}0   & \textcolor{teal}{100\%}          & \multicolumn{1}{r|}{\textcolor{teal}{100\%}}          & \cellcolor[HTML]{FFFFFF}0   & \textcolor{teal}{100\%}     & \multicolumn{1}{r|}{\textcolor{teal}{100\%}} & \cellcolor[HTML]{FFFFFF}0    & \textcolor{teal}{100\%}          & \multicolumn{1}{r|}{\textcolor{teal}{100\%}}          & \cellcolor[HTML]{FFFFFF}0    \\ \hline
		twitter            & \textcolor{teal}{100\%}          & \multicolumn{1}{r|}{\textcolor{teal}{100\%}}          & \cellcolor[HTML]{FFFFFF}0   & \textcolor{teal}{100\%}          & \multicolumn{1}{r|}{\textcolor{teal}{100\%}}          & \cellcolor[HTML]{FFFFFF}0   & \textcolor{teal}{100\%}     & \multicolumn{1}{r|}{\textcolor{teal}{100\%}} & \cellcolor[HTML]{FFFFFF}0    & \textcolor{teal}{100\%}          & \multicolumn{1}{r|}{\textcolor{teal}{100\%}}          & \cellcolor[HTML]{FFFFFF}0    \\ \hline
		vector\_distance   & 40\%                             & \multicolumn{1}{r|}{40\%}                             & \cellcolor[HTML]{FFFFFF}0   & \textcolor{teal}{100\%}          & \multicolumn{1}{r|}{\textcolor{teal}{100\%}}          & \cellcolor[HTML]{FFFFFF}0   & \textcolor{red}{0\%}        & \multicolumn{1}{r|}{\textcolor{red}{0\%}}    & \cellcolor[HTML]{FFFFFF}0    & 75\%                             & \multicolumn{1}{r|}{75\%}                             & \cellcolor[HTML]{FFFFFF}0    \\ \hline
	\end{tabular}
  }
	\vspace{2px}
	\caption{C++ - Edge case and random generated test pass ratio with exact and epsilon matching.}
	\label{t:tests-cpp}
	
\end{table*}

\subsubsection{Java}
We evaluated both models using Java language as well and the results are presented in Table \ref{t:tests-java}, similarly to the C++ evaluation.

During edge case testing, the ChatGPT model encountered runtime errors in 5 out of 10 tests for the task \textit{cut\_vector}, and in 1 out of 10 tests for the task \textit{leaders}.
Meanwhile, the Copilot model encountered runtime errors in 13 out of 21 tests for the task \textit{bowling}, and in 1 out of 10 tests for the task \textit{leaders}.

The runtime errors encountered by ChatGPT on the task \textit{cut\_vector} were due to improper handling of input where the code was not prepared for an array with only one element.
On the other hand, the runtime error on task \textit{leaders} was caused by the code's inability to handle an empty array input.

In the case of Copilot, the runtime errors encountered on the task \textit{bowling} were caused by improper handling of string indices.
The code was dereferencing characters based on an assumption that there would be more characters.
For the task \textit{leaders}, the code was not able to handle an empty array input.

Despite these errors, ChatGPT achieved a 100\% pass rate with epsilon testing on 15 tasks, while Copilot achieved a 100\% pass rate on 13 tasks during edge case testing.
ChatGPT performed better on 10 tasks with an average pass-rate of 55\% and a standard deviation of 31.77.
In comparison, Copilot performed better on 4 tasks with an average pass-rate of 58\% and a standard deviation of 40.4.

On random test cases, both ChatGPT and Copilot achieved a 100\% pass rate on 14 tasks.
During random testing ChatGPT performed worse on the following tasks than during edge case testing:
\textit{bouncing\_balls
	bowling
	dice\_game
	leaders
	snow\_day
	solve\_boolean
	vector\_distance}.
We conducted an investigation on the only task that attained a 100\% pass rate on the edge case test, namely \textit{vector\_distance}.
It was revealed that the discrepancies were due to floating point errors that exceeded the epsilon value employed in our matching method.
Compared to edge case testing, Copilot performed worse on the \textit{bowling, coin\_sums, leaders, mastermind, shopping\_list} and \textit{snow\_day} tasks.
Among Copilot generated solutions there was no such task that reached 100\% on the edge case testing and reached lower pass-rate on random testing.

Investigating the robustness of the generated code, using non-edge case tests compared to edge case tests, we found that ChatGPT's pass-rates increased on 2 tasks by an average of 22\% with a standard deviation of 19.8, while Copilot's pass-rates increased by average of 9.67\% with a standard deviation of 6.5.
We can conclude that the source code generated by both models that reached 100\% pass rate on the edge case mostly did not have lower pass-rates on random tests, and running non-edge test cases did not drastically increase the pass-rates.
Therefore, the working solutions are robust.
Regarding Java, ChatGPT performs slightly better, but Copilot's working examples seem to be more robust.


\begin{table*}[!h]
	\resizebox{0.99\textwidth}{!}{
	\begin{tabular}{|p{2.9cm}|rr|r|rr|r||rr|r|rr|r|}
		\hline
		& \multicolumn{3}{c|}{Edge - Exact}                                                                         & \multicolumn{3}{c||}{Edge - Epsilon}                                                                                     & \multicolumn{3}{c|}{Random - Exact}                                                                       & \multicolumn{3}{c|}{Random - Epsilon}                                                                                   \\ \hline
		Task               & \multicolumn{1}{l}{ChatGPT}          & \multicolumn{1}{l|}{Copilot}                          & \multicolumn{1}{l|}{Diff}    & \multicolumn{1}{l}{ChatGPT}      & \multicolumn{1}{l|}{Copilot}                          & \multicolumn{1}{l||}{Diff}    & \multicolumn{1}{l}{ChatGPT}         & \multicolumn{1}{l|}{Copilot}                 & \multicolumn{1}{l|}{Diff}    & \multicolumn{1}{l}{ChatGPT}      & \multicolumn{1}{l|}{Copilot}                          & \multicolumn{1}{l|}{Diff}    \\ \hline
		basement           & \textcolor{teal}{100\%}              & \multicolumn{1}{r|}{\textcolor{teal}{100\%}}          & \cellcolor[HTML]{FFFFFF}0    & \textcolor{teal}{100\%}          & \multicolumn{1}{r|}{\textcolor{teal}{100\%}}          & \cellcolor[HTML]{FFFFFF}0    & \textcolor{teal}{100\%}              & \multicolumn{1}{r|}{\textcolor{teal}{100\%}} & \cellcolor[HTML]{FFFFFF}0    & \textcolor{teal}{100\%}          & \multicolumn{1}{r|}{\textcolor{teal}{100\%}}          & \cellcolor[HTML]{FFFFFF}0    \\ \hline
		bouncing\_balls    & \textbf{80\%}                        & \multicolumn{1}{r|}{\textcolor{red}{0\%}}             & \cellcolor[HTML]{FFAE33}80   & \textbf{80\%}                    & \multicolumn{1}{r|}{10\%}                             & \cellcolor[HTML]{FFB84D}70   & \textbf{5\%}                         & \multicolumn{1}{r|}{\textcolor{red}{0\%}}    & \cellcolor[HTML]{FFFAF3}5    & 5\%                              & \multicolumn{1}{r|}{\textbf{20\%}}                    & \cellcolor[HTML]{E3ECFB}-15  \\ \hline
		bowling            & \textbf{38\%}                        & \multicolumn{1}{r|}{10\%}                             & \cellcolor[HTML]{FFE3B8}28   & \textbf{38\%}                    & \multicolumn{1}{r|}{10\%}                             & \cellcolor[HTML]{FFE3B8}28   & \textbf{5\%}                         & \multicolumn{1}{r|}{\textcolor{red}{0\%}}    & \cellcolor[HTML]{FFFAF3}5    & \textbf{5\%}                     & \multicolumn{1}{r|}{\textcolor{red}{0\%}}             & \cellcolor[HTML]{FFFAF3}5    \\ \hline
		camel\_case        & \textcolor{teal}{100\%}              & \multicolumn{1}{r|}{\textcolor{teal}{100\%}}          & \cellcolor[HTML]{FFFFFF}0    & \textcolor{teal}{100\%}          & \multicolumn{1}{r|}{\textcolor{teal}{100\%}}          & \cellcolor[HTML]{FFFFFF}0    & \textcolor{teal}{100\%}              & \multicolumn{1}{r|}{\textcolor{teal}{100\%}} & \cellcolor[HTML]{FFFFFF}0    & \textcolor{teal}{100\%}          & \multicolumn{1}{r|}{\textcolor{teal}{100\%}}          & \cellcolor[HTML]{FFFFFF}0    \\ \hline
		coin\_sums         & \textcolor{teal}{\textbf{100\%}}     & \multicolumn{1}{r|}{9\%}                              & \cellcolor[HTML]{FFA317}91   & \textcolor{teal}{\textbf{100\%}} & \multicolumn{1}{r|}{9\%}                              & \cellcolor[HTML]{FFA317}91   & \textcolor{teal}{\textbf{100\%}}     & \multicolumn{1}{r|}{\textcolor{red}{0\%}}    & \cellcolor[HTML]{FF9900}100  & \textcolor{teal}{\textbf{100\%}} & \multicolumn{1}{r|}{\textcolor{red}{0\%}}             & \cellcolor[HTML]{FF9900}100  \\ \hline
		cut\_vector        & \textbf{50\%}                        & \multicolumn{1}{r|}{\textcolor{red}{0\%}}             & \cellcolor[HTML]{FFCC80}50   & \textbf{50\%}                    & \multicolumn{1}{r|}{\textcolor{red}{0\%}}             & \cellcolor[HTML]{FFCC80}50   & \textbf{86\%}                        & \multicolumn{1}{r|}{\textcolor{red}{0\%}}    & \cellcolor[HTML]{FFA824}86   & \textbf{86\%}                    & \multicolumn{1}{r|}{\textcolor{red}{0\%}}             & \cellcolor[HTML]{FFA824}86   \\ \hline
		dice\_game         & 73\%                                 & \multicolumn{1}{r|}{\textcolor{teal}{\textbf{100\%}}} & \cellcolor[HTML]{CEDEF8}-27  & 73\%                             & \multicolumn{1}{r|}{\textcolor{teal}{\textbf{100\%}}} & \cellcolor[HTML]{CEDEF8}-27  & 35\%                                 & \multicolumn{1}{r|}{\textbf{65\%}}                    & \cellcolor[HTML]{C8DAF8}-30  & 54\%                             & \multicolumn{1}{r|}{\textcolor{teal}{\textbf{100\%}}} & \cellcolor[HTML]{ABC7F4}-46  \\ \hline
		find\_pair         & \textcolor{red}{0\%}                 & \multicolumn{1}{r|}{\textcolor{teal}{\textbf{100\%}}} & \cellcolor[HTML]{4A86E8}-100 & \textcolor{red}{0\%}             & \multicolumn{1}{r|}{\textcolor{teal}{\textbf{100\%}}} & \cellcolor[HTML]{4A86E8}-100 & \textcolor{red}{0\%}                 & \multicolumn{1}{r|}{\textcolor{teal}{\textbf{100\%}}} & \cellcolor[HTML]{4A86E8}-100 & \textcolor{red}{0\%}             & \multicolumn{1}{r|}{\textcolor{teal}{\textbf{100\%}}} & \cellcolor[HTML]{4A86E8}-100 \\ \hline
		fizz\_buzz         & \textcolor{teal}{100\%}              & \multicolumn{1}{r|}{\textcolor{teal}{100\%}}          & \cellcolor[HTML]{FFFFFF}0    & \textcolor{teal}{100\%}          & \multicolumn{1}{r|}{\textcolor{teal}{100\%}}          & \cellcolor[HTML]{FFFFFF}0    & \textcolor{teal}{100\%}              & \multicolumn{1}{r|}{\textcolor{teal}{100\%}} & \cellcolor[HTML]{FFFFFF}0    & \textcolor{teal}{100\%}          & \multicolumn{1}{r|}{\textcolor{teal}{100\%}}          & \cellcolor[HTML]{FFFFFF}0    \\ \hline
		fuel\_cost         & \textcolor{teal}{100\%}              & \multicolumn{1}{r|}{\textcolor{teal}{100\%}}          & \cellcolor[HTML]{FFFFFF}0    & \textcolor{teal}{100\%}          & \multicolumn{1}{r|}{\textcolor{teal}{100\%}}          & \cellcolor[HTML]{FFFFFF}0    & \textcolor{teal}{100\%}              & \multicolumn{1}{r|}{\textcolor{teal}{100\%}} & \cellcolor[HTML]{FFFFFF}0    & \textcolor{teal}{100\%}          & \multicolumn{1}{r|}{\textcolor{teal}{100\%}}          & \cellcolor[HTML]{FFFFFF}0    \\ \hline
		gcd                & \textcolor{teal}{100\%}              & \multicolumn{1}{r|}{\textcolor{teal}{100\%}}          & \cellcolor[HTML]{FFFFFF}0    & \textcolor{teal}{100\%}          & \multicolumn{1}{r|}{\textcolor{teal}{100\%}}          & \cellcolor[HTML]{FFFFFF}0    & \textcolor{teal}{100\%}              & \multicolumn{1}{r|}{\textcolor{teal}{100\%}} & \cellcolor[HTML]{FFFFFF}0    & \textcolor{teal}{100\%}          & \multicolumn{1}{r|}{\textcolor{teal}{100\%}}          & \cellcolor[HTML]{FFFFFF}0    \\ \hline
		indices\_of\_subs. & 15\%                                 & \multicolumn{1}{r|}{\textcolor{teal}{\textbf{100\%}}} & \cellcolor[HTML]{6598EB}-85  & 15\%                             & \multicolumn{1}{r|}{\textcolor{teal}{\textbf{100\%}}} & \cellcolor[HTML]{6598EB}-85  & 23\%                                 & \multicolumn{1}{r|}{\textcolor{teal}{\textbf{100\%}}} & \cellcolor[HTML]{73A1ED}-77  & 23\%                             & \multicolumn{1}{r|}{\textcolor{teal}{\textbf{100\%}}} & \cellcolor[HTML]{73A1ED}-77  \\ \hline
		leaders            & 50\%                                 & \multicolumn{1}{r|}{50\%}                             & \cellcolor[HTML]{FFFFFF}0    & 50\%                             & \multicolumn{1}{r|}{50\%}                             & \cellcolor[HTML]{FFFFFF}0    & 18\%                                 & \multicolumn{1}{r|}{18\%}                    & \cellcolor[HTML]{FFFFFF}0    & 18\%                             & \multicolumn{1}{r|}{18\%}                             & \cellcolor[HTML]{FFFFFF}0    \\ \hline
		luhn               & \textcolor{red}{0\%}                 & \multicolumn{1}{r|}{\textcolor{red}{0\%}}             & \cellcolor[HTML]{FFFFFF}0    & \textcolor{red}{0\%}             & \multicolumn{1}{r|}{\textcolor{red}{0\%}}             & \cellcolor[HTML]{FFFFFF}0    & \textcolor{red}{0\%}                 & \multicolumn{1}{r|}{\textcolor{red}{0\%}}    & \cellcolor[HTML]{FFFFFF}0    & \textcolor{red}{0\%}             & \multicolumn{1}{r|}{\textcolor{red}{0\%}}             & \cellcolor[HTML]{FFFFFF}0    \\ \hline
		mastermind         & \textcolor{teal}{\textbf{100\%}}     & \multicolumn{1}{r|}{50\%}                             & \cellcolor[HTML]{FFCC80}50   & \textcolor{teal}{\textbf{100\%}} & \multicolumn{1}{r|}{50\%}                             & \cellcolor[HTML]{FFCC80}50   & \textcolor{teal}{\textbf{100\%}}     & \multicolumn{1}{r|}{36\%}                    & \cellcolor[HTML]{FFBE5C}64   & \textcolor{teal}{\textbf{100\%}} & \multicolumn{1}{r|}{36\%}                             & \cellcolor[HTML]{FFBE5C}64   \\ \hline
		middle\_character  & \textcolor{teal}{100\%}              & \multicolumn{1}{r|}{\textcolor{teal}{100\%}}          & \cellcolor[HTML]{FFFFFF}0    & \textcolor{teal}{100\%}          & \multicolumn{1}{r|}{\textcolor{teal}{100\%}}          & \cellcolor[HTML]{FFFFFF}0    & \textcolor{teal}{100\%}              & \multicolumn{1}{r|}{\textcolor{teal}{100\%}} & \cellcolor[HTML]{FFFFFF}0    & \textcolor{teal}{100\%}          & \multicolumn{1}{r|}{\textcolor{teal}{100\%}}          & \cellcolor[HTML]{FFFFFF}0    \\ \hline
		paired\_digits     & \textcolor{teal}{\textbf{100\%}}     & \multicolumn{1}{r|}{65\%}                             & \cellcolor[HTML]{FFDCA6}35   & \textcolor{teal}{\textbf{100\%}} & \multicolumn{1}{r|}{65\%}                             & \cellcolor[HTML]{FFDCA6}35   & \textcolor{teal}{\textbf{100\%}}     & \multicolumn{1}{r|}{81\%}                    & \cellcolor[HTML]{FFECCF}19   & \textcolor{teal}{\textbf{100\%}} & \multicolumn{1}{r|}{81\%}                             & \cellcolor[HTML]{FFECCF}19   \\ \hline
		shopping\_list     & \textcolor{teal}{\textbf{100\%}}     & \multicolumn{1}{r|}{10\%}                             & \cellcolor[HTML]{FFA41A}90   & \textcolor{teal}{\textbf{100\%}} & \multicolumn{1}{r|}{10\%}                             & \cellcolor[HTML]{FFA41A}90   & \textcolor{teal}{\textbf{100\%}}     & \multicolumn{1}{r|}{\textcolor{red}{0\%}}    & \cellcolor[HTML]{FF9900}100  & \textcolor{teal}{\textbf{100\%}} & \multicolumn{1}{r|}{\textcolor{red}{0\%}}             & \cellcolor[HTML]{FF9900}100  \\ \hline
		snow\_day          & \textbf{50\%}                        & \multicolumn{1}{r|}{17\%}                             & \cellcolor[HTML]{FFDEAB}33   & \textbf{58\%}                    & \multicolumn{1}{r|}{25\%}                             & \cellcolor[HTML]{FFDEAB}33   & \textcolor{red}{0\%}                 & \multicolumn{1}{r|}{\textcolor{red}{0\%}}    & \cellcolor[HTML]{FFFFFF}0    & \textbf{10\%}                    & \multicolumn{1}{r|}{5\%}                              & \cellcolor[HTML]{FFFAF3}5    \\ \hline
		solve\_boolean     & 80\%                                 & \multicolumn{1}{r|}{\textcolor{teal}{\textbf{100\%}}} & \cellcolor[HTML]{DAE6FA}-20  & 80\%                             & \multicolumn{1}{r|}{\textcolor{teal}{\textbf{100\%}}} & \cellcolor[HTML]{DAE6FA}-20  & 75\%                                 & \multicolumn{1}{r|}{\textcolor{teal}{\textbf{100\%}}} & \cellcolor[HTML]{D1E0F9}-25  & 75\%                             & \multicolumn{1}{r|}{\textcolor{teal}{\textbf{100\%}}} & \cellcolor[HTML]{D1E0F9}-25  \\ \hline
		spin\_words        & \textcolor{teal}{100\%}              & \multicolumn{1}{r|}{\textcolor{teal}{100\%}}          & \cellcolor[HTML]{FFFFFF}0    & \textcolor{teal}{100\%}          & \multicolumn{1}{r|}{\textcolor{teal}{100\%}}          & \cellcolor[HTML]{FFFFFF}0    & \textcolor{teal}{100\%}              & \multicolumn{1}{r|}{\textcolor{teal}{100\%}} & \cellcolor[HTML]{FFFFFF}0    & \textcolor{teal}{100\%}          & \multicolumn{1}{r|}{\textcolor{teal}{100\%}}          & \cellcolor[HTML]{FFFFFF}0    \\ \hline
		square\_digits     & \textcolor{teal}{\textbf{100\%}}     & \multicolumn{1}{r|}{97\%}                             & \cellcolor[HTML]{FFFCF8}3    & \textcolor{teal}{\textbf{100\%}} & \multicolumn{1}{r|}{97\%}                             & \cellcolor[HTML]{FFFCF8}3    & \textcolor{teal}{100\%}              & \multicolumn{1}{r|}{\textcolor{teal}{100\%}} & \cellcolor[HTML]{FFFFFF}0    & \textcolor{teal}{100\%}          & \multicolumn{1}{r|}{\textcolor{teal}{100\%}}          & \cellcolor[HTML]{FFFFFF}0    \\ \hline
		substitution\_cip. & \textcolor{teal}{100\%}              & \multicolumn{1}{r|}{\textcolor{teal}{100\%}}          & \cellcolor[HTML]{FFFFFF}0    & \textcolor{teal}{100\%}          & \multicolumn{1}{r|}{\textcolor{teal}{100\%}}          & \cellcolor[HTML]{FFFFFF}0    & \textcolor{teal}{100\%}              & \multicolumn{1}{r|}{\textcolor{teal}{100\%}} & \cellcolor[HTML]{FFFFFF}0    & \textcolor{teal}{100\%}          & \multicolumn{1}{r|}{\textcolor{teal}{100\%}}          & \cellcolor[HTML]{FFFFFF}0    \\ \hline
		twitter            & \textcolor{teal}{100\%}              & \multicolumn{1}{r|}{\textcolor{teal}{100\%}}          & \cellcolor[HTML]{FFFFFF}0    & \textcolor{teal}{100\%}          & \multicolumn{1}{r|}{\textcolor{teal}{100\%}}          & \cellcolor[HTML]{FFFFFF}0    & \textcolor{teal}{100\%}              & \multicolumn{1}{r|}{\textcolor{teal}{100\%}} & \cellcolor[HTML]{FFFFFF}0    & \textcolor{teal}{100\%}          & \multicolumn{1}{r|}{\textcolor{teal}{100\%}}          & \cellcolor[HTML]{FFFFFF}0    \\ \hline
		vector\_distance   & \textbf{60\%}                        & \multicolumn{1}{r|}{\textcolor{red}{0\%}}             & \cellcolor[HTML]{FFC266}60   & \textcolor{teal}{\textbf{100\%}} & \multicolumn{1}{r|}{\textcolor{red}{0\%}}             & \cellcolor[HTML]{FF9900}100  & \textcolor{red}{0\%}                 & \multicolumn{1}{r|}{\textcolor{red}{0\%}}    & \cellcolor[HTML]{FFFFFF}0    & \textbf{98\%}                    & \multicolumn{1}{r|}{\textcolor{red}{0\%}}             & \cellcolor[HTML]{FF9C06}98   \\ \hline
	\end{tabular}
  }
	\vspace{2px}
	\caption{Java - Edge case and random generated test pass ratio with exact and epsilon matching.}
	\label{t:tests-java}
\end{table*}

\subsection{Checking technical quality}
\label{eval_static}


To perform a technical quality check, we employed static analyzers as a straightforward comparison method.
We utilized SonarQube\footnote{V.8.9, www.sonarsource.com/products/sonarqube/downloads/lts/8-9-lts/} and SonarScanner\footnote{V.4.6.1.2450, binaries.sonarsource.com/Distribution/sonar-scanner-cli/sonar-scanner-cli-4.6.1.2450-linux.zip} to analyze the projects, and extended SonarQube with the SourceMeter plug-in~\cite{sonar_plugin_sm} that offers additional metrics, coding rules, and language support.
As the community edition of SonarQube does not support C++ analysis, we relied on SourceMeter for C++ analysis.
This plug-in incorporates a variety of analyzers, such as ClangTidy, which are capable of detecting coding errors and bad programming practices of varying severity and calculating different metrics.


The coding smells and bugs were validated by hand as static analyzers are prone to false positive warnings and are tedious to configure.
The material referenced in Section \ref{l:results} includes a list of reported warnings and the fact whether it was a true or false positive warning.

Although C++ and Java are object-oriented, the solutions were very simple and did not use anything from OOP, therefore, there was no use of measuring OOP metrics like inheritance, coupling, or cohesion, thus we selected size and complexity metrics only.
We used several metrics, such as Logical Lines of Code (LLOC), Number of Statements (NOS), McCabe Cyclomatic Complexity (McCC), and Nesting Level (NLE), to evaluate the results of the static analysis.
(LLOC excludes lines containing comments or whitespace. NLE counts multiple \texttt{else if} statements as only one additional depth.)
These metrics are considered the lower the better.

On average, static analysis did not reveal any vulnerability hidden in the generated code either in ChatGPT- or in Copilot-generated solutions.
The tasks were quite small and did not require a large amount of library usage, this way, the generated code appears to be vulnerability-free but cannot be considered so due to the non-deterministic nature of the models.

\subsubsection{C++ results}

The summarized results of the static analysis are presented in Table \ref{t:static_analyzers}.
For C++, it is evident that the models are comparable in terms of metrics.
Therefore, we focused on a few cases where one model differs from the other more than the average.
Additionally, aside from pure metrics, it is crucial to consider the code smells that these models generate since they may be prioritized.

\begin{table*}[!h]
	\centering
	\resizebox{0.99\textwidth}{!}{
	\begin{tabular}{|l|rr|rr|rr|rr|rr|rr|rr|rr|rr|rr|}
		\hline
		& \multicolumn{10}{c|}{C++} & \multicolumn{10}{c|}{Java} \\ \hline
		\multirow{2}{*}{Task} & \multicolumn{2}{c|}{CodeSmells}  & \multicolumn{2}{c|}{LLOC} & \multicolumn{2}{c|}{NOS} & \multicolumn{2}{c|}{McCC} & \multicolumn{2}{c|}{NLE} & \multicolumn{2}{c|}{CodeSmells}  & \multicolumn{2}{c|}{LLOC} & \multicolumn{2}{c|}{NOS} & \multicolumn{2}{c|}{McCC} & \multicolumn{2}{c|}{NLE} \\ \cline{2-21}
		& G & P & G & P & G & P & G & P & G & P & G & P & G & P & G & P & G & P & G & P \\ \hline
		
		
		basement              &   0 &   0 &  20 &  16 &  20 &  15 &   4 &   3 &   2 &   2 & 0 & 0 & 17 & 17 & 11 & 10 & 3 & 3 & 2 & 2 \\ \hline
		bouncing\_balls       &   0 &   0 &  21 &  17 &  20 &  15 &   2 &   2 &   1 &   1 & 0 & 1 & 23 & 18 & 16 & 14 & 2 & 2 & 1 & 1 \\ \hline
		bowling               &   8 &   7 &  48 &  45 &  50 &  45 &  10 &   9 &   4 &   4 & 0 & 0 & 57 & 39 & 38 & 25 & 13& 7 & 3 & 3 \\ \hline
		camel\_case           &   2 &   3 &  20 &  22 &  20 &  21 &   4 &   4 &   2 &   3 & 0 & 3 & 16 & 26 & 11 & 18 & 2 & 5 & 1 & 3 \\ \hline
		coin\_sums            &   7 &   6 &  19 &  14 &  17 &  12 &   1 &   1 &   0 &   0 & 0 & 0 & 20 & 20 & 14 & 13 & 1 & 1 & 0 & 0 \\ \hline
		cut\_vector           &   2 &   3 &  27 &  50 &  29 &  55 &   5 &   7 &   2 &   2 & 1 & 0 & 28 & 34 & 21 & 26 & 3 & 5 & 2 & 2 \\ \hline
		dice\_game            &   2 &   0 &   9 &  16 &   8 &  17 &   2 &   4 &   1 &   3 & 0 & 0 & 20 & 20 & 15 & 14 & 4 & 4 & 3 & 3 \\ \hline
		find\_pair            &   0 &   2 &  17 &  34 &  19 &  32 &   4 &   4 &   3 &   2 & 0 & 0 & 20 & 29 & 13 & 25 & 3 & 6 & 2 & 3 \\ \hline
		fizz\_buzz            &   7 &   3 &  12 &  21 &  15 &  18 &   5 &   4 &   1 &   1 & 0 & 0 & 18 & 19 & 10 & 10 & 5 & 5 & 1 & 1 \\ \hline
		fuel\_cost            &   0 &   1 &  11 &  13 &  10 &  12 &   2 &   2 &   1 &   1 & 1 & 0 & 16 & 20 &  8 & 15 & 2 & 3 & 1 & 1 \\ \hline
		gcd                   &   3 &   0 &  12 &  17 &  10 &  14 &   2 &   2 &   1 &   1 & 0 & 1 & 16 & 19 &  9 & 12 & 2 & 5 & 1 & 2 \\ \hline
		indices\_of\_subtring &   1 &   0 &  19 &  18 &  18 &  14 &   3 &   6 &   1 &   2 & 1 & 0 & 18 & 23 & 10 & 17 & 2 & 4 & 1 & 2 \\ \hline
		leaders               &   1 &   2 &  21 &  22 &  22 &  23 &   4 &   4 &   2 &   2 & 1 & 2 & 20 & 34 & 13 & 30 & 3 & 6 & 2 & 2 \\ \hline
		luhn                  &   4 &   3 &  24 &  19 &  25 &  18 &   5 &   4 &   2 &   3 & 0 & 0 & 25 & 24 & 16 & 17 & 5 & 4 & 3 & 2 \\ \hline
		mastermind            &   0 &   1 &  29 &  23 &  29 &  26 &   5 &   6 &   2 &   3 & 1 & 1 & 31 & 33 & 21 & 28 & 5 & 7 & 2 & 3 \\ \hline
		middle\_character     &   2 &   3 &  14 &  14 &  12 &  11 &   2 &   2 &   1 &   1 & 0 & 0 & 15 & 17 &  7 & 8  & 2 & 2 & 1 & 1 \\ \hline
		paired\_digits        &   0 &   0 &  16 &  17 &  16 &  16 &   3 &   3 &   2 &   2 & 0 & 0 & 16 & 21 & 10 & 13 & 3 & 4 & 2 & 2 \\ \hline
		shopping\_list        &   2 &   1 &  19 &  14 &  19 &  13 &   3 &   2 &   1 &   1 & 0 & 2 & 16 & 25 & 11 & 21 & 2 & 4 & 1 & 1 \\ \hline
		snow\_day             &   1 &   0 &  21 &  17 &  21 &  17 &   2 &   2 &   1 &   1 & 0 & 1 & 15 & 19 &  9 & 13 & 1 & 2 & 0 & 1 \\ \hline
		solve\_boolean        &   1 &   2 &  30 &  30 &  33 &  32 &   9 &  10 &   2 &   2 & 2 & 1 & 30 & 27 & 22 & 17 & 11& 7 & 3 & 3 \\ \hline
		spin\_words           &   3 &   4 &  27 &  36 &  26 &  43 &   5 &   9 &   3 &   4 & 0 & 3 & 21 & 24 & 12 & 15 & 3 & 4 & 2 & 2 \\ \hline
		square\_digits        &   3 &   4 &  16 &  17 &  14 &  14 &   2 &   2 &   1 &   1 & 0 & 1 & 18 & 18 & 13 & 10 & 2 & 2 & 1 & 1 \\ \hline
		substitution\_cipher  &   1 &   2 &  24 &  26 &  25 &  27 &   3 &   4 &   1 &   3 & 0 & 1 & 29 & 19 & 22 & 13 & 4 & 2 & 2 & 1 \\ \hline
		twitter               &   4 &   4 &  17 &  12 &  16 &   9 &   3 &   3 &   1 &   1 & 0 & 1 & 16 & 18 &  8 & 9  & 3 & 3 & 1 & 1 \\ \hline
		vector\_distance      &   2 &   1 &  30 &  20 &  30 &  19 &   6 &   2 &   1 &   1 & 0 & 0 & 15 & 24 & 10 & 20 & 2 & 4 & 1 & 1 \\ \hline
		
	\end{tabular}
}
	\vspace{2px}
	\caption{Static analyzer result for C++ and Java (G = ChatGPT, P = Copilot).}
	\label{t:static_analyzers}
	\vspace{-25px}
\end{table*}

The average ratio of the metrics is around 1.0 so we discuss the tasks where one model scored 1.5 times more of a metric than the other.
These tasks are \textit{cut\_vector}, \textit{dice\_game}, \textit{find\_pair}, \textit{fizz\_buzz}, \textit{indices\_of\-substring}, \textit{spin\_words}, \textit{substitution\_cipher}, \textit{twitter} and \textit{cut\_vector}.

ChatGPT-generated sources are worse in \textit{twitter} and \textit{vector\_distance}.
Investigating these source codes we found that the higher McCC value in task \textit{vector\_distance} is due to the way it reads the input and the input is validated.
The higher NOS value in task \textit{twitter} is due to the fact that ChatGPT generated a \texttt{main} method in order to test the solution while Copilot did not.

Copilot scored worse metric values in the remaining tasks as there was no such task where one metric was better for one model and another metric was better for another model.
In task \textit{substitution\_cipher} the metric NLE is 3 times higher for Copilot, which is due to embedded for-loops.
ChatGPT used an unordered\_map while Copilot used for-loops resulting in higher NLE and higher complexity.
For task \textit{spin\_words} two metrics, NOS and McCC were worse in Copilot-generated source.
In this case, both McCC and NLE higher values were caused by the fact that Copilot used for-loops to reverse a string while ChatGPT used \texttt{std::reverse} calls.
For task \textit{indices\_of\_substring}, the McCC and NLE metrics were twice as bad.
Once again, Copilot generated code that compares substring meanwhile ChatGPT utilized the find method.
For task \textit{fizz\_buzz} the LLOC metric was worse, which was caused by the additional curly-brace pairs in Copilot's code.
For task \textit{find\_pair} the metrics LLOC and NOS were worse.
Investigating this source we found that Copilot used a vector for testing like ChatGPT, but it used push\_back methods multiple times instead of initializing it with the generic initializer.
For task \textit{cut\_vector} it was the same case.
Task \textit{dice\_game} scored worse in every metric and NLE was 3 times worse.
Investigating this task we found that ChatGPT used a formula to calculate the result while Copilot enumerated every possible outcome and this increased all the metrics.
Note, that in this case neither of the models generated correctly functioning code but Copilot's solution is closer to a working code.


Besides the pure metrics, the number of bad coding practices and the types of these must be investigated.
The number of code smells was very similar for both models.
The most common code smell was using magic constants.
In the solutions generated by ChatGPT, it occurred 25 times, while in the solutions of Copilot, it occurred 24 times.

From the validated coding smells the following smells were equally typical for both models: magic constant, narrowing conversion, string reallocation, and copying values instead of passing references.
Coding smells typical for ChatGPT: multiple declarations in one line, not using braces for single statements.
Coding smells typical for Copilot: redundant string initialization, not using range-based for-loops, using else after return, not using empty for emptiness check, redundant boolean operator.

The mentioned coding smells are categorized as Major in SonarQube.
We also took into consideration the Minor labeled smells.
There were three of them in ChatGPT-generated sources.
All of them suggest replacing a code part with a standard algorithm, like accumulate and transform.
For Copilot, the Minor labeled smells point out that there were return statements following each other and a variable's scope could be reduced.

Taking C++ both models are similar in the aspects of metrics and coding smells too.
Copilot tends to generate more coding smells specifically that relate to modern C++ code.

\subsubsection{Java results}

Similarly to C++, we only discuss tasks where the models are significantly different.
One of the tasks where ChatGPT scored worse is \textit{substitution\_cipher}.
The NOS, McCC, and NLE metrics are worse for ChatGPT.
Investigating the source code, ChatGPT generated a Map for characters meanwhile Copilot used single methods for look-up.
This affects the above-mentioned metrics as there are more statements and additional loops.
Another task where we investigate the code is \textit{bowling}.
In this case, ChatGPT generated a utility function for the task which led to the increased McCC as they usually include loops.

Copilot-generated tasks got worse metric values for many tasks and not only for one metric.
Investigating the source code we found that for \textit{vector\_distance}, \textit{shopping\_list}, \textit{fuel\_cost}, and \textit{cut\_vector} the difference is due to testing.
ChatGPT generates fixed tests while Copilot reads values from console resulting additional code e.g. for reading vectors.

In the remaining examples LLOC, McCC and NOS were higher for Copilot.
Investigating the code we found that all the tasks are similar, Copilot tends to generate more iterations than ChatGPT.
It uses iterations instead of using built-in iterations and methods.

Paying attention to the validated code smells again, there was no code smell which was typical for both models.
ChatGPT tends to have the following code smells: using synchronized class if not needed, not using interface or abstract return types, and ChatGPT-generated one example where an unused import was present.
Copilot tends to have the following code smells: misleading method name that matches the class name, not using string builder thus making the code less effective, bad naming convention for variables, having unused parameters and in one example Copilot introduced a bug by using a method overload that generates a temporary object.

For Java, we conclude that Copilot generates larger and more complex code which could be replaced by improved algorithmic solutions and function calls.

Based on these results we can answer \textbf{RQ 1: How does LLM-generated source code score in terms of source code quality?}
LLMs perform quite well in terms of static analysis.
They only occasionally introduce bad smells or bugs but the generated code passes the SonarQube quality gates mostly with the best score.
\vspace{-13px}
\subsection{Human evaluation}
\label{eval_reviews}

%In the human evaluation we asked 5-5 reviewers per programming language who had 5 to 20 years of experience for both languages. %years of industrial experience.
In the human evaluation, 5-5 developers were interviewed for each of the two programming languages. These developers had industry experience ranging from 5 to 20 years.
We developed a simple web application for performing the manual evaluation.
During the inspection, the developers were required to read the task text used to generate the source code.
Once they understood the task, the generated sources were presented side-by-side for both models, with the order randomly swapped to prevent the developer from identifying which model was used.
The developers were then asked to rate four properties of each source code on a [-2;2] interval, excluding 0.
We chose this interval because it does not allow the developer to remain neutral, and the plus and minus values represent liking and disliking, respectively.
The properties that had to be scored were the following (with their instructions towards the developers):
\begin{itemize}
	\item First impression: Score the source code according to your first impression. There are no special aspects given that affect this score what you should consider. It is solely your impression. Scale it from -2 to 2.
	\item Readability: Score the source code according to its readability. Readability refers to your experience during reading the code, while you are trying to understand the various steps and conditions. How clear the purpose of a variable or a method is. How easily you can follow the flow of the code. Score it from -2 to 2.
	\item Usability: Score the source code according to its usability. Usability refers to your experience during interacting with the source code. It involves writing tests, including it in your source base, or using it as a black box program. It should measure how satisfied you are after using this source code. Score it from -2 to 2.
	\item Modifiability: Score the source code according to it's modifiability. Modifiability determines how hard it is to change the code, to add or remove functionality. Score it from -2 to 2.
\end{itemize}

Besides these properties, the developers had to decide whether they accepted the source code or not.
The possible values were ``Strong reject'', ``Weak reject'', ``Weak accept'', and ``Strong accept''.
These string values are mapped to the already used [-2;+2] interval where ``Strong reject'' refers to -2 and ``Strong accept'' refers to +2.
Zero value is not allowed either.
After scoring all the values, the developers had to decide which of the shown source code was the better one.


\begin{table*}[!h]
	%\begin{adjustbox}{width=\linewidth,center} %\columnwidth
	\resizebox{0.99\textwidth}{!}{	
		\begin{tabular}{|p{2.9cm}|rr|rr|rr|rr|rr|r||rr|rr|rr|rr|rr|r|}
			\hline
			& \multicolumn{11}{c||}{C++} & \multicolumn{11}{c|}{Java} \\ \hline
			\multirow{2}{*}{Task} & \multicolumn{2}{c|}{F.Impr.}  & \multicolumn{2}{c|}{Usab.} & \multicolumn{2}{c|}{Read.} & \multicolumn{2}{c|}{Modif.} & \multicolumn{2}{c|}{Acc.} & \multirow{2}{*}{G/P} &\multicolumn{2}{c|}{F.Impr.}  & \multicolumn{2}{c|}{Usab.} & \multicolumn{2}{c|}{Read.} & \multicolumn{2}{c|}{Modif.} & \multicolumn{2}{c|}{Acc.} & \multirow{2}{*}{G/P} \\ \cline{2-11} \cline{13-22}
			& G & P & G & P & G & P & G & P & G & P & & G & P & G & P & G & P & G & P & G & P &\\ \hline
			basement               & \cellcolor[HTML]{A6CB96}6            & {\cellcolor[HTML]{A6CB96}6}            & \cellcolor[HTML]{79B161}9            & {\cellcolor[HTML]{A6CB96}6}            & \cellcolor[HTML]{97C384}7            & {\cellcolor[HTML]{88BA73}8}            & \cellcolor[HTML]{88BA73}8            & {\cellcolor[HTML]{88BA73}8}            & \cellcolor[HTML]{79B161}9            & {\cellcolor[HTML]{79B161}9}            & \cellcolor[HTML]{FFCC7F}-5            & \cellcolor[HTML]{6AA84F}10           & {\cellcolor[HTML]{6AA84F}10}           & \cellcolor[HTML]{6AA84F}10           & {\cellcolor[HTML]{6AA84F}10}           & \cellcolor[HTML]{6AA84F}10           & {\cellcolor[HTML]{6AA84F}10}           & \cellcolor[HTML]{6AA84F}10           & {\cellcolor[HTML]{6AA84F}10}           & \cellcolor[HTML]{6AA84F}10           & {\cellcolor[HTML]{6AA84F}10}           & \cellcolor[HTML]{AB906D}-1            \\ \hline
			bouncing\_balls        & \cellcolor[HTML]{D3E5CB}3            & {\cellcolor[HTML]{E2EEDC}2}            & \cellcolor[HTML]{FCF1F1}-1           & {\cellcolor[HTML]{EDA39D}-7}           & \cellcolor[HTML]{D3E5CB}3            & {\cellcolor[HTML]{B5D4A7}5}            & \cellcolor[HTML]{97C384}7            & {\cellcolor[HTML]{97C384}7}            & \cellcolor[HTML]{FCF1F1}-1           & {\cellcolor[HTML]{E88981}-9}           & \cellcolor[HTML]{FFD699}-4            & \cellcolor[HTML]{79B161}9            & {\cellcolor[HTML]{E2EEDC}2}            & \cellcolor[HTML]{FFFFFF}0            & {\cellcolor[HTML]{D3E5CB}3}            & \cellcolor[HTML]{88BA73}8            & {\cellcolor[HTML]{C4DDB9}4}            & \cellcolor[HTML]{79B161}9            & {\cellcolor[HTML]{97C384}7}            & \cellcolor[HTML]{E2EEDC}2            & {\cellcolor[HTML]{D3E5CB}3}            & \cellcolor[HTML]{BE9255}-3            \\ \hline
			bowling                & \cellcolor[HTML]{F7D7D5}-3           & {\cellcolor[HTML]{FAE4E3}-2}           & \cellcolor[HTML]{F1F7EE}1            & {\cellcolor[HTML]{E2EEDC}2}            & \cellcolor[HTML]{E2EEDC}2            & {\cellcolor[HTML]{FAE4E3}-2}           & \cellcolor[HTML]{F1F7EE}1            & {\cellcolor[HTML]{FCF1F1}-1}           & \cellcolor[HTML]{FAE4E3}-2           & {\cellcolor[HTML]{FAE4E3}-2}           & \cellcolor[HTML]{FFFFFF}0             & \cellcolor[HTML]{E2EEDC}2            & {\cellcolor[HTML]{A6CB96}6}            & \cellcolor[HTML]{97C384}7            & {\cellcolor[HTML]{B5D4A7}5}            & \cellcolor[HTML]{A6CB96}6            & {\cellcolor[HTML]{A6CB96}6}            & \cellcolor[HTML]{C4DDB9}4            & {\cellcolor[HTML]{D3E5CB}3}            & \cellcolor[HTML]{B5D4A7}5            & {\cellcolor[HTML]{C4DDB9}4}            & \cellcolor[HTML]{A18F7A}0             \\ \hline
			camel\_case            & \cellcolor[HTML]{79B161}9            & {\cellcolor[HTML]{B5D4A7}5}            & \cellcolor[HTML]{88BA73}8            & {\cellcolor[HTML]{C4DDB9}4}            & \cellcolor[HTML]{88BA73}8            & {\cellcolor[HTML]{D3E5CB}3}            & \cellcolor[HTML]{88BA73}8            & {\cellcolor[HTML]{88BA73}8}            & \cellcolor[HTML]{6AA84F}10           & {\cellcolor[HTML]{88BA73}8}            & \cellcolor[HTML]{FFC166}-6            & \cellcolor[HTML]{A6CB96}6            & {\cellcolor[HTML]{C4DDB9}4}            & \cellcolor[HTML]{FCF1F1}-1           & {\cellcolor[HTML]{B5D4A7}5}            & \cellcolor[HTML]{88BA73}8            & {\cellcolor[HTML]{FAE4E3}-2}           & \cellcolor[HTML]{88BA73}8            & {\cellcolor[HTML]{E2EEDC}2}            & \cellcolor[HTML]{FCF1F1}-1           & {\cellcolor[HTML]{C4DDB9}4}            & \cellcolor[HTML]{AB906D}-1            \\ \hline
			coin\_sums             & \cellcolor[HTML]{88BA73}8            & {\cellcolor[HTML]{D3E5CB}3}            & \cellcolor[HTML]{88BA73}8            & {\cellcolor[HTML]{D3E5CB}3}            & \cellcolor[HTML]{88BA73}8            & {\cellcolor[HTML]{A6CB96}6}            & \cellcolor[HTML]{88BA73}8            & {\cellcolor[HTML]{E2EEDC}2}            & \cellcolor[HTML]{97C384}7            & {\cellcolor[HTML]{C4DDB9}4}            & \cellcolor[HTML]{FFC166}-6            & \cellcolor[HTML]{79B161}9            & {\cellcolor[HTML]{97C384}7}            & \cellcolor[HTML]{B5D4A7}5            & {\cellcolor[HTML]{97C384}7}            & \cellcolor[HTML]{79B161}9            & {\cellcolor[HTML]{A6CB96}6}            & \cellcolor[HTML]{79B161}9            & {\cellcolor[HTML]{79B161}9}            & \cellcolor[HTML]{6AA84F}10           & {\cellcolor[HTML]{79B161}9}            & \cellcolor[HTML]{D0943D}-5            \\ \hline
			cut\_vector            & \cellcolor[HTML]{E2EEDC}2            & {\cellcolor[HTML]{EDA39D}-7}           & \cellcolor[HTML]{F7D7D5}-3           & {\cellcolor[HTML]{F2BDB9}-5}           & \cellcolor[HTML]{88BA73}8            & {\cellcolor[HTML]{C4DDB9}4}            & \cellcolor[HTML]{D3E5CB}3            & {\cellcolor[HTML]{E2EEDC}2}            & \cellcolor[HTML]{FAE4E3}-2           & {\cellcolor[HTML]{F0B0AB}-6}           & \cellcolor[HTML]{FFF4E5}-1            & \cellcolor[HTML]{79B161}9            & {\cellcolor[HTML]{F7D7D5}-3}           & \cellcolor[HTML]{88BA73}8            & {\cellcolor[HTML]{FFFFFF}0}            & \cellcolor[HTML]{79B161}9            & {\cellcolor[HTML]{C4DDB9}4}            & \cellcolor[HTML]{88BA73}8            & {\cellcolor[HTML]{C4DDB9}4}            & \cellcolor[HTML]{79B161}9            & {\cellcolor[HTML]{F1F7EE}1}            & \cellcolor[HTML]{DA9530}-6            \\ \hline
			dice\_game             & \cellcolor[HTML]{F7D7D5}-3           & {\cellcolor[HTML]{FAE4E3}-2}           & \cellcolor[HTML]{E88981}-9           & {\cellcolor[HTML]{C4DDB9}4}            & \cellcolor[HTML]{88BA73}8            & {\cellcolor[HTML]{97C384}7}            & \cellcolor[HTML]{C4DDB9}4            & {\cellcolor[HTML]{A6CB96}6}            & \cellcolor[HTML]{E67C73}-10          & {\cellcolor[HTML]{B5D4A7}5}            & \cellcolor[HTML]{5592F6}9             & \cellcolor[HTML]{88BA73}8            & {\cellcolor[HTML]{B5D4A7}5}            & \cellcolor[HTML]{A6CB96}6            & {\cellcolor[HTML]{79B161}9}            & \cellcolor[HTML]{79B161}9            & {\cellcolor[HTML]{97C384}7}            & \cellcolor[HTML]{79B161}9            & {\cellcolor[HTML]{88BA73}8}            & \cellcolor[HTML]{97C384}7            & {\cellcolor[HTML]{79B161}9}            & \cellcolor[HTML]{BE9255}-3            \\ \hline
			find\_pair             & \cellcolor[HTML]{88BA73}8            & {\cellcolor[HTML]{F5CAC7}-4}           & \cellcolor[HTML]{A6CB96}6            & {\cellcolor[HTML]{EDA39D}-7}           & \cellcolor[HTML]{88BA73}8            & {\cellcolor[HTML]{FFFFFF}0}            & \cellcolor[HTML]{88BA73}8            & {\cellcolor[HTML]{FAE4E3}-2}           & \cellcolor[HTML]{88BA73}8            & {\cellcolor[HTML]{E88981}-9}           & \cellcolor[HTML]{FF9900}-10           & \cellcolor[HTML]{79B161}9            & {\cellcolor[HTML]{FFFFFF}0}            & \cellcolor[HTML]{97C384}7            & {\cellcolor[HTML]{C4DDB9}4}            & \cellcolor[HTML]{88BA73}8            & {\cellcolor[HTML]{C4DDB9}4}            & \cellcolor[HTML]{88BA73}8            & {\cellcolor[HTML]{B5D4A7}5}            & \cellcolor[HTML]{88BA73}8            & {\cellcolor[HTML]{E2EEDC}2}            & \cellcolor[HTML]{D0943D}-5            \\ \hline
			fizz\_buzz             & \cellcolor[HTML]{88BA73}8            & {\cellcolor[HTML]{D3E5CB}3}            & \cellcolor[HTML]{88BA73}8            & {\cellcolor[HTML]{C4DDB9}4}            & \cellcolor[HTML]{88BA73}8            & {\cellcolor[HTML]{B5D4A7}5}            & \cellcolor[HTML]{B5D4A7}5            & {\cellcolor[HTML]{97C384}7}            & \cellcolor[HTML]{6AA84F}10           & {\cellcolor[HTML]{B5D4A7}5}            & \cellcolor[HTML]{FFE0B2}-3            & \cellcolor[HTML]{97C384}7            & {\cellcolor[HTML]{A6CB96}6}            & \cellcolor[HTML]{6AA84F}10           & {\cellcolor[HTML]{6AA84F}10}           & \cellcolor[HTML]{79B161}9            & {\cellcolor[HTML]{88BA73}8}            & \cellcolor[HTML]{97C384}7            & {\cellcolor[HTML]{97C384}7}            & \cellcolor[HTML]{79B161}9            & {\cellcolor[HTML]{79B161}9}            & \cellcolor[HTML]{8F8E92}2             \\ \hline
			fuel\_cost             & \cellcolor[HTML]{97C384}7            & {\cellcolor[HTML]{C4DDB9}4}            & \cellcolor[HTML]{88BA73}8            & {\cellcolor[HTML]{C4DDB9}4}            & \cellcolor[HTML]{88BA73}8            & {\cellcolor[HTML]{88BA73}8}            & \cellcolor[HTML]{88BA73}8            & {\cellcolor[HTML]{A6CB96}6}            & \cellcolor[HTML]{88BA73}8            & {\cellcolor[HTML]{A6CB96}6}            & \cellcolor[HTML]{FFD699}-4            & \cellcolor[HTML]{88BA73}8            & {\cellcolor[HTML]{A6CB96}6}            & \cellcolor[HTML]{79B161}9            & {\cellcolor[HTML]{6AA84F}10}           & \cellcolor[HTML]{6AA84F}10           & {\cellcolor[HTML]{B5D4A7}5}            & \cellcolor[HTML]{79B161}9            & {\cellcolor[HTML]{79B161}9}            & \cellcolor[HTML]{6AA84F}10           & {\cellcolor[HTML]{79B161}9}            & \cellcolor[HTML]{BE9255}-3            \\ \hline
			gcd                    & \cellcolor[HTML]{B5D4A7}5            & {\cellcolor[HTML]{79B161}9}            & \cellcolor[HTML]{88BA73}8            & {\cellcolor[HTML]{B5D4A7}5}            & \cellcolor[HTML]{88BA73}8            & {\cellcolor[HTML]{97C384}7}            & \cellcolor[HTML]{B5D4A7}5            & {\cellcolor[HTML]{88BA73}8}            & \cellcolor[HTML]{79B161}9            & {\cellcolor[HTML]{97C384}7}            & \cellcolor[HTML]{FFFFFF}0             & \cellcolor[HTML]{6AA84F}10           & {\cellcolor[HTML]{FCF1F1}-1}           & \cellcolor[HTML]{6AA84F}10           & {\cellcolor[HTML]{97C384}7}            & \cellcolor[HTML]{79B161}9            & {\cellcolor[HTML]{C4DDB9}4}            & \cellcolor[HTML]{79B161}9            & {\cellcolor[HTML]{97C384}7}            & \cellcolor[HTML]{6AA84F}10           & {\cellcolor[HTML]{D3E5CB}3}            & \cellcolor[HTML]{F6980C}-9            \\ \hline
			indices\_of\_sub.      & \cellcolor[HTML]{D3E5CB}3            & {\cellcolor[HTML]{FAE4E3}-2}           & \cellcolor[HTML]{F7D7D5}-3           & {\cellcolor[HTML]{B5D4A7}5}            & \cellcolor[HTML]{6AA84F}10           & {\cellcolor[HTML]{F1F7EE}1}            & \cellcolor[HTML]{B5D4A7}5            & {\cellcolor[HTML]{C4DDB9}4}            & \cellcolor[HTML]{F7D7D5}-3           & {\cellcolor[HTML]{97C384}7}            & \cellcolor[HTML]{B4CFFB}4             & \cellcolor[HTML]{79B161}9            & {\cellcolor[HTML]{E2EEDC}2}            & \cellcolor[HTML]{97C384}7            & {\cellcolor[HTML]{97C384}7}            & \cellcolor[HTML]{79B161}9            & {\cellcolor[HTML]{D3E5CB}3}            & \cellcolor[HTML]{79B161}9            & {\cellcolor[HTML]{B5D4A7}5}            & \cellcolor[HTML]{79B161}9            & {\cellcolor[HTML]{A6CB96}6}            & \cellcolor[HTML]{B49161}-2            \\ \hline
			leaders                & \cellcolor[HTML]{79B161}9            & {\cellcolor[HTML]{A6CB96}6}            & \cellcolor[HTML]{79B161}9            & {\cellcolor[HTML]{C4DDB9}4}            & \cellcolor[HTML]{79B161}9            & {\cellcolor[HTML]{97C384}7}            & \cellcolor[HTML]{88BA73}8            & {\cellcolor[HTML]{88BA73}8}            & \cellcolor[HTML]{6AA84F}10           & {\cellcolor[HTML]{88BA73}8}            & \cellcolor[HTML]{FFCC7F}-5            & \cellcolor[HTML]{79B161}9            & {\cellcolor[HTML]{F0B0AB}-6}           & \cellcolor[HTML]{88BA73}8            & {\cellcolor[HTML]{A6CB96}6}            & \cellcolor[HTML]{6AA84F}10           & {\cellcolor[HTML]{FCF1F1}-1}           & \cellcolor[HTML]{88BA73}8            & {\cellcolor[HTML]{FFFFFF}0}            & \cellcolor[HTML]{79B161}9            & {\cellcolor[HTML]{C4DDB9}4}            & \cellcolor[HTML]{F6980C}-9            \\ \hline
			luhn                   & \cellcolor[HTML]{C4DDB9}4            & {\cellcolor[HTML]{97C384}7}            & \cellcolor[HTML]{E2EEDC}2            & {\cellcolor[HTML]{E2EEDC}2}            & \cellcolor[HTML]{97C384}7            & {\cellcolor[HTML]{C4DDB9}4}            & \cellcolor[HTML]{88BA73}8            & {\cellcolor[HTML]{D3E5CB}3}            & \cellcolor[HTML]{E2EEDC}2            & {\cellcolor[HTML]{E2EEDC}2}            & \cellcolor[HTML]{FFFFFF}0             & \cellcolor[HTML]{E2EEDC}2            & {\cellcolor[HTML]{F2BDB9}-5}           & \cellcolor[HTML]{F1F7EE}1            & {\cellcolor[HTML]{F5CAC7}-4}           & \cellcolor[HTML]{88BA73}8            & {\cellcolor[HTML]{C4DDB9}4}            & \cellcolor[HTML]{A6CB96}6            & {\cellcolor[HTML]{E2EEDC}2}            & \cellcolor[HTML]{FCF1F1}-1           & {\cellcolor[HTML]{EDA39D}-7}           & \cellcolor[HTML]{D0943D}-5            \\ \hline
			mastermind             & \cellcolor[HTML]{A6CB96}6            & {\cellcolor[HTML]{D3E5CB}3}            & \cellcolor[HTML]{88BA73}8            & {\cellcolor[HTML]{FAE4E3}-2}           & \cellcolor[HTML]{D3E5CB}3            & {\cellcolor[HTML]{88BA73}8}            & \cellcolor[HTML]{E2EEDC}2            & {\cellcolor[HTML]{97C384}7}            & \cellcolor[HTML]{88BA73}8            & {\cellcolor[HTML]{FCF1F1}-1}           & \cellcolor[HTML]{FFD699}-4            & \cellcolor[HTML]{B5D4A7}5            & {\cellcolor[HTML]{D3E5CB}3}            & \cellcolor[HTML]{88BA73}8            & {\cellcolor[HTML]{B5D4A7}5}            & \cellcolor[HTML]{88BA73}8            & {\cellcolor[HTML]{C4DDB9}4}            & \cellcolor[HTML]{A6CB96}6            & {\cellcolor[HTML]{D3E5CB}3}            & \cellcolor[HTML]{88BA73}8            & {\cellcolor[HTML]{D3E5CB}3}            & \cellcolor[HTML]{BE9255}-3            \\ \hline
			middle\_char.          & \cellcolor[HTML]{97C384}7            & {\cellcolor[HTML]{97C384}7}            & \cellcolor[HTML]{88BA73}8            & {\cellcolor[HTML]{97C384}7}            & \cellcolor[HTML]{B5D4A7}5            & {\cellcolor[HTML]{88BA73}8}            & \cellcolor[HTML]{88BA73}8            & {\cellcolor[HTML]{88BA73}8}            & \cellcolor[HTML]{88BA73}8            & {\cellcolor[HTML]{88BA73}8}            & \cellcolor[HTML]{DAE7FD}2             & \cellcolor[HTML]{88BA73}8            & {\cellcolor[HTML]{88BA73}8}            & \cellcolor[HTML]{79B161}9            & {\cellcolor[HTML]{79B161}9}            & \cellcolor[HTML]{97C384}7            & {\cellcolor[HTML]{6AA84F}10}           & \cellcolor[HTML]{97C384}7            & {\cellcolor[HTML]{A6CB96}6}            & \cellcolor[HTML]{88BA73}8            & {\cellcolor[HTML]{6AA84F}10}           & \cellcolor[HTML]{858C9E}3             \\ \hline
			paired\_digits         & \cellcolor[HTML]{6AA84F}10           & {\cellcolor[HTML]{F7D7D5}-3}           & \cellcolor[HTML]{6AA84F}10           & {\cellcolor[HTML]{FAE4E3}-2}           & \cellcolor[HTML]{6AA84F}10           & {\cellcolor[HTML]{D3E5CB}3}            & \cellcolor[HTML]{88BA73}8            & {\cellcolor[HTML]{88BA73}8}            & \cellcolor[HTML]{6AA84F}10           & {\cellcolor[HTML]{FAE4E3}-2}           & \cellcolor[HTML]{FFAD33}-8            & \cellcolor[HTML]{79B161}9            & {\cellcolor[HTML]{C4DDB9}4}            & \cellcolor[HTML]{6AA84F}10           & {\cellcolor[HTML]{A6CB96}6}            & \cellcolor[HTML]{6AA84F}10           & {\cellcolor[HTML]{B5D4A7}5}            & \cellcolor[HTML]{79B161}9            & {\cellcolor[HTML]{97C384}7}            & \cellcolor[HTML]{6AA84F}10           & {\cellcolor[HTML]{C4DDB9}4}            & \cellcolor[HTML]{DA9530}-6            \\ \hline
			shopping\_list         & \cellcolor[HTML]{97C384}7            & {\cellcolor[HTML]{B5D4A7}5}            & \cellcolor[HTML]{A6CB96}6            & {\cellcolor[HTML]{B5D4A7}5}            & \cellcolor[HTML]{88BA73}8            & {\cellcolor[HTML]{88BA73}8}            & \cellcolor[HTML]{88BA73}8            & {\cellcolor[HTML]{88BA73}8}            & \cellcolor[HTML]{79B161}9            & {\cellcolor[HTML]{88BA73}8}            & \cellcolor[HTML]{FFE0B2}-3            & \cellcolor[HTML]{6AA84F}10           & {\cellcolor[HTML]{F1F7EE}1}            & \cellcolor[HTML]{6AA84F}10           & {\cellcolor[HTML]{E2EEDC}2}            & \cellcolor[HTML]{79B161}9            & {\cellcolor[HTML]{FAE4E3}-2}           & \cellcolor[HTML]{79B161}9            & {\cellcolor[HTML]{88BA73}8}            & \cellcolor[HTML]{6AA84F}10           & {\cellcolor[HTML]{D3E5CB}3}            & \cellcolor[HTML]{ED9718}-8            \\ \hline
			snow\_day              & \cellcolor[HTML]{97C384}7            & {\cellcolor[HTML]{F5CAC7}-4}           & \cellcolor[HTML]{A6CB96}6            & {\cellcolor[HTML]{EB968F}-8}           & \cellcolor[HTML]{79B161}9            & {\cellcolor[HTML]{B5D4A7}5}            & \cellcolor[HTML]{79B161}9            & {\cellcolor[HTML]{D3E5CB}3}            & \cellcolor[HTML]{A6CB96}6            & {\cellcolor[HTML]{EB968F}-8}           & \cellcolor[HTML]{FFA319}-9            & \cellcolor[HTML]{88BA73}8            & {\cellcolor[HTML]{C4DDB9}4}            & \cellcolor[HTML]{EB968F}-8           & {\cellcolor[HTML]{FCF1F1}-1}           & \cellcolor[HTML]{97C384}7            & {\cellcolor[HTML]{E2EEDC}2}            & \cellcolor[HTML]{D3E5CB}3            & {\cellcolor[HTML]{97C384}7}            & \cellcolor[HTML]{E88981}-9           & {\cellcolor[HTML]{FFFFFF}0}            & \cellcolor[HTML]{858C9E}3             \\ \hline
			solve\_boolean         & \cellcolor[HTML]{FCF1F1}-1           & {\cellcolor[HTML]{FFFFFF}0}            & \cellcolor[HTML]{FCF1F1}-1           & {\cellcolor[HTML]{FFFFFF}0}            & \cellcolor[HTML]{E2EEDC}2            & {\cellcolor[HTML]{D3E5CB}3}            & \cellcolor[HTML]{FAE4E3}-2           & {\cellcolor[HTML]{B5D4A7}5}            & \cellcolor[HTML]{FCF1F1}-1           & {\cellcolor[HTML]{FFFFFF}0}            & \cellcolor[HTML]{EDF3FE}1             & \cellcolor[HTML]{D3E5CB}3            & {\cellcolor[HTML]{C4DDB9}4}            & \cellcolor[HTML]{FFFFFF}0            & {\cellcolor[HTML]{C4DDB9}4}            & \cellcolor[HTML]{B5D4A7}5            & {\cellcolor[HTML]{88BA73}8}            & \cellcolor[HTML]{E2EEDC}2            & {\cellcolor[HTML]{F1F7EE}1}            & \cellcolor[HTML]{FAE4E3}-2           & {\cellcolor[HTML]{C4DDB9}4}            & \cellcolor[HTML]{8F8E92}2             \\ \hline
			spin\_words            & \cellcolor[HTML]{79B161}9            & {\cellcolor[HTML]{FFFFFF}0}            & \cellcolor[HTML]{88BA73}8            & {\cellcolor[HTML]{B5D4A7}5}            & \cellcolor[HTML]{88BA73}8            & {\cellcolor[HTML]{FAE4E3}-2}           & \cellcolor[HTML]{88BA73}8            & {\cellcolor[HTML]{FFFFFF}0}            & \cellcolor[HTML]{79B161}9            & {\cellcolor[HTML]{B5D4A7}5}            & \cellcolor[HTML]{FFB74C}-7            & \cellcolor[HTML]{6AA84F}10           & {\cellcolor[HTML]{B5D4A7}5}            & \cellcolor[HTML]{6AA84F}10           & {\cellcolor[HTML]{A6CB96}6}            & \cellcolor[HTML]{6AA84F}10           & {\cellcolor[HTML]{97C384}7}            & \cellcolor[HTML]{79B161}9            & {\cellcolor[HTML]{97C384}7}            & \cellcolor[HTML]{6AA84F}10           & {\cellcolor[HTML]{97C384}7}            & \cellcolor[HTML]{AB906D}-1            \\ \hline
			square\_digits         & \cellcolor[HTML]{88BA73}8            & {\cellcolor[HTML]{88BA73}8}            & \cellcolor[HTML]{88BA73}8            & {\cellcolor[HTML]{88BA73}8}            & \cellcolor[HTML]{79B161}9            & {\cellcolor[HTML]{88BA73}8}            & \cellcolor[HTML]{88BA73}8            & {\cellcolor[HTML]{88BA73}8}            & \cellcolor[HTML]{79B161}9            & {\cellcolor[HTML]{88BA73}8}            & \cellcolor[HTML]{FFE0B2}-3            & \cellcolor[HTML]{88BA73}8            & {\cellcolor[HTML]{88BA73}8}            & \cellcolor[HTML]{79B161}9            & {\cellcolor[HTML]{97C384}7}            & \cellcolor[HTML]{97C384}7            & {\cellcolor[HTML]{A6CB96}6}            & \cellcolor[HTML]{97C384}7            & {\cellcolor[HTML]{97C384}7}            & \cellcolor[HTML]{79B161}9            & {\cellcolor[HTML]{88BA73}8}            & \cellcolor[HTML]{E39624}-7            \\ \hline
			sub.\_cipher           & \cellcolor[HTML]{88BA73}8            & {\cellcolor[HTML]{B5D4A7}5}            & \cellcolor[HTML]{A6CB96}6            & {\cellcolor[HTML]{D3E5CB}3}            & \cellcolor[HTML]{79B161}9            & {\cellcolor[HTML]{B5D4A7}5}            & \cellcolor[HTML]{88BA73}8            & {\cellcolor[HTML]{B5D4A7}5}            & \cellcolor[HTML]{88BA73}8            & {\cellcolor[HTML]{E2EEDC}2}            & \cellcolor[HTML]{FFAD33}-8            & \cellcolor[HTML]{C4DDB9}4            & {\cellcolor[HTML]{79B161}9}            & \cellcolor[HTML]{79B161}9            & {\cellcolor[HTML]{C4DDB9}4}            & \cellcolor[HTML]{A6CB96}6            & {\cellcolor[HTML]{88BA73}8}            & \cellcolor[HTML]{E2EEDC}2            & {\cellcolor[HTML]{A6CB96}6}            & \cellcolor[HTML]{88BA73}8            & {\cellcolor[HTML]{B5D4A7}5}            & \cellcolor[HTML]{AB906D}-1            \\ \hline
			twitter                & \cellcolor[HTML]{79B161}9            & {\cellcolor[HTML]{D3E5CB}3}            & \cellcolor[HTML]{88BA73}8            & {\cellcolor[HTML]{B5D4A7}5}            & \cellcolor[HTML]{79B161}9            & {\cellcolor[HTML]{97C384}7}            & \cellcolor[HTML]{88BA73}8            & {\cellcolor[HTML]{88BA73}8}            & \cellcolor[HTML]{6AA84F}10           & {\cellcolor[HTML]{97C384}7}            & \cellcolor[HTML]{FFC166}-6            & \cellcolor[HTML]{6AA84F}10           & {\cellcolor[HTML]{88BA73}8}            & \cellcolor[HTML]{6AA84F}10           & {\cellcolor[HTML]{88BA73}8}            & \cellcolor[HTML]{6AA84F}10           & {\cellcolor[HTML]{97C384}7}            & \cellcolor[HTML]{6AA84F}10           & {\cellcolor[HTML]{79B161}9}            & \cellcolor[HTML]{6AA84F}10           & {\cellcolor[HTML]{88BA73}8}            & \cellcolor[HTML]{C79349}-4            \\ \hline
			vector\_dist.          & \cellcolor[HTML]{88BA73}8            & {\cellcolor[HTML]{C4DDB9}4}            & \cellcolor[HTML]{88BA73}8            & {\cellcolor[HTML]{C4DDB9}4}            & \cellcolor[HTML]{88BA73}8            & {\cellcolor[HTML]{79B161}9}            & \cellcolor[HTML]{88BA73}8            & {\cellcolor[HTML]{88BA73}8}            & \cellcolor[HTML]{79B161}9            & {\cellcolor[HTML]{C4DDB9}4}            & \cellcolor[HTML]{FFCC7F}-5            & \cellcolor[HTML]{A6CB96}6            & {\cellcolor[HTML]{E2EEDC}2}            & \cellcolor[HTML]{A6CB96}6            & {\cellcolor[HTML]{97C384}7}            & \cellcolor[HTML]{79B161}9            & {\cellcolor[HTML]{97C384}7}            & \cellcolor[HTML]{97C384}7            & {\cellcolor[HTML]{A6CB96}6}            & \cellcolor[HTML]{97C384}7            & {\cellcolor[HTML]{88BA73}8}            & \cellcolor[HTML]{AB906D}-1            \\ \hline
			average                & \cellcolor[HTML]{AACD9A}\textbf{5.8} & {\cellcolor[HTML]{DEECD8}\textbf{2.2}} & \cellcolor[HTML]{B4D4A7}\textbf{5.0} & {\cellcolor[HTML]{E2EEDD}\textbf{2.0}} & \cellcolor[HTML]{93C07F}\textbf{7.3} & {\cellcolor[HTML]{B5D4A7}\textbf{5.0}} & \cellcolor[HTML]{A1C890}\textbf{6.4} & {\cellcolor[HTML]{B0D1A1}\textbf{5.4}} & \cellcolor[HTML]{ACCF9D}\textbf{5.6} & {\cellcolor[HTML]{D8E9D1}\textbf{2.6}} & \cellcolor[HTML]{FFDDAC}\textbf{-3.2} & \cellcolor[HTML]{8FBE7B}\textbf{7.5} & {\cellcolor[HTML]{CAE1C1}\textbf{3.6}} & \cellcolor[HTML]{A0C88F}\textbf{6.4} & {\cellcolor[HTML]{AED0A0}\textbf{5.4}} & \cellcolor[HTML]{82B66C}\textbf{8.4} & {\cellcolor[HTML]{B6D4A8}\textbf{5.0}} & \cellcolor[HTML]{92BF7E}\textbf{7.4} & {\cellcolor[HTML]{A9CD99}\textbf{5.8}} & \cellcolor[HTML]{9DC68B}\textbf{6.6} & {\cellcolor[HTML]{B4D4A7}\textbf{5.0}} & \cellcolor[HTML]{BD9256}\textbf{-2.9} \\ \hline
			st. devition            & 3.7                                  & {4.3}                                  & 4.9                                  & {4.5}                                  & 2.4                                  & {3.2}                                  & 2.8                                  & {3.1}                                  & 5.5                                  & {5.8}                                  & 4.4                                   & 2.5                                  & {4.2}                                  & 4.6                                  & {3.5}                                  & 1.4                                  & {3.2}                                  & 2.3                                  & {2.7}                                  & 4.9                                  & {3.9}                                  & 3.5                                   \\ \hline
		\end{tabular}
	}
%	\end{adjustbox}
	\vspace{2px}
	\caption{Developer evaluation results for C++ and Java\\ (G = ChatGPT, P = Copilot, F.Impr.~= First Impression, Usab.~= Usability, Read.~=Readability, Modif.~= Modifiability, Acc.~=Acceptance, G/P = ChatGPT or Copilot is better)}
	\label{t:reviews}
	\vspace{-25px}
\end{table*}

As the previous sections presented the quality of the generated code, this section evaluates how that code was considered by experts.
Table \ref{t:reviews} shows the developer evaluation results for C++ and Java.
The values are summed scores\footnote{See supplementary material referenced in Section \ref{l:results}} for every property, each scoring from -2 to +2, therefore, with 5 developers for each language the lowest score is -10 (marked with red in the table) and the highest is +10 (marked green).
The developers not only had to score based on 5 properties, but had to decide which code was better (G/P column).
The negative values mean that ChatGPT was better according to the experts while the positive values denote the cases when Copilot was better.

\subsubsection{C++ results}
For C++, the results in Table ~\ref{t:reviews} show that ChatGPT is more welcome than Copilot.
The \textbf{First Impression} for ChatGPT on average over the tasks and developers is 5.8 with a 3.7 standard deviation while Copilot reached only 2.2 with a 4.27 standard deviation.
Looking at \textbf{Usability}, the average for ChatGPT was 5.0 and for Copilot it was 2.0.
The standard deviation for ChatGPT and Copilot were quite similar, 4.9 and 4.5.
It shows us that both models can generate quite usable code and also hard-to-use code according to our developers.
Regarding \textbf{Readability}, both models achieved quite good scores, in average ChatGPT scored 7.3 and Copilot reached 5.0.
The standard deviations were 2.4 and 3.2.
It shows that ChatGPT generates more readable code, slightly more frequently but Copilot's score is quite similar.
\textbf{Modifiability} scores are similar too in average, 6.4 for ChatGPT with a standard deviation of 2.8 and 5.4 for Copilot with 3.2 standard deviation.
The \textbf{Acceptance} has a larger difference.
ChatGPT reached 5.6 in average with a standard deviation of 5.5 and 2.6 for Copilot with standard deviation of 5.8.
It seems both models have accepted and rejected reviews and the large deviation shows that both models can generate very good and very bad code.

Although the pure numbers show that ChatGPT is more favorable among the developers, we investigated how these numbers would change if we included only the projects where the models reached good enough pass rate in random testing.
We did it for tasks where both models reached at least 75\% pass rate, for tasks where only ChatGPT scored this level of pass rate and for tasks where only Copilot did so.

Although the average ratings improved for both models, ChatGPT still performs better in every scenario but one.
Considering the tasks where only Copilot reached the minimum pass rate Copilot got better scores for modifiability.
From this we can conclude, although the developers had no information about the pass rates, they upscored the functional code and using this criteria ChatGPT is still favored over Copilot.

\subsubsection{Java results}

For Java, we did the same evaluations.
These results are also presented in Table \ref{t:reviews}.
Firstly, \textbf{First Impression} shows a great difference.
ChatGPT reached an average of 7.5 with a relatively small standard deviation of 2.5 and Copilot reachd on average 3.6 with standard deviation of 4.2.
It shows that ChatGPT-generated code is mostly favored by the developers while Copilot-generated code is less preferred.
Regarding \textbf{Usability}, the models are close to each other, ChatGPT scored 6.4 with standard deviation of 4.6 and Copilot reached 5.4 with standard deviation of 3.5.
On average, the generated code is usable, but it is varying for both models.
\textbf{Readability} also shows a great difference between the models.
ChatGPT reached 8.4 with standard deviation of 1.4, which means the ChatGPT generated code is consistently readable.
On the contrary, Copilot reached 5.0 with standard deviation of 3.2.
It shows that Copilot generated-code might be readable, but it is less readable.
Copilot generates more frequently unreadable code than ChatGPT.
Taking \textbf{Modifiability}, both models vary in the same manner, as the standard deviation for ChatGPT is 2.3 and for Copilot it is 2.7.
The average values are 7.4 and 5.8, which shows that the generated code can be modified quite easily, be it generated by ChatGPT or Copilot.
In \textbf{Acceptance}, there is no big difference compared to C++.
Both models are accepted in the same manner.
ChatGPT scored 6.6 while Copilot reached 5.0.
The standard deviations are 4.5 and 3.9, which means that both models generate code which is mostly accepted, but either of them can generate disliked code too.

In case of Java, we also tested if the pass ratio affects the results, thus we filtered again with the 75\% pass rate.
The filtered results show that ChatGPT is still favored over Copilot regarding the tasks where both models performed well on the functional testing.
Not surprisingly, ChatGPT is preferred over Copilot in tasks where only ChatGPT performed well in functional tests.
What is more surprising is that ChatGPT also has an advantage over Copilot in tasks where only Copilot performed well in the functional tests.

Regarding this case, the First Impression and Readability were much better for ChatGPT, which indicates that even if the code is bad, the style has a great impact on developers' opinion.
Usability values were similar, so developers considered bad ChatGPT code usable too.

After the human evaluations we can answer
\textbf{RQ 2: Is the generated source code accepted by experts?}
The developers approved the generated source code most of the time.
Developers did not consider the model-synthetized code perfect or out-of-the-box usable, but they considered the code acceptable.

\subsection{Aspects to consider}
With the final evaluations we can answer
\textbf{RQ 3: What aspects should be considered when choosing LLM-based generative tools?}

When choosing Large Language Model (LLM)-based generative tools, particularly for code generation, it is essential to consider a variety of aspects to ensure the model is both effective and efficient.
\begin{itemize}
	\item \textbf{High-Level Task Interpretation:} The LLM should be capable of understanding task definitions at a sufficiently abstract level. This means that the aim is to understand and effectively respond to developer requests with minimal input, ensuring both accuracy and efficiency.
	
	\item \textbf{Functional Validity Assessment:} Before considering a model suitable, it is critical to evaluate its functional validity. This involves testing the model's output in various scenarios, including both common use cases and edge cases. While edge case testing checks the model's performance under extreme or unusual conditions, providing a larger amount of random tests shows a better image of the general functionality. This comprehensive testing ensures that the model can reliably generate functional and robust code.
	
	\item \textbf{Technical Validity:} After establishing functional validity, the next step is to evaluate technical validity. This might involve performing static analysis on the generated code to detect any code smells, bugs, or unnecessarily complex solutions. The analyzer must be configured for the actual requirements of the final environment the model would be used in.
	
	\item \textbf{Human Evaluation:} Using the results from functional and technical validity assessments, you can compare different models with a high degree of confidence. Decisions can be based on these results as models might pass the given criteria, however, developers co-working with the model might discourage the usage as the generated source code does not fit their way of thinking. Developer reviews and evaluations play a crucial role in the model selection process, providing insights into each model's strengths and weaknesses. When conducting human reviews, one of the easiest ways is to provide scoring options, although it must be on an even scale to prevent neutral results. To prevent ties on certain properties a final decision should be made to decide which model-generated source code is better.
	
\end{itemize}

In summary, when selecting an LLM-based generative tool for code generation, it is essential to consider how well it understands high-level task definitions, its performance in functional and technical validity assessments, and factors related to human evaluation like first impression, usability, and modifiability.
These considerations help in choosing a tool that not only meets immediate coding needs but also integrates well into the broader development life cycle.

Although we included only the basic must-include parts, there could be scenarios where other aspects are important too.
In such scenarios, evaluators of the models should decide which criteria and techniques to use for assessing the quality of the generated source code.
Memory usage and time complexity are factors that could be important for specific applications, although they are not required in every evaluation.



\section{Threats to validity}
\label{l:threats_to_validity}
%\vspace{-5px}
Although we tried our best, there are still a few things that must be noted which might make readers doubt.

\textit{Model selection}: Both models are GPT-3 based models and we could have used other models.
Our main goal was to show how models could be compared properly, and from what aspects, which was done via an actual comparison.
We did choose from the most popular models, the ones which were already in use in the software engineering community.
Codex was trivial to use due to its plugin-like nature and ChatGPT was easy to use as it provided a handy web interface.
Although the selected models are similar and in the same lineage, the training sets and techniques are different.
ChatGPT is a later version, developed to be instruction-following and helpful for humans, while Codex is trained mainly for source code related tasks.

\textit{Language selection}: One might consider why we did not include a scripting language, like Python.
As we described the main aspects on language selection Python did not really fit those properties.
We assume that the results would be similar using Python as there is plenty of code (training data) written in Python.

\textit{Prompting}: As we could use a prompt that is designed for this purpose, we could not alter the inner workings of the models, such as randomness.
Although Copilot provides a possibility to set the temperature or top\_p values, the web interface for ChatGPT does not, thus in order to have a fair comparison we used default values for both.
Additionally, setting the temperature might not force the model to be deterministic enough as shown by Ouyang et al.~\cite{temperature_not_enough}.
Besides setting these values, a well-known technique is to not only select the first generated value but use more, frequently marked as @1 @5 for the first and the first five values.
As evaluating every task in both C++ and Java, e.g. 5 values would take too much human effort we decided to use @1 results.
Otherwise selecting from @5 values or interacting with further prompting would have required human interaction which would not be objective.
It is similar to interactive instructions for making the models' output better with human supervision.
Results would highly depend on the supervisor's expertise and ability to instruct LLMs.

The selected benchmark was published in July, 2021 therefore there is a slight chance that ChatGPT has already seen the tasks during training as it contains information up until September, 2021.
We cannot state such things from Codex as we have no insight into its real training data or dates.
Although it is a possible flaw, we consider the amount of text used in the benchmarks would be greatly outnumbered by the total training corpus therefore, it would not alter the results, furthermore, in real-life usage a specific benchmark usage resolves this problem.

\textit{Functional evaluation}: During the functional evaluation we used epsilon comparison for float comparisons.
Although float values are compared with epsilon values, the benchmark did not include such values.
The benchmark used in our evaluations could be changed but it was a key factor to have such a benchmark that is specially created for program generation or program synthesis.

\textit{Technical validity}: During static analysis, we did use metrics on C++ and Java, which languages support OOP.
The analyzed metrics did not include OOP-specific metrics as the tasks were way too small to properly utilize OOP.

\textit{Human evaluation}: We involved only 5-5 developers for both languages.
To overcome the low number of developers our main goal was to have professional reviewers qualified for such a role, so we preferred quality over quantity.
%\vspace{-15px}\section{Conclusions and Future Work}
\label{l:summary}


In this paper, we proposed a methodology for evaluating Large Language Models' (LLM) code synthesis capabilities to help developers choose the best available model.
This methodology takes into account that prompting is very important for LLMs and how detailed a usual task description is.
The primary concern is that the generated code works well, so it needs to be functionally tested. In addition, technical quality aspects are also important if the code is to be used in the long term. On the basis of this evaluation, a model can be chosen, but it is also worth asking the experts for their opinion.

We applied this methodology in a case study and evaluated and compared ChatGPT and Copilot on a publicly available code synthesis benchmark consisting of 25 tasks.
For prompting, we used the specifications in the benchmark, which were specifically designed to test program synthesis, and accepted the first generated solution.
We functionally tested the generated code with general and edge cases and found that the majority of the generated code was functionally correct.
We then used static code analysis to check the technical quality and found that despite minor errors, the models generated good-quality code.
Finally, we involved experts to review the solutions and their opinion supports our results.
Our conclusion is that both ChatGPT and Copilot can be used for program synthesis, but based on the comparison it seems that ChatGPT is better.

We consider the proper comparison of LLMs in code generation an important task which will be inevitable during industrial software development.
Based on this study we aim to create a framework in the future where LLMs can be compared in the most automatized way possible, including memory and time consumption values.
Using this comparison we aim to observe the preferred models of developers among the vast amount of models.
This opens the opportunity to investigate what model features are preferred providing more information to develop better models.
Using a decent comparing methodology we can also investigate the effects of various prompting techniques and fine-tunings, therefore, providing knowledge for improving models.

\section{Summary}

In this work, we introduce a methodology for comparing LLMs.
This methodology enables developers to evaluate models in depth and across a broad range of dimensions.
To illustrate its application, we present a use case involving two LLMs: ChatGPT and Codex.
The case study demonstrates that these models generally produce functionally correct code and adhere to good coding practices.
Although developers typically accept the generated code, it often requires additional refinement before final use.

%%%%%%%%%%%%%%%%%%%%%%%%%%%%%% COMMANDS FOR THESIS3 %%%%%%%%%

\definecolor{lightgraybox}{gray}{0.95}
\newenvironment{ieeepromptbox}
{\def\FrameCommand{\fboxrule=1pt \fcolorbox{black}{lightgraybox}}%
	\MakeFramed{\advance\hsize-\width \FrameRestore}%
	\noindent\textbf{LLM Prompt}\par\vspace{0.5em}}
{\endMakeFramed}

%%%%%%%%%%%%%%%% END OF COMMANDS THESIS3 %%%%%%%%%%%%%%%%%%%%%
\chapter{A Program Synthesis Dataset for LLM Temperature Analysis}
\label{chapter_4}

In the previous chapter, we introduced a methodology for comparing LLMs; however, it is equally important to examine individual models in greater detail.
Within the domain of code synthesis, the influence of temperature, a key hyper-parameter, on the outcomes remains insufficiently explored.
To study this effect, we require appropriate data, and therefore this section first presents a dataset well suited for temperature-based analysis.
Using this dataset, we then demonstrate a use case that provides a brief examination of temperature’s impact.

\section{Introduction}
\label{th4:section:introduction}

%Large Language Models (LLMs) became integral to software engineering research, addressing various tasks such as test case generation~\cite{llm_test_code_generation_survey}, vulnerability detection~\cite{llm_vuln_apr_survey}, automated program repair~\cite{llm_vuln_apr_survey}, source code comprehension~\cite{llm_code_comprehension_1, llm_code_comprehension_2}, and program synthesis~\cite{llm_code_generation_survey}.
Many studies evaluate LLMs at large scale; however, they often do not provide access to the generated outputs~\cite{data_no_share_1,data_no_share_2,data_no_share_3,data_no_share_4,data_no_share_5}.
While evaluation scripts are frequently shared, this raises several concerns.

First, since LLMs rely on probabilistic generation, their outputs can vary, making it difficult to reproduce the exact results reported in studies.
Second, which our paper is mostly built around, the computational cost of running these evaluations is substantial, requiring significant GPU resources.
By reusing previously generated LLM outputs, researchers can mitigate these computational demands, reducing energy consumption and enabling broader accessibility.
The hardware requirements are also considerable, as the evaluation of larger models necessitates access to specialized computational resources, such as high-memory GPUs, which may be beyond the reach of many researchers due to limited availability or prohibitive cost.

Our dataset, containing 18,900 raw and 18,896 processed LLM outputs, 
was produced in a study where various LLM families (Llama, Qwen, DeepSeek) were inferenced under various temperature settings.
This dataset enables researchers to analyze LLM outputs without the need to execute the models themselves.
Beyond energy efficiency, this approach also addresses hardware constraints, as some LLMs demand substantial GPU memory, which may not be available to all researchers.
By making these generated outputs accessible, we facilitate analysis even for those without the necessary computational resources.
Our dataset also addresses the limitations of well-known benchmarks by incorporating problems sourced from a programming competition, named Sapientia ECN\footnote{\url{https://ecn.ms.sapientia.ro/}}.
Although this is an international competition, its likelihood of being included in LLM training corpora is lower, reducing the risk of bias in model evaluation, thus making the generated LLM outputs relatively unique compared to the well-known benchmark results.

In addition to releasing the dataset, we also provide several potential use cases to illustrate its applicability in different contexts.
To demonstrate its practical value, we include a brief analysis that explores one specific scenario.
In this analysis, we focus on examining the effect of temperature on the success rate of the Qwen model family.
These results emphasize the need for a more comprehensive and systematic investigation to uncover the nuanced ways in which temperature influences performance.


\section{Related Work}
\label{th4:sec:related}

Generative AI became an integral part of software engineering especially with the raise of LLM, when the Transformer~\cite{attention} architecture was presented.
	LLMs are now widely adopted across many domains~\cite{llm_story}, including education, finance, and healthcare, where they support tasks ranging from personalized learning to risk assessment and medical decision support.
	LLMs are also extensively used and researched in software engineering.

A key application of LLMs in software engineering is program code synthesis, where a model generates source code based on a given natural language (NL) description.
This capability serves as the foundation for various software engineering tasks.
Schäfer et al.~\cite{unittest_llm} used LLMs to generate unit tests, Xia et al.~\cite{shared_common_data} performed automated program repair using LLMs, Song et al.~\cite{generation_llm} synthesized full programs.
Besides the results, researchers must consider the energy consumption of their LLM related research without reusing generated text.
	Samsi et al.~\cite{from_words_to_watts} measured the energy requirements of various models.
Given the computational cost of LLM-based code synthesis, reusing previously generated outputs can significantly enhance efficiency, reducing redundant computations.

Reusing generated output not only mitigates the energy requirements but also provides consistent research base.
	Although reusing generated texts could help researchers, there are works which did not share the results.
	Song et al.~\cite{data_no_share_1} generated projects for educational purposes.
	Xia et al.~\cite{data_no_share_2} used LLMs for automated program repair.
	While benchmarks provide fixed versions, therefore, the fixed versions are available, models often do not generate only the fixed versions, rather including extra information in the output from which the useful part should be extracted.
	Similarly, there are works~\cite{data_no_share_3,data_no_share_4,data_no_share_5} that do not share their valuable resources publicly.

Although some studies do publish LLM-generated outputs, these are often limited to the initial papers introducing a particular model.
Works with available LLM outputs frequently rely on well-established benchmarks, leading to repetitive LLM outputs such as outputs generated on HumanEval~\cite{codex}.
Xia et al.~\cite{shared_common_data} used multiple common datasets, such as Defects4J~\cite{defects4j}, QuixBugs~\cite{quixbugs}, and ManyBugs~\cite{manybugs_introclass}.
Li et al.~\cite{common_shared_llm_output} also used Defects4J.
A key limitation of using well-known benchmarks is the potential for biased evaluations.
Since LLMs are trained on vast text corpora that may include these benchmarks, their performance can be artificially inflated.
A notable example is HumanEval\footnote{\url{https://github.com/openai/human-eval}}, originally designed to assess OpenAI’s Codex~\cite{codex} model, which has since become a widely used benchmark in LLM evaluation.
Other widely used benchmark is for example Vul4J~\cite{vul4j} due to its proof of vulnerability tests.


\section{Methodology}
\label{th4:section:methods}

In this section, we describe the approach for selecting the models and programming tasks included in the dataset and also describe the framework and prompt engineering we used for inference.


\subsection{Inference overview}

We inferenced nine open-source LLMs on seven programming tasks from Sapientia ECN\footnote{\url{https://ecn.ms.sapientia.ro/}}. The descriptions of these programming tasks were embedded within a fixed prompt template (See in Section~\ref{th4:our_prompt}), ensuring consistency across all models and tasks.

Model inference was conducted using the Hugging Face Transformers library\footnote{\url{https://huggingface.co/docs/transformers/en/index}}, adhering to the recommended model configurations. The temperature parameter was systematically varied for each execution, with an incremental step of 0.01.
As a result, for every model and programming task we generated 100 outputs.
We performed the entire process three times.

\subsection{Model Selection}
\label{th4:section:model_selection}

Our methodology began with the selection of multiple large language models (LLMs).
We first curated a list of widely used open-source LLMs relevant to software-related research.
From this list, we selected the top three models based on their official performance results on the HumanEval~\cite{codex} and MBPP~\cite{mbpp} benchmarks.
To ensure a realistic assessment, we considered top-1 evaluation for both benchmarks.
Unlike top-k evaluation, which runs the model \texttt{k} times and selects the best outcome, thereby leveraging probabilistic variability, top-1 evaluation assesses the model based on a single execution per task.
This approach more accurately reflects real-world usage scenarios, where multiple attempts are often infeasible, rather than 10 or 100 evaluations.
The selected models and their corresponding benchmark results are summarized in Table~\ref{th4:table:model_list}.

We selected the top three from the collected models.
This process required evaluating model performance across both benchmarks while addressing cases where official results were unavailable for one of them.
We could not find an official evaluation on MBPP for multiple models, thus we assigned the worst value in order not to make a model better as it could be.
To achieve a balanced and principled selection, we employed the Pareto ranking method~\cite{pareto_usage}, which is widely used in economics to determine optimal trade-offs between two competing variables and even handles 0 values without completely biasing the result, such as average.
It can take into consideration lower values and promote options with lower value for one variable only if the option is outstanding in another variable.
The lower the rank is, the better the model performance is based on its combined score on the two benchmarks.
This approach not only identifies the best-performing models but also guarantees that the final selection is globally optimal with respect to the given evaluation criteria.
The Pareto ranks are also included in Table~\ref{th4:table:model_list}.
Based on the ranks, we selected \texttt{CodeQwen1.5-7B-Chat}~\cite{CodeQwen1.5-7B-Chat}, \texttt{deepseek-coder-33b-instruct}~\cite{deepseek-coder-33b-instruct}, \texttt{Meta-Llama-\\3-70B-Instruct}~\cite{Meta-Llama-3-70B-Instruct}.

\begin{table}[h]
	\centering
	\resizebox{0.99\textwidth}{!}{
		\begin{tabular}{|l|r|r|r|}
			\hline
			\textbf{Model Name}                    & \textbf{HumanEval Score} & \textbf{MBPP Score} & \textbf{Pareto Score} \\
			\hline
			\textbf{CodeQwen1.5-7B-Chat}           & 83.50                    & 77.70               & \textbf{1.0}           \\ \hline
			\textbf{deepseek-coder-33b-instruct}   & 79.30                    & 70.00               &\textbf{2.0}            \\ \hline
			\textbf{Meta-Llama-3-70B-Instruct}     & 81.70                    & 0.00                & \textbf{2.0}           \\ \hline
			DeepSeek-Coder-instruct-6.7B           & 78.60                    & 65.40               & 3.0                    \\ \hline
			SantaCoder-1.1B                        & 49.00                    & 68.00               & 3.0                    \\ \hline
			CodeLlama-instruct-70B                 & 67.80                    & 62.20               & 4.0                    \\ \hline
			Magicoder-S-DS-6.7B                    & 70.70                    & 62.00               & 4.0                    \\ \hline
			StarCoder(-prompted)                   & 33.60                    & 52.70               & 5.0                    \\ \hline
			WizardCoder                            & 57.30                    & 51.80               & 5.0                    \\ \hline
			CodeGen-Mono-16B                       & 29.28                    & 35.28               & 6.0                    \\ \hline
			Instruct-CodeGen                       & 37.10                    & 0.00                & 6.0                    \\ \hline
			CodeGeeX  13B                          & 22.89                    & 24.37               & 7.0                    \\ \hline
			CodeT5+-instruct-16B                   & 35.00                    & 0.00                & 7.0                    \\ \hline
			CodeGen2-7B                            & 19.09                    & 0.00                & 8.0                    \\ \hline
			InCoder                                & 15.00                    & 19.00               & 8.0                    \\ \hline
			PolyCoder-2.7B                         & 5.59                     & 0.00                & 9.0                    \\ \hline
		\end{tabular}
	}
	\caption{Performance of various models on HumanEval, MBPP, and their Pareto scores.}
	\label{th4:table:model_list}
\end{table}


While the selected top three models represent the best-performing base models, they belong to different model families and vary in size.
To provide a more comprehensive analysis, we extended our scope to include multiple model sizes within each family.

In addition to the top three models, we incorporated six additional models, ensuring a more holistic comparison across architectures (Llama, Qwen, DeepSeek) and model sizes.
The additionally selected model versions are as follows:
\begin{itemize}
	\item Llama family: \texttt{CodeLlama-13b-Instruct-hf}~\cite{CodeLlama-13b-Instruct-hf}, \texttt{Meta-Llama-3-8B-Instruct}~\cite{Meta-Llama-3-8B-Instruct}
	\item DeepSeek family: \texttt{deepseek-coder-6.7b-\\instruct}~\cite{deepseek-coder-6.7b-instruct}, \texttt{deepseek-llm-67b-chat}~\cite{deepseek-llm-67b-chat}
	\item Qwen family: \texttt{Qwen1.5-32B-Chat}~\cite{Qwen1.5-32B-Chat}, \texttt{Qwen1.5-\\72B-Chat}~\cite{Qwen1.5-72B-Chat}
\end{itemize}
\subsection{Programming Task Selection}
\label{th4:section:programming_task_selection}


As discussed in Section~\ref{th4:section:introduction}, widely used and well-established benchmarks may exhibit inherent biases.
Such biases can limit the generalizability of evaluation results, thereby reducing their effectiveness in assessing LLM behavior under diverse conditions.
Although most open-source models disclose details regarding their training datasets, this information is not always entirely accurate and may contain omissions.
An excellent example is the Llama model.
As statet by Meta the Llama-3 model ''Llama 3 is pretrained on over 15T tokens that were all collected from publicly available sources.''\footnote{\url{https://ai.meta.com/blog/meta-llama-3/}}

Initially, we considered leveraging tasks from prominent programming platforms such as LeetCode.
However, our investigation revealed that the publicly available problems and their corresponding test inputs are already incorporated into various existing benchmarks.
Consequently, we sought alternative, lesser-known programming challenges and identified a Transylvanian university that organizes annual programming competitions (Sapientia ECN\footnote{\url{https://ecn.ms.sapientia.ro/}}).
A distinguishing characteristic of these competitions is that the problems are manually crafted each year, ensuring uniqueness.

To systematically process and select appropriate algorithmic problems, we established the following criteria:
\begin{itemize}
	\item Each problem must include at least 10 test cases to enable a comprehensive evaluation across diverse scenarios.
	\item The problem descriptions should not rely on images for explanations, as LLMs are primarily optimized for text-based processing.
\end{itemize}

Furthermore, we deliberately selected an odd number of problems to facilitate decision-making in subsequent evaluation stages.
The final set of selected problems is presented in Table~\ref{th4:table:selected_problems}.
We manually categorized the programming tasks, based on the descriptions of the ECN rules\footnote{\url{https://ecn.ms.sapientia.ro/rules.php}}, to provide a broader overview of the tasks. 

\begin{table}[h]
	\centering
	\begin{tabular}{|l|c|c|}
		\hline
		\textbf{Selected problem} & \textbf{Problem type}  & \textbf{Number of testcases} \\ \hline
		2017-L                    &  Sorting and Searching & 16 \\ \hline
		2018-H                    &  Mathematics           & 20 \\ \hline
		2019-D                    &  Graph                 & 21 \\ \hline
		2019-J                    &  String Processing     & 13 \\ \hline
		2019-M                    &  Dynamic Programming   & 27 \\ \hline
		2022-B                    &  Dynamic Programming   & 16 \\ \hline
		2023-A                    &  Graph                 & 20 \\ \hline 
	\end{tabular}
	\caption{Selected problems from the ECN competition, identified by their respective year and problem number, along with their corresponding problem type and number of test cases. Original source:\url{https://ecn.ms.sapientia.ro/problems.php}}
	\label{th4:table:selected_problems}
\end{table}
We prompted the LLMs to solve these problems using C++.
As a strongly typed, compiled language, C++ enables the detection of not only semantic errors but also syntactic and certain logical errors at compile time.
This reduces the need for extensive runtime testing, as fundamental issues can be identified during compilation without executing the program.


\subsection{Inference Framework}
\label{th4:section:framework}

For our experiments, we selected HuggingFace Transformers\footnote{\url{https://huggingface.co/docs/transformers/en/index}}, a widely used framework for training and evaluating machine learning algorithms and models.
To ensure optimal performance, we followed the recommended configurations for each model, mostly defined by a config.json or described in the model sheet on HuggingFace.

The inferences were conducted on the Komondor supercomputer\footnote{\url{https://hpc.kifu.hu/hu/komondor}}, located in Hungary.
We were able to allocate four NVIDIA A100 GPUs for our experiments.
Additionally, we had access to two NVIDIA H100 GPUs, which significantly accelerated the inference of larger models.

\subsection{Prompt engineering}

Given that this dataset primarily investigates the effects of temperature on code synthesis, we did not introduce variations in prompts.
However, we employed fundamental prompt engineering techniques to enhance model responses.
The applied strategies include:
\begin{itemize}
	\item \textbf{Role Specification:} Providing a predefined role helps establish context for the model. This was implemented within the system prompt.
	\item \textbf{Structured Output Requests:} Enforcing a specific output format improves response predictability, facilitating more reliable result parsing.
	\item \textbf{Explanation Enforcement:} Requiring explanations in responses encourages the model to generate semantically coherent solutions.
	\item \textbf{Few-shot Learning:} When example input-output pairs were available within the task description, we incorporated them to guide the model behavior.
\end{itemize}
Each prompt inherently included the unique description of the corresponding programming task.
While different models may exhibit distinct sensitivities to prompt engineering techniques, we aimed to isolate temperature as the sole variable in our experiments.
Consequently, we used a standardized prompt across all models and temperature settings.

The final prompt template is provided alongside the dataset.
Although we did not introduce multiple prompt variations, the scripts are designed to accommodate distinct system prompts per model and task.

The prompt utilized in our experiments is as follows:\\
\begin{minipage}{0.99\textwidth}
	\label{th4:our_prompt}
	
	\begin{ieeepromptbox}
		Your task is to solve problems that are described as real-life likely scenarios, meaning you have to figure out the solution and implement it.
		You have to provide the solution C++ code between ` ` ` tags.
		Explain the solution why it solves the problem.
		Pay attention to possible edge cases.
		You will be provided with the description and a few input output examples.
		I will not alter your code so follow strictly how to read, in what format to read, the input and how to write the output.
		It is very important for me.
		
		The description:
		"""
		$<$TASK-DESCRIPTION$>$
		"""
		
		Example I/O:
		"""
		$<$EXAMPLE$>$
		"""
		
		Generate the solution in C++ between ` ` ` tags and an explanation why it works.
	\end{ieeepromptbox}
	
\end{minipage}


\section{Data Records}
\label{th4:section:data_records}

The dataset\cite{result_data} comprises raw and processed outputs derived from the inference of large language models (LLMs) across diverse model families, and parameter sizes.
Inferences were conducted under controlled conditions, with temperature values systematically varied within the interval $[0.01;1.00]$, using 0.01 increments.

The data is organized hierarchically to reflect the inference structure.
At the top level, the dataset is partitioned into three distinct model families.
Within each model family, there are three models differing primarily in size, with approximately 10 billion (10B), 30 billion (30B), and 70 billion (70B) parameters, respectively.
For each model, inference was repeated across three independent runs (denoted as run1, run2, and run3) to account for stochastic variability inherent in LLM outputs.
Each run directory contains subfolders corresponding to 100 distinct temperature settings.
Within each temperature setting, the data is further subdivided into seven task-specific folders, each representing a unique programming task identified by year-letter codes as described in Section~\ref{th4:section:programming_task_selection}.

Every task folder contains two artifacts:
\begin{itemize}
	\item Raw output file, which is the unprocessed output generated by the LLM;
	\item Processed file, containing curated outputs where only the source code segments have been extracted. This processing step was necessary as the original LLM responses included both source code and accompanying explanations, with the latter excluded to focus on code-specific analysis. From DeepSeek family 4 results did not contain any source code, therefore, we could not prepare cpp files for those, resulting in 18,896 processed files.
\end{itemize}

This meticulous structure facilitates efficient retrieval and comparison of results across model families, model sizes, temperature settings, and programming tasks. In total, the dataset encompasses:
\textbf{3} model families,
\textbf{3} model sizes per family,
\textbf{3} runs per model,
\textbf{100} temperature settings per run, and
\textbf{7} programming tasks per temperature setting.
In total, this yields \textbf{18,900} raw LLM outputs and an additional \textbf{18,896} corresponding processed files.

The data record also contains a JSON file (\texttt{temperature\\\_stats.json}), which contains test case results related to the generated programs.
We discuss these results in Section~\ref{th4:sec:results_profile}

\section{Technical Validation}
\label{th4:section:technical_validation}

The raw outputs remain entirely unaltered, preserving their integrity for reproducibility and unbiased analysis.

In contrast, the processed outputs were generated using a heuristic-based text processing pipeline designed to extract source code from the raw responses.
This approach, while efficient, introduces the potential risk of incomplete or erroneous parsing, where valid source code segments might not be correctly identified.
In four cases there are no cpp files, as the model did not generate any source code.
The list is available besides the scripts, named \texttt{result\_without\_source\_code.list}.

To validate the accuracy of the processed outputs, we attempted to compile each extracted C++ source code file using {\ttfamily\bfseries g++ (Debian 10.2.1-6) 10.2.1 20210110}.
Successful compilation was treated as an indicator of both syntactic correctness in the generated code and the successful parsing of the raw output, therefore, we only had to validate the cases with compilation error.
There were \textbf{3,959} instances where compilation failed.
These failures could stem from two primary sources: genuine compilation errors present in the LLM-generated code, or errors introduced during the heuristic processing phase.

To differentiate between these causes, we conducted a manual verification of \textbf{351} randomly selected cases from the failed compilations, ensuring a \textbf{95\%} confidence level with a \textbf{5\%} margin of error.
The manual evaluations showed no error in our process and all the compile errors were in fact errors generated by the models.
The full list of failing cases (\texttt{failing\_compilation.list}), along with the manually sampled ids (\texttt{random\_ids.list}), is provided within the shared dataset besides the scripts.

\section{Results profile and applicability}
\label{th4:sec:results_profile}

Following the compilation of the processed files intended for validation, we proceeded to execute the resulting binaries against the test cases supplied by Sapientia ECN\footnote{\url{https://ecn.ms.sapientia.ro/}}.
The number of results varied per binary depending on the specific task (see Table~\ref{th4:table:selected_problems}).
For easier reproducibility, test cases and the required outputs are collected in the \texttt{io} folder of the dataset.

Subsequent to execution, all observed outcomes were merged into a single structured data file which is named \texttt{temperature\_stats.json}.
This JSON file serves as a comprehensive repository, capturing detailed execution information down to the granularity of individual test cases.
The design of this dataset facilitates both fine-grained analysis and aggregated evaluations of system behavior under varying runtime conditions.

A summary of the high-level execution outcomes is presented in Table~\ref{th4:table:outcomes}.
These results offer a foundation upon which users of the dataset may conduct further analyses to identify and quantify behavioral patterns exhibited by the underlying language models.
Notably, the experiments were carried out using three independent inference runs per test case, configured to perform Top-1 generation.
Furthermore, temperature values were varied incrementally in steps of 0.01, thereby enabling precise control over the generation randomness.

This setup inherently supports multiple aggregation strategies.
For instance, one possible interpretation could be to consider the three inference runs as producing a Top-3 result set, thereby emphasizing diversity in successful outputs.
Alternatively, users may choose to re-aggregate the temperature steps using coarser intervals—such as merging several 0.01 increments into larger step sizes—to derive broader trends and reduce noise in the data.

\begin{table}[h]
	\centering
	\begin{tabular}{|l|c|}
		\hline
		\textbf{Outcome} & \textbf{Number}  \\ \hline
		Passed           & 6.900            \\ \hline
		Failed           & 18.755           \\ \hline
		Runtime Error    & 2.306            \\ \hline
		Timeout          & 1.921            \\ \hline
		\textbf{Sum}     & \textbf{29.882}  \\ \hline
	\end{tabular}
	\caption{Number of different outcomes during program executions on test cases.}
	\label{th4:table:outcomes}
\end{table}

As evident from the distribution in Table~\ref{th4:table:outcomes}, a significant portion of executions resulted in failures, with a smaller yet non-negligible number of runtime errors and timeouts.
These results, in conjunction with the detailed metadata preserved in the JSON file, enable robust downstream analysis and interpretation tailored to diverse research or evaluation objectives.
We list a few scenarios, not aiming for completeness, which our dataset can be used for, without any further inference:
\begin{itemize}
	\item \textbf{Simple code generation evaluations on non-standard coding tasks}: analyzing fail-pass rates, the characteristics of runtime errors or even the code similarities. With our dataset researchers just have to parse a JSON file and figure out relevant metrics. With the output files authors can perform in depth analysis on the generated texts as a series of tokens.
	
	\item \textbf{Analyzing task based behavior}: various tasks annotated with their problem types (see Table \ref{th4:table:selected_problems}) allows researchers to investigate various metrics, such as fail-pass rates or runtime/memory complexity, depending on the task type itself.
	
	\item \textbf{Deeper analysis of temperature}: there are papers \cite{temperature_optimization_2, temperature_optimization_3}which discuss the temperature setting for optimal generated code, although, it is not clearly stated whether it is required or even useful on finer grained scale. As our dataset contains a wide range of temperatures and 18.900 raw generations the effects of temperature can be analyzed at multiple granularities.
	
	\item \textbf{Investigation of the possible correlation between performance and the size or family of the model}: as our dataset contains multiple model families with multiple model sizes, it is possible to evaluate how different sized models behave and even to evaluate how different families perform on the same model size.
\end{itemize}


\section{Use case: analysis of temperature}
\label{th4:sec:use_case}

To illustrate the applicability of our dataset, especially the \texttt{temperature\_stats.json} file, we present a brief use case.
Specifically, we examine whether model performance depends on the temperature parameter.
Furthermore, we demonstrate that the results can be meaningfully aggregated by reducing the granularity of temperature sampling from the original step size of 0.01 to 0.1.
To get the required data, we simply load the provided JSON file and access the required elements.

For this analysis, we focus on the Qwen model family across two randomly selected benchmark tasks: Task 2019 D and Task 2022 B.
Our working hypothesis is that models exhibit comparable performance dynamics across tasks when the temperature parameter is varied.
To obtain the coarser 0.1 sampling, we compute the mean performance over every 10 consecutive values in the original dataset and mark it as the highest value in the interval.
For example, aggregating the interval [0.01;0.1] we average the values and assign it to the temperature value of 0.1.

Based on this aggregated dataset, we generate two plots (Figure~\ref{th4:fig:temperature_independence_averaged}), each corresponding to one of the tasks.
Each plot includes results for the three models from the Qwen family.
To evaluate the hypothesis, we conduct a comparative analysis of the two plots.
Divergent performance trends across the tasks would indicate that the models respond differently to temperature adjustments, thereby refuting the hypothesis.

\begin{figure}[!h]
	\centering
	\begin{subfigure}[t]{0.5\textwidth}
		\centering
		\includegraphics[height=1.7in]{Chapters/Thesis2/fig/usecase/rq1-averaged-Qwen-family-2019_D}
		\caption{Qwen models on task 2019 D}
	\end{subfigure}%
	\\
	\begin{subfigure}[t]{0.5\textwidth}
		\centering
		\includegraphics[height=1.7in]{Chapters/Thesis2/fig/usecase/rq1-averaged-Qwen-family-2022_B}
		\caption{Qwen models on task 2022 B}
	\end{subfigure}%
	\\
	
	\caption{Qwen models' performance on two programming tasks for different temperature values with a step size of 0.1.}
	\label{th4:fig:temperature_independence_averaged}
\end{figure}

In the plots, the X-axis represents the aggregated temperature values, while the Y-axis indicates the percentage of successfully solved test cases.
Although the absolute Y-axis scales differ between the two tasks, our analysis focuses on the relative changes in performance trends.

For the blue line (CodeQwen1.5-7B-Chat), the performance on chart (a) is relatively stable, with a slight peak around a temperature of 0.6.
In contrast, on chart (b) the curve exhibits multiple peaks, with a noticeable drop near 0.6 and a declining trend toward the end of the temperature range.
The orange line (Qwen1.5-32B-Chat) shows a high starting value on chart (a), after which performance drops and remains consistently low.
On chart (b), the curve begins at a low level, remains flat, and only increases slightly at the upper end of the temperature range.
Finally, the green line (Qwen1.5-72B-Chat) displays a similar initial trend in both charts, but the locations of the peaks differ between the two tasks, indicating task-dependent sensitivity to the temperature parameter.


\begin{framed}
	\noindent In our use case with two randomly selected tasks, all models from the Qwen family exhibit distinct performance trends as the temperature parameter varies.
		Since the observed trajectories differ across tasks, our initial hypothesis is not supported.
\end{framed}

A more comprehensive analysis would be needed to account for both fine and coarse-grained temperature settings and extend the comparison to additional model families.
Furthermore, employing more sophisticated similarity metrics would enable a more rigorous evaluation of performance dynamics.
Nevertheless, the presented case study is intended primarily as an illustrative example of how the dataset can be utilized, rather than as an in-depth analysis.
A thorough investigation along these lines would constitute separate future work.


\section{Threats to validity}
\label{th4:sec:threats_to_validity}

Although we did our best to have the most general and reliable results, there are still factors which could mislead our dataset, therefore, all the research based on this dataset.

\textit{Internal threats to validity.}
\\
\textbf{Source code:} our results were created using scripts that we created, therefore, it is possible that we made mistakes which alter the results.
The best we could do is have our code reviewed and we publish the scripts (see Section \ref{th4:section:code_availability}) so anyone can check its validity.
\\
\textbf{Badly processed results:} we did not only provide the raw outputs from the models, but also processed them.
This process involved heuristics, thus the results are not guaranteed to be perfect.
We did manual evaluation on a statistically significant scale and found no mistakes, however, this does not guarantee that there are no mistakes at all.

%Although our results are designed to provide data for researchers and we did not make results which should be generalized, researchers using our data might face the following 

\textit{External threats to validity.}
\\
\textbf{Model selection:} we used 3 model families with 3 different sizes which results in 9 models, it still cannot be considered large enough to cover every possible LLM, we did our best to cover the ones that might be used by other researchers who could select models based on popular benchmark scores.
We selected the models based on those scores using Pareto ranking, which is a commonly used method to take multiple factors into consideration.

\section{Code Availability}
\label{th4:section:code_availability}

For the generation of our dataset, we inferenced 9 distinct LLMs, and we provide full transparency by publishing the corresponding scripts along with the Python environment descriptors necessary for reproducibility.

The processed outputs were generated using a dedicated Python script, which is also included in the shared dataset.
All scripts are located within the scripts directory, organized alongside the top-level model-family folders.
This directory contains:
\begin{itemize}
	\item The Python scripts used for each LLM inference;
	\item The Python environment configuration files;
	\item The processing script responsible for extracting source code from the raw outputs;
	\item An additional Python script designed to automate the compilation of all parsed source code files;
	\item A Python script that evaluates the compiled binaries;
	\item The list of cpp files, which did not compile and the IDs of manually evaluated files.
\end{itemize}

The script for our use case are also included in order to help other authors by providing an example code that processes our dataset.
It is found in the \texttt{use\_case} folder.

\section{Summary}
\label{th4:sec:conclusion}

In this work we describe how we processed 18,900 raw code generations from Large Language Models, using fine-grained temperature settings during inference.
Before evaluation, we applied a pre-processing step to extract only the actual source code segments from the outputs.
This extraction step was manually validated, although not across the entire dataset.

The cleaned and compiled sources were then evaluated using test cases.
To make the results easily accessible, we exported all evaluation outcomes into a JSON file, enabling faster and more convenient reuse of the data.

We also outlined potential scenarios where this dataset could support further research, and showcased a short use case focused on temperature analysis.
Interestingly, the analysis revealed an unexpected finding: temperature does not have a general effect on the success rate of generated code.
This result highlights the value of our dataset, showing that it can serve as a strong foundation for deeper investigations into the behavior of LLMs.

%%%%%%%%%%%%%%%%%%%%%%%%%%%%%% COMMANDS FOR THESIS3 %%%%%%%%%

\definecolor{lightgraybox}{gray}{0.95}
\newenvironment{ieeepromptbox}
{\def\FrameCommand{\fboxrule=1pt \fcolorbox{black}{lightgraybox}}%
	\MakeFramed{\advance\hsize-\width \FrameRestore}%
	\noindent\textbf{LLM Prompt}\par\vspace{0.5em}}
{\endMakeFramed}

%%%%%%%%%%%%%%%% END OF COMMANDS THESIS3 %%%%%%%%%%%%%%%%%%%%%
\chapter{A Program Synthesis Dataset for LLM Temperature Analysis}
\label{chapter_4}

In the previous chapter, we introduced a methodology for comparing LLMs; however, it is equally important to examine individual models in greater detail.
Within the domain of code synthesis, the influence of temperature, a key hyper-parameter, on the outcomes remains insufficiently explored.
To study this effect, we require appropriate data, and therefore this section first presents a dataset well suited for temperature-based analysis.
Using this dataset, we then demonstrate a use case that provides a brief examination of temperature’s impact.

\section{Introduction}
\label{th4:section:introduction}

%Large Language Models (LLMs) became integral to software engineering research, addressing various tasks such as test case generation~\cite{llm_test_code_generation_survey}, vulnerability detection~\cite{llm_vuln_apr_survey}, automated program repair~\cite{llm_vuln_apr_survey}, source code comprehension~\cite{llm_code_comprehension_1, llm_code_comprehension_2}, and program synthesis~\cite{llm_code_generation_survey}.
Many studies evaluate LLMs at large scale; however, they often do not provide access to the generated outputs~\cite{data_no_share_1,data_no_share_2,data_no_share_3,data_no_share_4,data_no_share_5}.
While evaluation scripts are frequently shared, this raises several concerns.

First, since LLMs rely on probabilistic generation, their outputs can vary, making it difficult to reproduce the exact results reported in studies.
Second, which our paper is mostly built around, the computational cost of running these evaluations is substantial, requiring significant GPU resources.
By reusing previously generated LLM outputs, researchers can mitigate these computational demands, reducing energy consumption and enabling broader accessibility.
The hardware requirements are also considerable, as the evaluation of larger models necessitates access to specialized computational resources, such as high-memory GPUs, which may be beyond the reach of many researchers due to limited availability or prohibitive cost.

Our dataset, containing 18,900 raw and 18,896 processed LLM outputs, 
was produced in a study where various LLM families (Llama, Qwen, DeepSeek) were inferenced under various temperature settings.
This dataset enables researchers to analyze LLM outputs without the need to execute the models themselves.
Beyond energy efficiency, this approach also addresses hardware constraints, as some LLMs demand substantial GPU memory, which may not be available to all researchers.
By making these generated outputs accessible, we facilitate analysis even for those without the necessary computational resources.
Our dataset also addresses the limitations of well-known benchmarks by incorporating problems sourced from a programming competition, named Sapientia ECN\footnote{\url{https://ecn.ms.sapientia.ro/}}.
Although this is an international competition, its likelihood of being included in LLM training corpora is lower, reducing the risk of bias in model evaluation, thus making the generated LLM outputs relatively unique compared to the well-known benchmark results.

In addition to releasing the dataset, we also provide several potential use cases to illustrate its applicability in different contexts.
To demonstrate its practical value, we include a brief analysis that explores one specific scenario.
In this analysis, we focus on examining the effect of temperature on the success rate of the Qwen model family.
These results emphasize the need for a more comprehensive and systematic investigation to uncover the nuanced ways in which temperature influences performance.


\section{Related Work}
\label{th4:sec:related}

Generative AI became an integral part of software engineering especially with the raise of LLM, when the Transformer~\cite{attention} architecture was presented.
	LLMs are now widely adopted across many domains~\cite{llm_story}, including education, finance, and healthcare, where they support tasks ranging from personalized learning to risk assessment and medical decision support.
	LLMs are also extensively used and researched in software engineering.

A key application of LLMs in software engineering is program code synthesis, where a model generates source code based on a given natural language (NL) description.
This capability serves as the foundation for various software engineering tasks.
Schäfer et al.~\cite{unittest_llm} used LLMs to generate unit tests, Xia et al.~\cite{shared_common_data} performed automated program repair using LLMs, Song et al.~\cite{generation_llm} synthesized full programs.
Besides the results, researchers must consider the energy consumption of their LLM related research without reusing generated text.
	Samsi et al.~\cite{from_words_to_watts} measured the energy requirements of various models.
Given the computational cost of LLM-based code synthesis, reusing previously generated outputs can significantly enhance efficiency, reducing redundant computations.

Reusing generated output not only mitigates the energy requirements but also provides consistent research base.
	Although reusing generated texts could help researchers, there are works which did not share the results.
	Song et al.~\cite{data_no_share_1} generated projects for educational purposes.
	Xia et al.~\cite{data_no_share_2} used LLMs for automated program repair.
	While benchmarks provide fixed versions, therefore, the fixed versions are available, models often do not generate only the fixed versions, rather including extra information in the output from which the useful part should be extracted.
	Similarly, there are works~\cite{data_no_share_3,data_no_share_4,data_no_share_5} that do not share their valuable resources publicly.

Although some studies do publish LLM-generated outputs, these are often limited to the initial papers introducing a particular model.
Works with available LLM outputs frequently rely on well-established benchmarks, leading to repetitive LLM outputs such as outputs generated on HumanEval~\cite{codex}.
Xia et al.~\cite{shared_common_data} used multiple common datasets, such as Defects4J~\cite{defects4j}, QuixBugs~\cite{quixbugs}, and ManyBugs~\cite{manybugs_introclass}.
Li et al.~\cite{common_shared_llm_output} also used Defects4J.
A key limitation of using well-known benchmarks is the potential for biased evaluations.
Since LLMs are trained on vast text corpora that may include these benchmarks, their performance can be artificially inflated.
A notable example is HumanEval\footnote{\url{https://github.com/openai/human-eval}}, originally designed to assess OpenAI’s Codex~\cite{codex} model, which has since become a widely used benchmark in LLM evaluation.
Other widely used benchmark is for example Vul4J~\cite{vul4j} due to its proof of vulnerability tests.


\section{Methodology}
\label{th4:section:methods}

In this section, we describe the approach for selecting the models and programming tasks included in the dataset and also describe the framework and prompt engineering we used for inference.


\subsection{Inference overview}

We inferenced nine open-source LLMs on seven programming tasks from Sapientia ECN\footnote{\url{https://ecn.ms.sapientia.ro/}}. The descriptions of these programming tasks were embedded within a fixed prompt template (See in Section~\ref{th4:our_prompt}), ensuring consistency across all models and tasks.

Model inference was conducted using the Hugging Face Transformers library\footnote{\url{https://huggingface.co/docs/transformers/en/index}}, adhering to the recommended model configurations. The temperature parameter was systematically varied for each execution, with an incremental step of 0.01.
As a result, for every model and programming task we generated 100 outputs.
We performed the entire process three times.

\subsection{Model Selection}
\label{th4:section:model_selection}

Our methodology began with the selection of multiple large language models (LLMs).
We first curated a list of widely used open-source LLMs relevant to software-related research.
From this list, we selected the top three models based on their official performance results on the HumanEval~\cite{codex} and MBPP~\cite{mbpp} benchmarks.
To ensure a realistic assessment, we considered top-1 evaluation for both benchmarks.
Unlike top-k evaluation, which runs the model \texttt{k} times and selects the best outcome, thereby leveraging probabilistic variability, top-1 evaluation assesses the model based on a single execution per task.
This approach more accurately reflects real-world usage scenarios, where multiple attempts are often infeasible, rather than 10 or 100 evaluations.
The selected models and their corresponding benchmark results are summarized in Table~\ref{th4:table:model_list}.

We selected the top three from the collected models.
This process required evaluating model performance across both benchmarks while addressing cases where official results were unavailable for one of them.
We could not find an official evaluation on MBPP for multiple models, thus we assigned the worst value in order not to make a model better as it could be.
To achieve a balanced and principled selection, we employed the Pareto ranking method~\cite{pareto_usage}, which is widely used in economics to determine optimal trade-offs between two competing variables and even handles 0 values without completely biasing the result, such as average.
It can take into consideration lower values and promote options with lower value for one variable only if the option is outstanding in another variable.
The lower the rank is, the better the model performance is based on its combined score on the two benchmarks.
This approach not only identifies the best-performing models but also guarantees that the final selection is globally optimal with respect to the given evaluation criteria.
The Pareto ranks are also included in Table~\ref{th4:table:model_list}.
Based on the ranks, we selected \texttt{CodeQwen1.5-7B-Chat}~\cite{CodeQwen1.5-7B-Chat}, \texttt{deepseek-coder-33b-instruct}~\cite{deepseek-coder-33b-instruct}, \texttt{Meta-Llama-\\3-70B-Instruct}~\cite{Meta-Llama-3-70B-Instruct}.

\begin{table}[h]
	\centering
	\resizebox{0.99\textwidth}{!}{
		\begin{tabular}{|l|r|r|r|}
			\hline
			\textbf{Model Name}                    & \textbf{HumanEval Score} & \textbf{MBPP Score} & \textbf{Pareto Score} \\
			\hline
			\textbf{CodeQwen1.5-7B-Chat}           & 83.50                    & 77.70               & \textbf{1.0}           \\ \hline
			\textbf{deepseek-coder-33b-instruct}   & 79.30                    & 70.00               &\textbf{2.0}            \\ \hline
			\textbf{Meta-Llama-3-70B-Instruct}     & 81.70                    & 0.00                & \textbf{2.0}           \\ \hline
			DeepSeek-Coder-instruct-6.7B           & 78.60                    & 65.40               & 3.0                    \\ \hline
			SantaCoder-1.1B                        & 49.00                    & 68.00               & 3.0                    \\ \hline
			CodeLlama-instruct-70B                 & 67.80                    & 62.20               & 4.0                    \\ \hline
			Magicoder-S-DS-6.7B                    & 70.70                    & 62.00               & 4.0                    \\ \hline
			StarCoder(-prompted)                   & 33.60                    & 52.70               & 5.0                    \\ \hline
			WizardCoder                            & 57.30                    & 51.80               & 5.0                    \\ \hline
			CodeGen-Mono-16B                       & 29.28                    & 35.28               & 6.0                    \\ \hline
			Instruct-CodeGen                       & 37.10                    & 0.00                & 6.0                    \\ \hline
			CodeGeeX  13B                          & 22.89                    & 24.37               & 7.0                    \\ \hline
			CodeT5+-instruct-16B                   & 35.00                    & 0.00                & 7.0                    \\ \hline
			CodeGen2-7B                            & 19.09                    & 0.00                & 8.0                    \\ \hline
			InCoder                                & 15.00                    & 19.00               & 8.0                    \\ \hline
			PolyCoder-2.7B                         & 5.59                     & 0.00                & 9.0                    \\ \hline
		\end{tabular}
	}
	\caption{Performance of various models on HumanEval, MBPP, and their Pareto scores.}
	\label{th4:table:model_list}
\end{table}


While the selected top three models represent the best-performing base models, they belong to different model families and vary in size.
To provide a more comprehensive analysis, we extended our scope to include multiple model sizes within each family.

In addition to the top three models, we incorporated six additional models, ensuring a more holistic comparison across architectures (Llama, Qwen, DeepSeek) and model sizes.
The additionally selected model versions are as follows:
\begin{itemize}
	\item Llama family: \texttt{CodeLlama-13b-Instruct-hf}~\cite{CodeLlama-13b-Instruct-hf}, \texttt{Meta-Llama-3-8B-Instruct}~\cite{Meta-Llama-3-8B-Instruct}
	\item DeepSeek family: \texttt{deepseek-coder-6.7b-\\instruct}~\cite{deepseek-coder-6.7b-instruct}, \texttt{deepseek-llm-67b-chat}~\cite{deepseek-llm-67b-chat}
	\item Qwen family: \texttt{Qwen1.5-32B-Chat}~\cite{Qwen1.5-32B-Chat}, \texttt{Qwen1.5-\\72B-Chat}~\cite{Qwen1.5-72B-Chat}
\end{itemize}
\subsection{Programming Task Selection}
\label{th4:section:programming_task_selection}


As discussed in Section~\ref{th4:section:introduction}, widely used and well-established benchmarks may exhibit inherent biases.
Such biases can limit the generalizability of evaluation results, thereby reducing their effectiveness in assessing LLM behavior under diverse conditions.
Although most open-source models disclose details regarding their training datasets, this information is not always entirely accurate and may contain omissions.
An excellent example is the Llama model.
As statet by Meta the Llama-3 model ''Llama 3 is pretrained on over 15T tokens that were all collected from publicly available sources.''\footnote{\url{https://ai.meta.com/blog/meta-llama-3/}}

Initially, we considered leveraging tasks from prominent programming platforms such as LeetCode.
However, our investigation revealed that the publicly available problems and their corresponding test inputs are already incorporated into various existing benchmarks.
Consequently, we sought alternative, lesser-known programming challenges and identified a Transylvanian university that organizes annual programming competitions (Sapientia ECN\footnote{\url{https://ecn.ms.sapientia.ro/}}).
A distinguishing characteristic of these competitions is that the problems are manually crafted each year, ensuring uniqueness.

To systematically process and select appropriate algorithmic problems, we established the following criteria:
\begin{itemize}
	\item Each problem must include at least 10 test cases to enable a comprehensive evaluation across diverse scenarios.
	\item The problem descriptions should not rely on images for explanations, as LLMs are primarily optimized for text-based processing.
\end{itemize}

Furthermore, we deliberately selected an odd number of problems to facilitate decision-making in subsequent evaluation stages.
The final set of selected problems is presented in Table~\ref{th4:table:selected_problems}.
We manually categorized the programming tasks, based on the descriptions of the ECN rules\footnote{\url{https://ecn.ms.sapientia.ro/rules.php}}, to provide a broader overview of the tasks. 

\begin{table}[h]
	\centering
	\begin{tabular}{|l|c|c|}
		\hline
		\textbf{Selected problem} & \textbf{Problem type}  & \textbf{Number of testcases} \\ \hline
		2017-L                    &  Sorting and Searching & 16 \\ \hline
		2018-H                    &  Mathematics           & 20 \\ \hline
		2019-D                    &  Graph                 & 21 \\ \hline
		2019-J                    &  String Processing     & 13 \\ \hline
		2019-M                    &  Dynamic Programming   & 27 \\ \hline
		2022-B                    &  Dynamic Programming   & 16 \\ \hline
		2023-A                    &  Graph                 & 20 \\ \hline 
	\end{tabular}
	\caption{Selected problems from the ECN competition, identified by their respective year and problem number, along with their corresponding problem type and number of test cases. Original source:\url{https://ecn.ms.sapientia.ro/problems.php}}
	\label{th4:table:selected_problems}
\end{table}
We prompted the LLMs to solve these problems using C++.
As a strongly typed, compiled language, C++ enables the detection of not only semantic errors but also syntactic and certain logical errors at compile time.
This reduces the need for extensive runtime testing, as fundamental issues can be identified during compilation without executing the program.


\subsection{Inference Framework}
\label{th4:section:framework}

For our experiments, we selected HuggingFace Transformers\footnote{\url{https://huggingface.co/docs/transformers/en/index}}, a widely used framework for training and evaluating machine learning algorithms and models.
To ensure optimal performance, we followed the recommended configurations for each model, mostly defined by a config.json or described in the model sheet on HuggingFace.

The inferences were conducted on the Komondor supercomputer\footnote{\url{https://hpc.kifu.hu/hu/komondor}}, located in Hungary.
We were able to allocate four NVIDIA A100 GPUs for our experiments.
Additionally, we had access to two NVIDIA H100 GPUs, which significantly accelerated the inference of larger models.

\subsection{Prompt engineering}

Given that this dataset primarily investigates the effects of temperature on code synthesis, we did not introduce variations in prompts.
However, we employed fundamental prompt engineering techniques to enhance model responses.
The applied strategies include:
\begin{itemize}
	\item \textbf{Role Specification:} Providing a predefined role helps establish context for the model. This was implemented within the system prompt.
	\item \textbf{Structured Output Requests:} Enforcing a specific output format improves response predictability, facilitating more reliable result parsing.
	\item \textbf{Explanation Enforcement:} Requiring explanations in responses encourages the model to generate semantically coherent solutions.
	\item \textbf{Few-shot Learning:} When example input-output pairs were available within the task description, we incorporated them to guide the model behavior.
\end{itemize}
Each prompt inherently included the unique description of the corresponding programming task.
While different models may exhibit distinct sensitivities to prompt engineering techniques, we aimed to isolate temperature as the sole variable in our experiments.
Consequently, we used a standardized prompt across all models and temperature settings.

The final prompt template is provided alongside the dataset.
Although we did not introduce multiple prompt variations, the scripts are designed to accommodate distinct system prompts per model and task.

The prompt utilized in our experiments is as follows:\\
\begin{minipage}{0.99\textwidth}
	\label{th4:our_prompt}
	
	\begin{ieeepromptbox}
		Your task is to solve problems that are described as real-life likely scenarios, meaning you have to figure out the solution and implement it.
		You have to provide the solution C++ code between ` ` ` tags.
		Explain the solution why it solves the problem.
		Pay attention to possible edge cases.
		You will be provided with the description and a few input output examples.
		I will not alter your code so follow strictly how to read, in what format to read, the input and how to write the output.
		It is very important for me.
		
		The description:
		"""
		$<$TASK-DESCRIPTION$>$
		"""
		
		Example I/O:
		"""
		$<$EXAMPLE$>$
		"""
		
		Generate the solution in C++ between ` ` ` tags and an explanation why it works.
	\end{ieeepromptbox}
	
\end{minipage}


\section{Data Records}
\label{th4:section:data_records}

The dataset\cite{result_data} comprises raw and processed outputs derived from the inference of large language models (LLMs) across diverse model families, and parameter sizes.
Inferences were conducted under controlled conditions, with temperature values systematically varied within the interval $[0.01;1.00]$, using 0.01 increments.

The data is organized hierarchically to reflect the inference structure.
At the top level, the dataset is partitioned into three distinct model families.
Within each model family, there are three models differing primarily in size, with approximately 10 billion (10B), 30 billion (30B), and 70 billion (70B) parameters, respectively.
For each model, inference was repeated across three independent runs (denoted as run1, run2, and run3) to account for stochastic variability inherent in LLM outputs.
Each run directory contains subfolders corresponding to 100 distinct temperature settings.
Within each temperature setting, the data is further subdivided into seven task-specific folders, each representing a unique programming task identified by year-letter codes as described in Section~\ref{th4:section:programming_task_selection}.

Every task folder contains two artifacts:
\begin{itemize}
	\item Raw output file, which is the unprocessed output generated by the LLM;
	\item Processed file, containing curated outputs where only the source code segments have been extracted. This processing step was necessary as the original LLM responses included both source code and accompanying explanations, with the latter excluded to focus on code-specific analysis. From DeepSeek family 4 results did not contain any source code, therefore, we could not prepare cpp files for those, resulting in 18,896 processed files.
\end{itemize}

This meticulous structure facilitates efficient retrieval and comparison of results across model families, model sizes, temperature settings, and programming tasks. In total, the dataset encompasses:
\textbf{3} model families,
\textbf{3} model sizes per family,
\textbf{3} runs per model,
\textbf{100} temperature settings per run, and
\textbf{7} programming tasks per temperature setting.
In total, this yields \textbf{18,900} raw LLM outputs and an additional \textbf{18,896} corresponding processed files.

The data record also contains a JSON file (\texttt{temperature\\\_stats.json}), which contains test case results related to the generated programs.
We discuss these results in Section~\ref{th4:sec:results_profile}

\section{Technical Validation}
\label{th4:section:technical_validation}

The raw outputs remain entirely unaltered, preserving their integrity for reproducibility and unbiased analysis.

In contrast, the processed outputs were generated using a heuristic-based text processing pipeline designed to extract source code from the raw responses.
This approach, while efficient, introduces the potential risk of incomplete or erroneous parsing, where valid source code segments might not be correctly identified.
In four cases there are no cpp files, as the model did not generate any source code.
The list is available besides the scripts, named \texttt{result\_without\_source\_code.list}.

To validate the accuracy of the processed outputs, we attempted to compile each extracted C++ source code file using {\ttfamily\bfseries g++ (Debian 10.2.1-6) 10.2.1 20210110}.
Successful compilation was treated as an indicator of both syntactic correctness in the generated code and the successful parsing of the raw output, therefore, we only had to validate the cases with compilation error.
There were \textbf{3,959} instances where compilation failed.
These failures could stem from two primary sources: genuine compilation errors present in the LLM-generated code, or errors introduced during the heuristic processing phase.

To differentiate between these causes, we conducted a manual verification of \textbf{351} randomly selected cases from the failed compilations, ensuring a \textbf{95\%} confidence level with a \textbf{5\%} margin of error.
The manual evaluations showed no error in our process and all the compile errors were in fact errors generated by the models.
The full list of failing cases (\texttt{failing\_compilation.list}), along with the manually sampled ids (\texttt{random\_ids.list}), is provided within the shared dataset besides the scripts.

\section{Results profile and applicability}
\label{th4:sec:results_profile}

Following the compilation of the processed files intended for validation, we proceeded to execute the resulting binaries against the test cases supplied by Sapientia ECN\footnote{\url{https://ecn.ms.sapientia.ro/}}.
The number of results varied per binary depending on the specific task (see Table~\ref{th4:table:selected_problems}).
For easier reproducibility, test cases and the required outputs are collected in the \texttt{io} folder of the dataset.

Subsequent to execution, all observed outcomes were merged into a single structured data file which is named \texttt{temperature\_stats.json}.
This JSON file serves as a comprehensive repository, capturing detailed execution information down to the granularity of individual test cases.
The design of this dataset facilitates both fine-grained analysis and aggregated evaluations of system behavior under varying runtime conditions.

A summary of the high-level execution outcomes is presented in Table~\ref{th4:table:outcomes}.
These results offer a foundation upon which users of the dataset may conduct further analyses to identify and quantify behavioral patterns exhibited by the underlying language models.
Notably, the experiments were carried out using three independent inference runs per test case, configured to perform Top-1 generation.
Furthermore, temperature values were varied incrementally in steps of 0.01, thereby enabling precise control over the generation randomness.

This setup inherently supports multiple aggregation strategies.
For instance, one possible interpretation could be to consider the three inference runs as producing a Top-3 result set, thereby emphasizing diversity in successful outputs.
Alternatively, users may choose to re-aggregate the temperature steps using coarser intervals—such as merging several 0.01 increments into larger step sizes—to derive broader trends and reduce noise in the data.

\begin{table}[h]
	\centering
	\begin{tabular}{|l|c|}
		\hline
		\textbf{Outcome} & \textbf{Number}  \\ \hline
		Passed           & 6.900            \\ \hline
		Failed           & 18.755           \\ \hline
		Runtime Error    & 2.306            \\ \hline
		Timeout          & 1.921            \\ \hline
		\textbf{Sum}     & \textbf{29.882}  \\ \hline
	\end{tabular}
	\caption{Number of different outcomes during program executions on test cases.}
	\label{th4:table:outcomes}
\end{table}

As evident from the distribution in Table~\ref{th4:table:outcomes}, a significant portion of executions resulted in failures, with a smaller yet non-negligible number of runtime errors and timeouts.
These results, in conjunction with the detailed metadata preserved in the JSON file, enable robust downstream analysis and interpretation tailored to diverse research or evaluation objectives.
We list a few scenarios, not aiming for completeness, which our dataset can be used for, without any further inference:
\begin{itemize}
	\item \textbf{Simple code generation evaluations on non-standard coding tasks}: analyzing fail-pass rates, the characteristics of runtime errors or even the code similarities. With our dataset researchers just have to parse a JSON file and figure out relevant metrics. With the output files authors can perform in depth analysis on the generated texts as a series of tokens.
	
	\item \textbf{Analyzing task based behavior}: various tasks annotated with their problem types (see Table \ref{th4:table:selected_problems}) allows researchers to investigate various metrics, such as fail-pass rates or runtime/memory complexity, depending on the task type itself.
	
	\item \textbf{Deeper analysis of temperature}: there are papers \cite{temperature_optimization_2, temperature_optimization_3}which discuss the temperature setting for optimal generated code, although, it is not clearly stated whether it is required or even useful on finer grained scale. As our dataset contains a wide range of temperatures and 18.900 raw generations the effects of temperature can be analyzed at multiple granularities.
	
	\item \textbf{Investigation of the possible correlation between performance and the size or family of the model}: as our dataset contains multiple model families with multiple model sizes, it is possible to evaluate how different sized models behave and even to evaluate how different families perform on the same model size.
\end{itemize}


\section{Use case: analysis of temperature}
\label{th4:sec:use_case}

To illustrate the applicability of our dataset, especially the \texttt{temperature\_stats.json} file, we present a brief use case.
Specifically, we examine whether model performance depends on the temperature parameter.
Furthermore, we demonstrate that the results can be meaningfully aggregated by reducing the granularity of temperature sampling from the original step size of 0.01 to 0.1.
To get the required data, we simply load the provided JSON file and access the required elements.

For this analysis, we focus on the Qwen model family across two randomly selected benchmark tasks: Task 2019 D and Task 2022 B.
Our working hypothesis is that models exhibit comparable performance dynamics across tasks when the temperature parameter is varied.
To obtain the coarser 0.1 sampling, we compute the mean performance over every 10 consecutive values in the original dataset and mark it as the highest value in the interval.
For example, aggregating the interval [0.01;0.1] we average the values and assign it to the temperature value of 0.1.

Based on this aggregated dataset, we generate two plots (Figure~\ref{th4:fig:temperature_independence_averaged}), each corresponding to one of the tasks.
Each plot includes results for the three models from the Qwen family.
To evaluate the hypothesis, we conduct a comparative analysis of the two plots.
Divergent performance trends across the tasks would indicate that the models respond differently to temperature adjustments, thereby refuting the hypothesis.

\begin{figure}[!h]
	\centering
	\begin{subfigure}[t]{0.5\textwidth}
		\centering
		\includegraphics[height=1.7in]{Chapters/Thesis2/fig/usecase/rq1-averaged-Qwen-family-2019_D}
		\caption{Qwen models on task 2019 D}
	\end{subfigure}%
	\\
	\begin{subfigure}[t]{0.5\textwidth}
		\centering
		\includegraphics[height=1.7in]{Chapters/Thesis2/fig/usecase/rq1-averaged-Qwen-family-2022_B}
		\caption{Qwen models on task 2022 B}
	\end{subfigure}%
	\\
	
	\caption{Qwen models' performance on two programming tasks for different temperature values with a step size of 0.1.}
	\label{th4:fig:temperature_independence_averaged}
\end{figure}

In the plots, the X-axis represents the aggregated temperature values, while the Y-axis indicates the percentage of successfully solved test cases.
Although the absolute Y-axis scales differ between the two tasks, our analysis focuses on the relative changes in performance trends.

For the blue line (CodeQwen1.5-7B-Chat), the performance on chart (a) is relatively stable, with a slight peak around a temperature of 0.6.
In contrast, on chart (b) the curve exhibits multiple peaks, with a noticeable drop near 0.6 and a declining trend toward the end of the temperature range.
The orange line (Qwen1.5-32B-Chat) shows a high starting value on chart (a), after which performance drops and remains consistently low.
On chart (b), the curve begins at a low level, remains flat, and only increases slightly at the upper end of the temperature range.
Finally, the green line (Qwen1.5-72B-Chat) displays a similar initial trend in both charts, but the locations of the peaks differ between the two tasks, indicating task-dependent sensitivity to the temperature parameter.


\begin{framed}
	\noindent In our use case with two randomly selected tasks, all models from the Qwen family exhibit distinct performance trends as the temperature parameter varies.
		Since the observed trajectories differ across tasks, our initial hypothesis is not supported.
\end{framed}

A more comprehensive analysis would be needed to account for both fine and coarse-grained temperature settings and extend the comparison to additional model families.
Furthermore, employing more sophisticated similarity metrics would enable a more rigorous evaluation of performance dynamics.
Nevertheless, the presented case study is intended primarily as an illustrative example of how the dataset can be utilized, rather than as an in-depth analysis.
A thorough investigation along these lines would constitute separate future work.


\section{Threats to validity}
\label{th4:sec:threats_to_validity}

Although we did our best to have the most general and reliable results, there are still factors which could mislead our dataset, therefore, all the research based on this dataset.

\textit{Internal threats to validity.}
\\
\textbf{Source code:} our results were created using scripts that we created, therefore, it is possible that we made mistakes which alter the results.
The best we could do is have our code reviewed and we publish the scripts (see Section \ref{th4:section:code_availability}) so anyone can check its validity.
\\
\textbf{Badly processed results:} we did not only provide the raw outputs from the models, but also processed them.
This process involved heuristics, thus the results are not guaranteed to be perfect.
We did manual evaluation on a statistically significant scale and found no mistakes, however, this does not guarantee that there are no mistakes at all.

%Although our results are designed to provide data for researchers and we did not make results which should be generalized, researchers using our data might face the following 

\textit{External threats to validity.}
\\
\textbf{Model selection:} we used 3 model families with 3 different sizes which results in 9 models, it still cannot be considered large enough to cover every possible LLM, we did our best to cover the ones that might be used by other researchers who could select models based on popular benchmark scores.
We selected the models based on those scores using Pareto ranking, which is a commonly used method to take multiple factors into consideration.

\section{Code Availability}
\label{th4:section:code_availability}

For the generation of our dataset, we inferenced 9 distinct LLMs, and we provide full transparency by publishing the corresponding scripts along with the Python environment descriptors necessary for reproducibility.

The processed outputs were generated using a dedicated Python script, which is also included in the shared dataset.
All scripts are located within the scripts directory, organized alongside the top-level model-family folders.
This directory contains:
\begin{itemize}
	\item The Python scripts used for each LLM inference;
	\item The Python environment configuration files;
	\item The processing script responsible for extracting source code from the raw outputs;
	\item An additional Python script designed to automate the compilation of all parsed source code files;
	\item A Python script that evaluates the compiled binaries;
	\item The list of cpp files, which did not compile and the IDs of manually evaluated files.
\end{itemize}

The script for our use case are also included in order to help other authors by providing an example code that processes our dataset.
It is found in the \texttt{use\_case} folder.

\section{Summary}
\label{th4:sec:conclusion}

In this work we describe how we processed 18,900 raw code generations from Large Language Models, using fine-grained temperature settings during inference.
Before evaluation, we applied a pre-processing step to extract only the actual source code segments from the outputs.
This extraction step was manually validated, although not across the entire dataset.

The cleaned and compiled sources were then evaluated using test cases.
To make the results easily accessible, we exported all evaluation outcomes into a JSON file, enabling faster and more convenient reuse of the data.

We also outlined potential scenarios where this dataset could support further research, and showcased a short use case focused on temperature analysis.
Interestingly, the analysis revealed an unexpected finding: temperature does not have a general effect on the success rate of generated code.
This result highlights the value of our dataset, showing that it can serve as a strong foundation for deeper investigations into the behavior of LLMs.

% References:
\include{Chapters/references}

\backmatter

% Summary
\chapter*{Summary}
\markboth{Summary}{Summary}
\addcontentsline{toc}{chapter}{Summary}

Software engineering is a rapidly evolving field.
Developers have demonstrated, and will continue to demonstrate, a persistent need for source code analysis, both in direct and indirect forms.
Direct forms include tools that assess code quality or detect vulnerabilities, while indirect forms encompass applications that leverage such analyses internally, such as IDEs or automated program repair (APR) tools.

Large Language Models can be considered a type of static analysis tool, as they can operate directly on source code without requiring compilation.
LLMs are increasingly integrated into key software engineering processes, including design, documentation, and task management.
Although these models are applied in diverse areas, this thesis focuses specifically on a form of source code synthesis.

The aim of this thesis is to discover and deepen the connection between developers and LLMs as an ultimate development aid.
This thesis is designed to discover many aspects of LLMs which could be useful for a developer, regarding possible comparison techniques and the highlight of hyper-parameters.
This thesis also provides an actual status report not just an abstract idea on what the realistic expectations are.


\section*{Work I.}

In Chapter~\ref{chapter_2}, We created a comparison methodology based on various papers that compare LLMs.
This way We came-up with a methodology, that takes multiple aspects into consideration not only a few.
The methodology details how every step should be performed what must be taken into consideration.

The methodology starts with the right prompt selections.
In contrast to methods We are all got used to, LLMs cannot be used without proper prompting.
The first thing We must do, is to select the right prompt for the model, for the task We are going to use the model for.

After selecting the right prompt, we must evaluate the functional validity of the models.
Functional validity simply said, measures if the model can provide such code that does what it should do.
The hardest task at this point is to find a way, how can We measure ,,it does what it should do''.
It is most likely be performed via automatized tests, although it is important to note, that these tests might not take the generated code as is, so a framework might be needed.
Using a framework or not completely depends on the requirements, as the requirements might contain that the code must be testable by the existing tests.

Getting the tests is not an easy task either, as either it is up to the team to create the tests or use existing benchmarks, although at that case overfitting must be taken into consideration.
Either way, tests are not easy to get in good quality.

At the third step of the methodology, we must evaluate the non-functional validity or technical quality.
This step provides that the introduced LLM maintains the source code quality or rather increases it.

As our last step, We included the human evaluation, in order to see if developers are willing to accept source code generated by LLMs.
This step, although difficult and expensive to perform, provides a better overview if an LLM should be introduced to the developer team or not.

As an additional step, We done a case study to provide an actual example, how to use our methodology.
We also note, that additional evaluations might be added, such as complexity or performance tests, depending on the needs of the development team.

\section*{Work II.}

In Chapter~\ref{chapter_4}, We created a dataset, that contains 18,900 raw and 18,896 processed LLM outputs in C++ source code generation.
Inferencing LLMs especially larger ones requires special hardware which might not be avaliable for everyone who wants to research LLMs.
To broaden the possibilities of the research community, we evaluated 9 LLMs from 3 different LLM-families with various model sizes, including models with the size of 70 billion parameters.
These models were inferred with variadic temperature settings ranging from 0.01 to 1.00.

The creation of this dataset did not include randomly selected values.
We meticulously researched available open-source models and selected the best ones according  to multiple benchmarks.
As the models were selected with great attention, the tasks to be generated are also considered with great attention.

We needed tasks that provide a meaningful task for the LLMs, meaning they are not trivial to understand and implement.
These tasks also had to be hidden from the major benchmarks of LLMs, therefore, reducing the risk of overfitting.
To achieve this goal, we used tasks from Sapientia ECN, a programming competition.
This competition guarantees that the tasks are non-trivial and since the tasks are hand-made by the organizers the originality of the tasks is also guaranteed.

These tasks were provided to the LLMs alongside our prompt which we created considering good prompting practices such as role specification or few-shot learning.
The LLMs were inferred through our inference framework.

We also validated the generated outputs by compiling them.
Ones that compiled successfully were taken as successful generations, although ones that failed the compilations had to be checked.
We manually checked a statistically significant amount of those examples, and found that there were no mistakes in our framework, only the LLMs failed to generate source code that compiles.

The examples that could be compiled were also evaluated with the available testcases from the programming competition.
We collected these results in a JSON file which we published.

Using this JSON file We also presented, that the temperature hyper-parameter of LLMs does not hide a pattern for source code generation and this topic requires more research.

\section*{Work III.}

In Chapter~\ref{chapter_3}, We took an alternative form of source code generation and used to generate fixed source code to perform so called automatic program repair.
We used GPT-4 to generate the source code in a way that maintains the original functionality without the hidden vulnerability inside.

This task is evaluated on a real-life vulnerability benchmark, Vul4J.
This benchmarks contains tests that validate whether the plausible fix correctly removes the vulnerability or not.
To prevent our results to be influenced by GPT-4 being familiar with the vulnerabilities collected in the benchmark, We also used a benchmark, which had no validation tests, thus We could check correctness only by hand, that has vulnerabilities only after the training period of that particular GPT-4 version.

We created our prompt by considering multiple prompting strategies and we selected the best performing on a sub-set of Vul4J.
With our prompt, we prompted GPT-4 to fix the source code and also provide a textual answer.
This two sided request allowed us to discover if GPT can be a useful guide even if it cannot provide a valid vulnerability fix.
Our findings show that in real-world scenario the GPT-4 model can fix 33.33\% of the vulnerabilities  and can provide useful answers in around 50\% of the cases.

\newpage
\section*{Future Work}

Overall, this thesis highlights an important segment of the broader topic of LLMs in software engineering, while also indicating directions for future research.
In Chapter~\ref{chapter_2}, optional comparison criteria could be incorporated, and an automated system might be developed to perform functional and non-functional tests.
While human evaluation cannot be fully automated, a substantial portion of the process could be streamlined.

In Chapter~\ref{chapter_4}, the patterns associated with the temperature hyper parameter can be further explored at multiple levels.
This includes not only understanding its effects on model behavior but also determining the optimal granularity for its application.

Although Chapter~\ref{chapter_3} may appear complete, given that GPT-4 is now considered an older model, there remains room for further work.
Newer models could be evaluated, and improved inference techniques could be developed to achieve higher success rates.
While the one-third success rate observed is promising, real-world developers require even more reliable outcomes.

\vspace{5em}
To conclude this thesis, Large Language Models have become an integral part of software engineering, but they still require extensive research across multiple areas and levels to fully meet the needs of the developer community and the developer community needs to understand and adapt to these new tools.


\vfill
\pagebreak

\section*{Contributions of the thesis}
\renewcommand{\chapterautorefname}{Chapter}

In the \textbf{first thesis group}, the contributions are related to Large Language Model comparison in software synthesis. Detailed discussion can be found in \autoref{chapter_2}.

\begin{enumerate}[wide = 0pt, widest = {I/4.}, leftmargin =*]
	
	\item[I/1.] I collected relating methodologies and determined the gaps in those methodologies.
	
	\item[I/2.] I created the comparison methodology.
	
	\item[I/3.] I searched for an evaluation benchmark.
	
	\item[I/4.] I evaluated the Language Models on the benchmark.
	
	\item[I/5.] I performed the quality analysis of the synthetized programs.
	
	\item[I/6.] I designed the Human Evaluation.
	
	\item[I/7.] I created the Human Evaluation framework.
	
\end{enumerate}

\vspace{1cm}

\noindent
In the \textbf{second thesis group}, the contributions are related to creation of a dataset and the hyper parameter investigation of Large Language Models in software synthesis. Detailed discussion can be found in \autoref{chapter_4}.

\begin{enumerate}[wide = 0pt, widest = {II/5.}, leftmargin =*]
	
	\item[II/1.] I designed the study.
	
	\item[II/2.] I created the model selection framework.
	
	\item[II/3.] I created the model evaluation framework.
	
	\item[II/4.] I evaluated the data according to the temperature variable effect.
	
\end{enumerate}

\vspace{1cm}

\noindent
In the \textbf{third thesis group}, the contributions are related to the evaluation of GPT-4 in real-life vulnerability fixing. Detailed discussion can be found in \autoref{chapter_3}.

\begin{enumerate}[wide = 0pt, widest = {III/5.}, leftmargin =*]
	
	\item[III/1.] I searched for prompt engineering techniques.
	
	\item[III/2.] I designed the evaluation of the prompts and also the models.
	
	\item[III/3.] I participated in the manual evaluation of the generated source code.
	
	\item[III/4.] I participated in the manual evaluation of the generated textual responses.
	
\end{enumerate}

The author states that while the thesis results are primarily his own work, the
pronoun we is used instead of I to recognize the input of co-authors in the papers
forming the basis of this thesis.


\chapter*{Összefoglalás}
\markboth{Összefoglalás}{Összefoglalás}
\addcontentsline{toc}{chapter}{Összefoglalás}

A Szoftverfejlesztés egy gyorsan fejlődő terület.
A fejlesztőknek eddig is volt és ezután is lesz igényük a forráskód elemzésre, legyen ez az igény direkt vagy indirekt módon kifejezve.
Direkt mód alatt érthetjük a kód minőségének mérését vagy esetleg a sérülékenységek detektálását.
Az indirekt módba tartoznak azok az eszközök melyek a háttérben használják a forráskód elemzést, pl. IDE-k vagy automatikus kódjavító eszközök.

A nagy nyelvi modellek a statikus elemző eszközök egy speciális típusának te\-kint\-he\-tők, mivel elegendő közvetlenül a forráskódot megadni, fordítási lépés nélkül.
Az LLM-ek egyre inkább beépülnek a szoftverfejlesztés kulcsfontosságú folyamataiba, beleértve a tervezést, a dokumentációt és a feladatkezelést.
Bár ezek a modellek számos területen alkalmazhatók, jelen dolgozat kifejezetten a forráskód-szintézis egyik formájára fókuszál.

A dolgozat célja a fejlesztők és a nagy nyelvi modellek közötti kapcsolat feltárása és elmélyítése, hogy a nyelvi modellek a fejlesztők végső fejlesztéstámogató esz\-kö\-ze\-i\-vé válhassanak.
A disszertáció számos olyan aspektusát vizsgálja az LLM-eknek, amelyek a fejlesztők számára hasznosak lehetnek, különös tekintettel az összehasonlítási technikákra és a hiperparaméterek kiemelt szerepére.
A munka nem csupán elméleti megközelítést ad, hanem egy aktuális állapotjelentést is nyújt arról, hogy mik a reális elvárások ezekkel a modellekkel szemben.

\section*{Tézis I.}

A \ref{chapter_2}. fejezetben egy összehasonlításhoz használható metodólógiát alakítottunk ki azon fellelhető szakirodalom alapján, melyek valamilyen módon LLM-eket ha\-son\-lí\-tot\-tak össze.
Ezeket a cikkeket elemezve, olyan rendszert alakítottunk ki, mely segítségével két LLM teljeskörűen összehasonlítható.
Arra is kitérünk, hogy az egyes lépéseket hogyan kell kivitelezni és mire kell figyelmet fordítani.

A folyamat a megfelelő propmt kiválasztásával kezdődik.
A jólmegszokott esz\-kö\-zök\-kel szemben, az LLM-eket csakis a megfelelő prompt segítségével tudjuk a legjobb minőségben kihasználni, ezért fontos, hogy kiválasszuk a feladathoz és a modellhez illő legjobb promptot.

Miután kiválasztottuk a promptot, a modellek funkcionális tesztelése következik, azaz a modellek által generált kód valóban azt csinálja-e amit kell.
A legnehezebb feladat ebben a lépésben, hogy lemérjük, a kód valóban azt csinálja amit kell.
A legegyszerűbb módja az, ha automatizált teszteket használunk, azonban nagyon fontos figyelni arra, hogy ezek a tesztek gyakran nem tudják az LLM generálta kódot felhasználni, így valamilyen teszt keretrendszert kell alkalmaznunk.
Az, hogy kell-e kiegészítő rendszereket használnunk, módosítanunk a kódon attól függ, hogy mi az elvárt kimenet.
Ha a fealdat része, hogy a kódnak az adott tesztekkel kompatibilisnek kell lenniük, akkor nem szükséges további módosítást végeznünk.

A tesztek beszerzése sem egyszerű feladat, hiszen vagy a fejlesztő csapatnak kell saját teszteket alkotniuk, vagy már meglévő kódbázist használhatnak a tesztek során, ekkor azonban fennáll a lehetősége, hogy a modellek rátanulnak a publikusan elérhető kódokra.
Bármely megoldást is választjuk, megfelelő minőségű tesztesetet szerezni nem könnyű feladat.

Harmadik lépésként a nem funkcionális tesztelésen avagy a technikai mi\-nő\-ség\-mér\-és\-en van a sor.
Ez a lépés biztosítja azt, hogy az LLM által generált kód min\-ő\-ség\-é\-ben legalább olyan jó mint az eddig meglévő kódbázis.

Az utolsó lépésként az összehasonlítási lépésekhez bevettük az emberi kiértékelést, annak érdekében, hogy látható legyen, a fejlesztők hajlandók-e elfogadni a modellek által generált kódot.
Habár ez a lépés költséges és bonyolultan kivitelezhető, nagyon fontos, hiszen végsősoron a fejlesztőknek kell együttdolgozniuk a modellekkel.

A metodológia felállítása után egy esettanulmányban megmutattuk, hogyan is használatos a módszertanunk.
Megjegyezzük azt is, hogy extra kiértékelési szempontok is bevethetők, legyen az teljesítmény mérés vagy komplexitás vizsgálat, attól függően, a fejlesztőcsapatnak milyen igényei vannak.


\section*{Tézis II.}

A \ref{chapter_4}. fejezetben egy adathalmazt állítottunk elő, mely 18.900 fel\-dol\-go\-zat\-lan és 18.896 feldolgozott LLM által generált C++ kódot tartalmaz.
A nyelvi modellek futtatása, kifejezetten a nagyobbaké speciális hardvert igényel, mely nem mindenkinek áll rendelkezésre, aki ebben a témában tervez kutatni.
Annak érdekében, hogy minél nagyobb lehetőségei legyenek a kutatóközösségnek 9 nyelvi modellt futtattunk le, 3 modellcsalád mentés és több különféle méretben, mely méretek között a legnagyobbak nagyjából 70 miliárd paraméterrel rendelkeznek.
A kiértékelt modelleket változó temperature hiperparaméter értékkel futtattuk, melyet 0,01-től kezdődően egészen 1,00-ig állítottunk századnyi lépésközel.

Ennek az adatnak az előállítása nem véletlen értékek felhasználásával készült.
Alaposan megvizsgáltuk az elérhető szabadon felhasználható modelleket és a legjobbakat választottuk több referencia teszt alapján.
Ahogy a modellek választására, a generálandó feladatok kiválasztására is nagy figyelmet fordítottunk.

Olyan feladatokra volt szükségünk, melyek megfelelő nehézségűek, azaz nem triviálisak, de megfelelő módon értelmezhetők is.
Olyan feladatokra volt szükségünk, melyek a hirhedtebb referencia tesztbázisok elől rejtve maradtak, ezáltal a modellek feladatokra való rátanulását megelőzhettük.
ENnek érdekében a Sapientia ECN programozó verseny fealdatait használtuk fel.
Mivel ez egy programozó verseny, garantált, hogy nem triviálisak a fealdatok, és mivel a feladatokat a szervezők készítik, így a rátanuklásra is elenyésző az esély.

Ezek a feladatok kerültek a modellek promptjába a saját promptunk mellé, melyet úgy alkottunk meg, hogy figyelembe vettünk több jó promptolási gyakorlatot, a\-kár\-csak a szerepkör definiálás vagy a few-shot-learning, mely példákat ad a modellnek a promptban.
A modelleket a saját modellfuttató rendszerünket felhasználva futtattuk.

Az elkészült kimeneteket fordítással validáltuk.
A sikeresen forduló kimenetet sikeres modellfuttatásnak vettük, azonban azok melyek fordítási hibásak voltak, meg\-vizs\-gál\-tuk.
Kézzel statisztikailag szignifikáns mennyiségű példát megvizsgáltunk és azt találtuk, hogy nem a futtató rendszerünkben volt a hiba, hanem valóban a modellek fordítási hibás kódokat gyártottak.

A lefordítható példákat ki is értékeltük a verseny teszteseteinek segítségével.
Az eredményeket egy JSON fájlban gyűjtöttük össze, melyet közzé is tettünk.

Ennek a JSON fájlnak a felhasználásával prezentáltuk, hogyan használható az adatunk.
Ezt a temperature hiperparaméter vizsgálatával tettük meg, melyenek eredménye az, hogy nincsen jól elkülöníthatő optimális optimum kódgeneráláshoz, azonban ez a terület további kutatást igényel.

\section*{Tézis III.}

Az \ref{chapter_3}. fejezetben a forráskód-szintézis alternatív formáját használtuk, hogy megjavítsuk a hibás kódokat, ezáltal automatikus kódjavítást elérve.
A GPT-4 mo\-dell segítségével úgy generáltattuk a kódot, hogy az egy eredeti verzióból kiindulva megtartsa annak működését, de javítsa ki a benne rejlő sérülékenységet.

Ezt a Vul4J teszthalmazon értékeltük ki, mely valós életbeli sérülékenységet gyűj\-te\-mé\-nye.
Ez a teszthalmaz olyan tesztkódokat is tartalmaz, melyek a sérülékeny kódon elbuknak, de átmennek a javított változaton.
Annak érdekében, hogy minimalizáljuk azt a lehetőséget, hogy a GPT-4 modell rátanult a példákra, olyan teszthalmazt is kerestünk, melyben a sérülékenységek csakis a GPT-4 modell tanítási ideje után kerültek javításra.
Ennek a gyűjteménynek nem voltak automatikus tesztjei, így ezt csakis kézzel tudtuk kiértékelni.

A promptunkat több promptolási technika mentén elkészített prompthalmazból választottuk ki úgy, hogy a Vul4J tesztek egy részhalmazán a legjobban teljesítő promptot használtuk.
A végső prompt a modellt a kód javítására kérte meg, de emellett szöveges leírást is kért.
Ez a kettősség lehetőséget adott arra, hogy megvizsgáljuk, a GPT-4 modell képes-e hasznos tanácsokat adni abban az esetben is, ha nem tudja a forráskódot kijavítani.
Az eredmények azt mutatták, hogy a GPT-4 modell az esetek 33,33\%-ában sikeresen javítja ki a valós életbeli példákat és hasznos tanácsot az esetek közel felében képes adni.

\newpage
\section*{További kutatási irányok}

Összességében a dolgozat a nyelvi modellek a szoftverfejlesztésben átfogó té\-ma\-kö\-ré\-nek egy fontos részére világít rá, miközben iránymutatást nyúj a jövő kutatásainak.
A \ref{chapter_2}. fejezetben opcionális összehasonlítási teszteket lehet bevenni és egy autiomatikus kiértékelő rendszert is lehet fejleszteni, mely a modelleket a funkcionális és nem funkcionális elvárások alapján automatikusan összehasonlítja.
Habár az emberi té\-nye\-zőt nem lehet automatizálni, így is igen jelentős része a munkának meg\-kön\-nyít\-he\-tő.

A \ref{chapter_4}. fejezetben a temperature hiperparaméternek és a hozzákötődő mintáknak a vizsgálata elvégezhető több felbontásban.
Ez nem csak az optimális érték meg\-ha\-tá\-ro\-zá\-sát rejti magában, hanem annak meg\-ha\-tá\-ro\-zá\-sát is, hogy milyen mértékkel érdemes léptetni a paramétert.

Habár az \ref{chapter_3}. fejezet teljesnek tűnik, hiszen a GPT-4 modell már régebbi modellnek számít, így is van hely tovbbi kutatásnak.
Újabb modelleket lehet kiértékelni, fejlettebb és hatékpnyabb futtató módszereket lehet fejleszteni annak érdekében, hogy minél sikeresebb legyen az automatikus kódjavítás.
Habár az egyharmados javítási arány biztatónak tűnik, a valós fejlesztőknek ennél megbízhatóbb eszközre van szükségük.

\vspace{5em}
A dolgozat összegzéseként az mondható el, hogy a nagy nyelvi modellek a szoftverfejlesztés szerves részévé váltak, azonban még mindig rengeteg kutatást igényelnek több területen is, annak érdekében, hogy a fejlesztő közösség igényeihez igazodjanak és a fejlesztők is megértsék az új eszközöket és megtanuljanak azokkal együttdolgozni.


\vfill
\pagebreak

\section*{Tudományos hozzájárulások}

\noindent
Az első tézispontban a tudományos hozzájárulások a szoftverszintézishez használt nagy nyelvi modellek összehasonlításához kapcsolódnak.
Részletesebb kifejtés a \ref{chapter_2}. fejezetben található. 

\begin{enumerate}[wide = 0pt, widest = {I/4.}, leftmargin =*]
	
	\item[I/1.] Összegyűjtöttem a kapcsolódó módszertanokat, és azonosítottam azok hi\-á\-nyos\-sá\-ga\-it.
	
	\item[I/2.] Kidolgoztam az összehasonlítási módszertant.
	
	\item[I/3.] Kiválasztottam egy értékelési teszthalmazt.
	
	\item[I/4.] Kiértékeltem a nyelvi modelleket a teszthalmaz segítségével.
	
	\item[I/5.] Elvégeztem a szintetizált programok minőségi elemzését.
	
	\item[I/6.] Megterveztem az emberi értékelési folyamatot.
	
	\item[I/7.] Létrehoztam az emberi értékelési keretrendszert.
	
	
\end{enumerate}

\vspace{1cm}

\noindent
A második tézispontban a tudományos hozzájárulások egy adathalmaz el\-ké\-szí\-té\-sé\-hez és a nagy nelvi modellek szoftverszintéziskor használatos hiperparaméter vizs\-gá\-la\-tá\-hoz köthetők.
Részletesebb kifejtés a Chapter~\ref{chapter_4}. fejezetben található.

\begin{enumerate}[wide = 0pt, widest = {II/5.}, leftmargin =*]
	
	\item[II/1.] Megterveztem a kutatás keretét.
	
	\item[II/2.] Elkészítettem a modellválasztó keretrendszert.
	
	\item[II/3.] Elkészítettem a modell kiértékelő keretrendszert.
	
	\item[II/4.] Kiértékeltem az adatot a temperature hiperparaméterre fókuszálva.
	
\end{enumerate}

\vspace{1cm}

\noindent
A harmadik tézispontban a tudományos hozzájárulások a GPT-4 modell ké\-pes\-ség\-fel\-mé\-ré\-sé\-hez kapcsolódnak valós sérülékenységek automatikus javításakor.
Részletesebb kifejtés az \ref{chapter_3}. fejezetben található.

\begin{enumerate}[wide = 0pt, widest = {III/5.}, leftmargin =*]
	
	\item[III/1.] Promptolási technikákat gyűjtöttem.
	
	\item[III/2.] Megterveztem a promptok és modellek kiértékelését.
	
	\item[III/3.] Résztvettem a generált kódok kézi kiértékelésében.
	
	\item[III/4.] RÉsztvettem a generált szöveges válaszok kézi kiértékelésében.
	
\end{enumerate}

A szerző kijelenti, hogy bár a dolgozat eredményei elsősorban saját mun\-ká\-ján alapulnak, az ,,mi'' névmás használata az ,,én'' helyett a disszertáció alapjául szolgáló publikációk társszerzőinek hozzájárulását hivatott elismerni.

\chapter*{Publications}
\markboth{Publications}{Publications}
\addcontentsline{toc}{chapter}{Publications}

\vspace*{2em}

\subsection*{Journal publications}

\vspace*{1em}

\begin{itemize}[wide = 0pt, widest = {[4]}, leftmargin =*]


\item  \textbf{Zoltán Ságodi}, István Kolláth, Péter Heged\H{u}s and Rudolf Ferenc
\newblock {A Program Synthesis Dataset for LLM Temperature Analysis}.
\newblock \emph{IEEE Access}, VOL(13), 184022-184029, 2025.

\item  \textbf{Zoltán Ságodi}, István Siket, and Rudolf Ferenc
\newblock {Methodology for Code Synthesis Evaluation of LLMs Presented by a Case Study of ChatGPT and Copilot}.
\newblock \emph{IEEE Access}, VOL(12), 72303-72316, 2024.

\item  \textbf{Zoltán Ságodi}, Edit Pengő, Judit Juhász, István Siket and Rudolf Ferenc
\newblock {Static Call Graph Combination to Simulate Dynamic Call Graph Behavior}.
\newblock \emph{IEEE Access}, VOL(10), 131829-131840, 2022.

\item  Edit Pengő, \textbf{Zoltán Ságodi}
\newblock {A Preparation Guide for Java Call Graph Comparison}.
\newblock \emph{Acta Cybernetica}, VOL(24), 131–155, 2019.

\end{itemize}

\subsection*{Full papers in conference proceedings}
\vspace*{1em}

\begin{itemize}[wide = 0pt, widest = {[4]}, leftmargin =*]
	
\item  \textbf{Zoltán Ságodi}, Gábor Antal, Bence Bogenfürst, Martin Isztin, Péter Heged\H{u}s and Rudolf Ferenc
\newblock {Reality Check: Assessing GPT-4 in Fixing Real-World Software Vulnerabilities}.
\newblock In \emph{Proceedings of the 28th International Conference on Evaluation and Assessment in Software Engineering (EASE '24)}, Association for Computing Machinery, 252–261, 2024.

\item  Judit Jász, István Siket, Edit Pengő, \textbf{Zoltán Ságodi}, Rudolf Ferenc
\newblock {Systematic Comparison of Six Open-source Java Call Graph Construction Tools}.
\newblock In \emph{Proceedings of the 14th International Conference on Software Technologies - ICSOFT}, SciTePress, pages 117-128, 2019.

\end{itemize}

\chapter*{Declarations}
\markboth{Declarations}{Declarations}


\noindent \large{Statement on the Use of AI}

\begin{itemize}
	\item \textit{Language Models}: This document was edited with support from an AI language model, which contributed to text styling, and readability improvements.
	No AI-generated content was used beyond these editorial refinements. The AI langauge model in particular: ChatGPT Free-tier. (Version January 30, 2026.\footnote{\url{https://help.openai.com/en/articles/6825453-chatgpt-release-notes\#january-30-2026}})
\end{itemize}

\chapter*{Nyilatkozatok}
\markboth{Nyilatkozatok}{Nyilatkozatok}

\noindent \large{Nyilatkozat MI felhasználásáról}

\begin{itemize}
	\item \textit{Nyelvi modellek}: A dokumentum szerkesztése során egy mesterséges intelligencián alapuló nyelvi modell nyújtott támogatást, amely a szöveg stílusának és olvashatóságának javításához járult hozzá.
	A szerkesztési finomításokon túl nem került felhasználásra MI által generált tartalom.
	Az alkalmazott nyelvi modell: ChatGPT ingyenes verzió. (2026. január 30. verzió\footnote{\url{https://help.openai.com/en/articles/6825453-chatgpt-release-notes\#january-30-2026}})
\end{itemize}

% Acknowledgement
\include{ack}

% Bibliography:
\bibliographystyle{plain}
\addcontentsline{toc}{chapter}{Bibliography}
\bibliography{refs}

% Empty page
\thispagestyle{empty}

\end{document}  
