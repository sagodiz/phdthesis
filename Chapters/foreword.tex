

\thispagestyle{empty}
\vspace{4em}
\begin{flushright}
	\parbox[t]{10cm}{\selectedquote} \\
	\vspace{1em}
	\parbox[t]{5cm}{\raggedleft\small(\theguy)}
\end{flushright}


\vspace*{4em}
{\Huge \textbf{Foreword}}
\vspace{2em}

From a very young age, I found myself sitting in front of a computer, wondering not just what it could do, but how on earth someone had made it do that.
What began as curiosity quickly turned into a hobby, and that hobby, rather inconveniently for any other career plans, turned into my profession.

Spending so much time with code, I started to see computer science less as engineering and more as art.
Given an idea, a keyboard, and a palette of languages like C and C++, I could turn imagination into something tangible and occasionally even useful.
Few hobbies let you create worlds before lunch and makes you mad before dinner.

Choosing a path in high school was therefore not particularly dramatic: IT had already chosen me.
The only real question was how far I wanted to go.
At some point I decided that “far” meant a PhD.
From then on, that goal sat permanently in the back of my mind.
Ironically, the closer I came to it, the more it felt like the finish line was politely stepping away, just to keep things interesting.

As the distance grew, so did the amount of time I spent talking with people.
That turned out to be just as important as talking to machines, surprise.
I learned that research brings us new technology, and technology brings us comfort.
Comfort, when mixed with a little too much confidence, can quietly become our worst enemy — in research, in our jobs, and sometimes in life itself.

So yes, I happily use technology to its fullest.
But I also try to pause and ask whether common sense has an opinion before the machine does.

\vspace*{3em}
\textit{Zoltán Ságodi, 2025}