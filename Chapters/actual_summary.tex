Software engineering is a rapidly evolving field.
Developers have demonstrated, and will continue to demonstrate, a persistent need for source code analysis, both in direct and indirect forms.
Direct forms include tools that assess code quality or detect vulnerabilities, while indirect forms encompass applications that leverage such analyses internally, such as IDEs or automated program repair (APR) tools.

Large Language Models can be considered a type of static analysis tool, as they can operate directly on source code without requiring compilation.
LLMs are increasingly integrated into key software engineering processes, including design, documentation, and task management.
Although these models are applied in diverse areas, this thesis focuses specifically on a form of source code synthesis.

The aim of this thesis is to discover and deepen the connection between developers and LLMs as an ultimate development aid.
This thesis is designed to discover many aspects of LLMs which could be useful for a developer, regarding possible comparison techniques and the highlight of hyper-parameters.
This thesis also provides an actual status report not just an abstract idea on what the realistic expectations are.
